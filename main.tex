##DEPRECATED  
\documentclass{book}
\usepackage{geometry}       % helps mess with margins, currently breaking margin notes
\usepackage{tipa}           % IPA text
\usepackage{multirow}       % merges cells
\usepackage{longtable}      % multi-page tables
\usepackage{colortbl}       % colors in table lines
\usepackage{xeCJK}          % CJK Typesetting
\usepackage{pxrubrica}      % Ruby Text for JP
\usepackage{polyglossia}    % general multi-lang typesetting
\usepackage{phonrule}       % ???


\setdefaultlanguage{english}
\setotherlanguages{sanskrit,thai,lao,khmer}
\setmainfont{Noto Serif}
\setCJKmainfont{Noto Serif CJK KR}\CJKspace
\newfontfamily\devanagarifont[Script=Devanagari]{Noto Serif Devanagari}
\newfontfamily\thaifont{Noto Serif Thai}
\newfontfamily\laofont{Noto Serif Lao}
\newfontfamily\khmerfont{Noto Serif Khmer}
\definecolor{gray}{rgb}{.75,.75,.75}
\setlength{\marginparwidth}{1in}

\title{%
    Manmino\\
    \large An Auxlang of East Asia}
\author{Team Manmino}

\begin{document}
\maketitle\thispagestyle{empty} \frontmatter

\addcontentsline{toc}{chapter}{\textit{Somun}}
\chapter*{\textit{Somun}}
    \section{Manmino Somun} 
        \paragraph{}
        Manmino na Nitpon-bat Bama ne hwa Mongol-hwa Indonesya ji dayang hwa syangho-joyung ji Dong-Asya munhwagwen ji lenket wi ningong pojo eno cey. 
        Gyun heng-dek im'un Sysutem, 
        SOV-dek sap syunso,  gandan gowjaw, gandan nemjak-in dan bunsek-in munbap, 
        hwa Mun'enmun hwa Sansukulit hwa da ta Dong Asya ji eno-tat-bat lay-in sap, 
        ko jenbu siyong-in da imi da bawyu-in dan gandan siyong neng eno cey wan. 
        \paragraph{}
        Manmino ji mokdek na, ing'o bi da-dek siyong-em gandan gokdey Komyunikation ji Pulatfom, ko yu liksi-dek cungdot-tat lu caw'wet neng dong-Asya ji danhap,
        hwa itgay hokko ji dokjem dengji-em wi Dong Asya ji yeysut-dek hwa jidek hwangeng nay jeng'bu-dek mun'hwa-dek hwa eno-dek kampeng lu cawwat neng jiyu-dek gyaw'lyow cey.
        \paragraph{}
        Lakten-dek hoknin beknen apae ko la sap le: \lq\lq Bonlay-bat bamhap senggong cey-em hwa bundan baybok cey-em manmin ninsik-in jinsit cey\rq\rq.
        Team Manmino na Dong-Asya ji manmin ko eno ji haksip-em hwa siyong-em yu yeysut-dek hwa citek hepdong cangjak ji seng' gong-dek heng-em lu wan.
        
        \vfill \newpage
    \section{English Foreword} 
        \paragraph{} 
        Manmino is a constructed East Asian Auxiliary language intended to bridge the gap between the diverse but undeniably interconnected cultural region of East Asia, spanning from Japan to Myanmar and Mongolia to Indonesia. 
        Utilizing a phonetic inventory that balances ubiquity without overly favoring any one group, 
        a simple agglutinating yet analytic SOV structure with high structural flexibility, 
        and vocabulary drawing from a well-balanced mix of languages such as Classical Chinese, Sanskrit, and other regional languages, 
        Manmino aims to be a nuanced yet easy to use language for those in East Asia. 
        \paragraph{}
        The cause for which Manmino stands behind is fostering a sense of unity in East Asia that would triumph over historic tensions and rivalries in the region by creating a platform for international communication that is more easily accessible than English, 
        as well as fostering free flow of information and ideas in the region beyond government, cultural, or linguistic barriers in a manner that prevents the domination of any one particular entity in the region's artistic and intellectual environment.
        \paragraph{}
        As one idealistic Pan-Asianist said a hundred years ago, \lq\lq Success through union and defeat through disunion is a law of nature known to all\rq\rq; 
        Team Manmino hopes that through learning and using this language, people of East Asia would be able to successfully work together to create and share great feats of art and intellect. 
    \section{日本語の序文}
        \paragraph{}
        マンミン語は、日本からミャンマー、モンゴルからインドネシアまで、多様で相互に連結された東アジア文化圏を結ぶ人工国際補助言語(IAL)です。
        バランスのとれた音韻システム、主語-目的語-述語語順、柔軟な構造、簡単さ膠着的ながら分析的な文法、
        そして漢文、サンスクリット、その他多くの東アジアの言語でから来た語彙、
        これらすべてを活用するヌアンスながらも使いやすい言語になることを追求します。
        \paragraph{}
        万民語が追求する目的は、英語よりも使いやすい国際疎通のプラットフォーム、これを通じて歴史的葛藤を超越できる東アジアの団結、
        そして単一団体の独占を防ぐため、東アジアの芸術的および知的環境の中で政治的、文化的、そして言語的障害物を超越できる自由的交流です。
        \paragraph{}
        誰かが百年前に言った:【夫合成散敗、萬古常定之理也: \ruby{夫}{そ}れ\vspace{5pt}、
        \ruby{合}{あ}はば\ruby{成}{しょう}じ、\ruby{散}{ち}らば\ruby{敗}{はい}す、万古常定の\ruby{理}{ことわり}なり。】 
        チームマンミン語は、東アジアのすべての人々がこの言語を学習して使用することによって、
        術的・知的協同創作の成功を願っています。
        \vfill \newpage
    \section{한국어 서문} 
        \paragraph{}
        일본에서 미얀마까지, 몽골에서 인도네시아까지, 광대한 동아시아 문화권은 다양한 모습을 보이지만 서로 분명히 연결되어 있습니다. 만민어는 이러한 동아시아 문화권을 잇는 다리가 될 수 있도록 만들어진 인공 보조어입니다. 만민어의 목표는 풍부한 표현력을 잃지 않으면서도 동 아시아인들이 쉽게 쓸 수 있는 언어이며, 이를 위해 어느 한 집단에 지나치게 유리하지 않도록 보편성을 추구한 음운 목록, 유연한 문장 구조를 실현할 수 있으며 교착어와 분석어의 특징을 모두 갖는 간단한 SOV 구조, 그리고 한문, 산스크리트, 기타 지역어 등 다양한 언어에서 고르게 추출한 어휘를 채용하였습니다.
        \paragraph{}
        만민어의 이상은 동아시아 지역에서 영어보다 접근하기 쉬운 국제적 의사 소통의 플랫폼을 구축함으로써 역사적인 갈등과 대립 구도를 극복할 수 있는 연대 의식을 함양하는 것, 그리고 예술 및 지식 분야에서 어느 한 주체가 지배적인 지위를 차지하지 않도록 정부, 문화, 언어의 장벽을 넘어 정보와 사상이 자유롭게 유통될 수 있는 환경을 조성하는 것입니다.
        \paragraph{}
        백여 년 전, 한 이상적인 범아시아주의자가 이렇게 말했습니다 . \lq\lq \ruby{夫}{부}\ruby{合}{합}\ruby{成}{성}\ruby{散}{산}\ruby{敗}{패}, \ruby{萬}{만}\ruby{古}{고}\ruby{常}{상}\ruby{定}{정}\ruby{之}{지}\ruby{理}{리}\ruby{也}{야} —대저 합치면 성공하고 흩어지면 패망한다는 것은 만고에 분명히 정해져 있는 이치다.\rq\rq
        팀 만민어는 이 언어를 배우고 사용하는 것으로 동아시아의 사람들이 서로 힘을 합쳐 훌륭한 예술과 지식을 창조해 낼 수 있기를 바랍니다.
        
        
        
        
        
        
        
        
        
        
        
        
        \tableofcontents{}  
        \mainmatter 
\chapter{\textit{Ing'o} | English}
    \section{Alphabet and Sounds}
    \paragraph{}
    Manmino is written in the Latin Alphabet (\textit{Latin Aksala} \marginpar[]{\raggedright \scriptsize{\textbf{Aksala}: Manmino word, originating from Sanskrit. In Manmino, \textit{munji} can be used in their narrow sense to specify writing systems organized by syllables or mora (Such as Sinitic Script, Hangul, or Japanese Kana), while \textit{aksala} covers writing systems that are written in phonetic units (such as Alphabet or Devanagari). They are considered exchangeable synonyms but distinguished in strict contexts.}} or \textit{Latin Munji} in Manmino), a system familiar to all in East Asia and Southeast Asia. The Manimno Alphabet (\textit{Manmino Aksala}) uses the following letters:
    \begin{center}
    A B C D E F G H I J K L M N O P S T U W Y 
    
    a b c d e f g h i j k l m n o p s t u w y '
    \end{center}
    \paragraph{}
    Among the set, the following are consonants.
    \footnotesize{} 
    \begin{center}
    \begin{tabular}{ | c | c | c | c | c | c | c | } \hline 
    \multicolumn{2}{|c|}{} & Labial & Dental & Palatal & Velar & Glottal \\  \hline
    \multicolumn{2}{|c|}{Nasal}  & m & n & & ng [\textipa{N}]$^1$ & \\        \hline
    \multirow{2}{*}{Stops} & Fortis$^2$ & p [p\textsuperscript{h}] & t [t\textsuperscript{h}] & & k [k\textsuperscript{h}] & \multirow{2}{*}{' [\textipa{P}]$^3$} \\  \cline{2-6}
     & Lenis$^2$ & b & d & & \textipa{g}  &\\  \hline
    \multirow{2}{*}{Affricates} & Fortis$^2$ & & & c [\texttctclig{}\textsuperscript{h}] & & \\ \cline{2-7}
     & Lenis$^2$ & & & j [\textdctzlig{}] & & \\ \hline
    \multicolumn{2}{|c|}{Fricative} & f [\textipa{F}$\sim{}$f] & s & (\textipa{C})$^4$ &  & h [x $\sim{}$h] \\ \hline
    \multicolumn{2}{|c|}{Lateral} & & l & & & \\ \hline
    \multicolumn{2}{|c|}{Semivowel} & & & y [j] & w & \\ \hline
    \end{tabular}
    \end{center}
    \begin{enumerate} \marginpar[]{\raggedright \scriptsize{ \textbf{Phonology}: There is a specific method that was used to determine the phonology: After obtaining the full phonetic inventory of 20 major languages in greater East Asia, the phonetic inventory of each language was catalogued and assigned a value equal to the cube root of the language's speakers. Then all phonetic inventories were added together, and the phonemes that had a value exceeding the halfway point were added while the rest were culled. }}
      \item The phoneme [\textipa{N}] may not be used as an initial consonant. 
      \item The precise realization of Fortis consonants and Lenis consonants is provisionally left undefined due to split demographic preferences in greater East Asia.
      \item The phoneme [\textipa{P}] may only occur between syllables. 
      \item $\langle{}$s$\rangle{}$ is pronounced as [\textipa{C}] when preceding $\langle{}$\textipa{y}$\rangle{}$ or $\langle{}$\textipa{i}$\rangle{}$.
      \item Among all consonants in Manmino, only the nasals [\textipa{m, n, N}], non-glottal Fortis stops [\textipa{p, t, k}], and the lateral [l] can be used as coda consonants.
    \end{enumerate}\normalsize{}
    \paragraph{}
    Other letters of the Latin Alphabet not included here such as \textit{q, x, z}, may be used to write words not defined in  Manmino, such as proper nouns. In such cases, the word as a whole should be pronounced as accurate to the language of origin as possible. 
    \footnotesize{}
    \begin{center}
    \begin{tabular}{ | c | c | c | c | c | c | c | c | } \hline
    \multicolumn{2}{|c|}{\multirow{2}{*}{}} & \multicolumn{2}{|c|}{+\O} & \multicolumn{2}{|c|}{+[w]$^1$} & \multicolumn{2}{|c|}{+[j]$^1$} \\ \cline{3-8}
    \multicolumn{2}{|c|}{} & Front & Rear & Front & Rear & Front & Rear \\ \hline
    \multicolumn{2}{|c|}{Closed} & i & u & yu [ju] &   &   & wi \\ \hline
    \multirow{3}{*}{Middle} & \O+ & e & o & ew & ow & ey [ej] & oy [oj] \\ \cline{2-8}
    & [j]$^2$+ & ye [je] & yo [jo] & yew [jew] & yow [jow] & yey [jej] &  \\ \cline{2-8}
    & [w]+ & we & & & & wey [wej] & \\ \hline
    \multirow{3}{*}{Open} & \O+ & \multicolumn{2}{|c|}{a} & \multicolumn{2}{|c|}{aw} & \multicolumn{2}{|c|}{ay [aj]} \\ \cline{2-8}
    & [j]+ & \multicolumn{2}{|c|}{ya [ja]} & \multicolumn{2}{|c|}{yaw [jaw]} & \multicolumn{2}{|c|}{yay [jaj]} \\ \cline{2-8}
    & [w]+ & \multicolumn{2}{|c|}{wa} & \multicolumn{2}{|c|}{waw} & \multicolumn{2}{|c|}{way [waj]} \\ \hline
    
    
    \end{tabular} 
    \end{center}\marginpar[\footnotesize \raggedright \textbf{Vowels}: There was actually a sizable support for intro- ducing \lbrack \textipa{@}\rbrack {} as $\langle{}$e$\rangle{}$ (which would have moved \lbrack e\rbrack {} as $\langle{}$ye$\rangle{}$), but ultimately, it was slightly below the halfway point and was culled. Phonemic tones were culled in similar fashion as well.]{}
    \begin{enumerate}
      \item Consonants cannot come after $\langle{}$w$\rangle{}$ or $\langle{}$y$\rangle{}$, but they can come after $\langle{}$u$\rangle{}$ or $\langle{}$i$\rangle{}$.
      \item In front of the vowel $\langle{}$e$\rangle{}$,  $\langle{}$y$\rangle{}$is only allowed as a initial consonant and not as a glide. The only exception is after $\langle{}$s$\rangle{}$.
      
    \end{enumerate} 
    \normalsize{}
    \paragraph{}
    Using the phonemes listed above, Manmino utilizes C$_1$VC$_2$ syllable structure: with C$_1$ excluding [\textipa{N}] and [\textipa{P}]; V being any of the short vowel, diphthong, or triphthong listed above save for the listed exceptions; and C$_2$ being limited to the nasals [\textipa{m, n, N}], non-glottal Fortis stops [\textipa{p, t, k}], and the lateral [l]. As mentioned earlier, [\textipa{P}] occurs in between syllables rather than as part of any one syllable. The glottal stop is only necessary in disambiguation, and therefore may be considered mostly non-phonemic in common usage. 
    \paragraph{}
    In order to accommodate a wider array of speakers and to facilitate faster speech, Manmino allows flexible allophonic pronunciation of certain diagraphs. Those rules are as follows:
        \begin{itemize}
            \item \textbf{The Nasal Coda Rule:} If a nasal coda $\langle{}$m, n, ng$\rangle{}$ is followed by a plosive or affricate of a different place of articulation, the nasal coda may be articulated at the place of articulation for the following consonant instead of the one for its original place of articulation. For example, the digraph $\langle{}$mk$\rangle{}$ can be pronounced as [\textipa{N}k] rather than [mk]. For this rule, Palatal consonants are considered a subset of Dental consonants. .
            \item \textbf{The Stop Coda Rule:} If a stop coda $\langle{}$p, t, k$\rangle{}$ is followed by one of the lenis consonant $\langle{}$b, d, j, g$\rangle{}$, the coda can be dropped in exchange for pronouncing the following consonant as its fortis counterpart. 
            \item \textbf{The $\langle{}$ph$\rangle{}$ Rule:} The diagraph $\langle{}$ph$\rangle{}$ may be pronounced as [f].
            \item \textbf{The $\langle{}$ty/dy$\rangle{}$ Rule: }The digraphs $\langle{}$ty$\rangle{}$ and $\langle{}$dy$\rangle{}$ may be pronounced as [\texttctclig{}] and [\textdctzlig{}] respectively.
            \item \textbf{The $\langle{}$kh$\rangle{}$ Rule: }In the diagraph $\langle{}$kh$\rangle{}$, the [k] may be dropped.
        \end{itemize}
    In case of any confusion, it is recommended that you clarify your pronunciation to the best of your ability by either using dummy vowels or simply not utilizing the above diagraph rules in favor of speaking slower.
    \vfill \newpage 
    \section{Particles and Word Function}
            \paragraph{}
        In Manmino, the function of a word as a noun, verb, adjective or adverb is primarily identified through position and particles both pre-positional and post-positional. As such, in Manmino, words in isolation are classified not as nouns, verbs, and modifiers, but objects and concepts, actions and phenomena, and qualities. Their function as noun, verb, adjective, or adverb is determined by the particles around them. 
        \subsection{Nouns}
            \paragraph{}
            Manmino nouns are notable for not accounting for number. Plurality in Manmino can be emphasized with either reduplication of the noun or with the suffix \textit{-tat}. However, Manmino nouns are marked with the following post-positional particles that determine the function of the noun:
            \marginpar[h]{\footnotesize \raggedright \textbf{na}: From Japanese wa and Korean 은/는.}                  \marginpar[h]{\footnotesize \raggedright \textbf{lu}: From Japanese wo and Korean 을/를.}                      \marginpar[h]{\footnotesize \raggedright \textbf{ji}: From Classical Chinese 之.}                            \marginpar[h]{\footnotesize \raggedright \textbf{bat}: From Classical Chinese 發.}                         \marginpar[h]{\footnotesize \raggedright \textbf{bang}: From Classical Chinese 方.}                       \marginpar[h]{\footnotesize \raggedright \textbf{nay}: From Classical Chinese 內.}                         \marginpar[h]{\footnotesize \raggedright \textbf{ne}\textsuperscript{1 2}: Both from Japanese ni and Korean 에.}                 \marginpar[h]{\footnotesize \raggedright \textbf{em}: From Malay/Indonesian \textit{an} and Korean 임.}           \marginpar[h]{\footnotesize \raggedright \textbf{la}: From Korean 라.}  
            
            \footnotesize{}\begin{center}
            \begin{tabular}{ c | c | l }  
            \textbf{Particle} & \textbf{Function} & \textbf{Description} \\ \hline
            \multirow{2}{*}{\textit{na}} & \multirow{2}{*}{Topic} & \multirow{2}{*}{Marks the noun or noun phrase as the topic of the sentence.} \cr{} & & \\ \hline
              & \multirow{2}{*}{Nominative} & \multirow{2}{*}{In Manmino, the subject is unmarked.} \cr{} & & \\ \hline
            \multirow{3}{*}{\textit{lu}} & \multirow{3}{*}{Accusative} & Marks the noun as the direct object of the sentence. \cr{} & & Optional when noun is in between subject and verb, \cr{} & & otherwise required. \\ \hline
            \multirow{2}{*}{\textit{ji}} & \multirow{2}{*}{Genitive} & Marks the noun as the possessor of the next item in the \cr{} & & noun phrase. \\ \hline
            \multirow{2}{*}{\textit{bat}} & \multirow{2}{*}{Ablative} & Marks the noun as the origin of a verb.\cr{} & & Equivalent to "from". \\ \hline
            \multirow{2}{*}{\textit{bang}} & \multirow{2}{*}{Lative} & Marks the noun as the destination of a verb.\cr{} & & Equivalent to "to" in reference to location. \\ \hline
            \multirow{2}{*}{\textit{nay}} & \multirow{2}{*}{Locative} &  Marks the noun as the location of the verb.\cr{} & & Equivalent to "in" or "at".\\ \hline
            \multirow{2}{*}{\textit{ne}\textsuperscript{1}} & \multirow{2}{*}{Dative} & Marks the noun as the indirect object of the verb. \cr{} & & Equivalent to "to" in reference to a target of action.\\ \hline
            \multirow{2}{*}{\textit{ne}\textsuperscript{2}} & \multirow{2}{*}{Agent} & Marks the noun as the agent in a passive voice construction. \cr{} & & Equivalent to prepositional "by". \\ \hline
            \multirow{2}{*}{\textit{em}} & \multirow{2}{*}{Verbal Noun} & A particle that marks a word that usually functions as a  \cr{} & &quality or an action or phenomena as being a noun. \\ \hline
            \multirow{2}{*}{\textit{la}} & \multirow{2}{*}{Quoting} & \multirow{2}{*}{Marks the noun phrase as a quote.} \cr{} & &
            \end{tabular}
            \end{center}\normalsize
            \vfill \newpage
            \paragraph{}
            Manmino verbs are augmented with particles for multiple aspects, conveying information in the following order.\marginpar[]{\footnotesize \raggedright \textbf{Verb System}: The stacking verb modification system comes from Korean, albeit in a simplified form. This was done so to increase flexibility in how a word can be used.}
            
            \footnotesize{}\begin{center}
            \begin{tabular}{ | c | c| c| c| c| c | c | }  \hline
            Passive & Polite & Imperative$^1$ & Progressive & Time$^2$ & Completion$^3$ & Final \\ \hline
            \multirow{6}{*}{\textit{li}} & \multirow{6}{*}{\textit{si}} & \multirow{6}{*}{\textit{ye}} & \multirow{6}{*}{\textit{jung}} & \multirow{2}{*}{Past} & \multirow{2}{*}{Perfect} & Emphatic\\
               &    &    &      & \multirow{2}{*}{\textit{le}} & \multirow{2}{*}{\textit{le}} & \textit{ya} \\ \cline{7-7}
            & & & &  &  & Interrogative\\\cline{5-6}
            & & & & \multirow{2}{*}{Future}  & \multirow{2}{*}{Prospective}  & \textit{ka} \\ \cline{7-7}
            & & & & \multirow{2}{*}{\textit{kalu}} & \multirow{2}{*}{\textit{kalu}} & Suggestive\\
            & & & &  &  & \textit{ne} \\ \hline
            
            \end{tabular}
            \end{center}
            \begin{enumerate}
                \item The imperative particle is incompatible with other particles that follow it.
                \item Tense may be implied.
                \item Completion Marking Particles only function in context of a pre-existing tense-marking Particle. 
            \end{enumerate}\normalsize{}
            \paragraph{}
            Manmino Modifiers always prefix the word they modify and are emphasized by reduplication. Depending on the basic nature of the word (either as Action, Quality, or Object), a Modifier may need certain suffixes to clarify function. If the word that normally describes an object as opposed to a quality is in the place of an adjective, the word must be followed by the particle \textit{dek}, while if in the place of an adverb, the word must be followed by the particle \textit{syang}. If a word that normally describes an action or phenomenon is in the place of an adjective, the word must be followed by the particle \textit{in} if perfective and \textit{ji} otherwise, and in the place of an adverb, it must use \textit{syang}. 
            \paragraph{}
            Negation are the only affixes used in Manmino. Verbs are negated by the affix \textit{but}, and adjectives are negated by the affix \textit{bi}. A statement can be denied in Manmino by the phrase \textit{but cey}, or by its abbreviated form \textit{bey}. 
            \paragraph{}
            There are also particles that function independently from words (instead being attached to phrases as a whole) These can either function as conjunctions, or mark the function of a sentence. Note that clauses are generally modified by the particle placed ahead of it.
                \begin{center} \footnotesize{}
                    \begin{tabular}{|c|c|l|} \hline
                        Particle & Function & Description \\ \hline
                        \multirow{2}{*}{\textit{i}} & Coordinating & Equivalent to \textit{and}. \\ & (Additive - Clausal) & Sets two clauses as co-ordinate to each other. \\ \hline
                        \multirow{2}{*}{\textit{hwa}} & Coordinating & A particle that ties the word with the following word.  \cr{} & (Additive - Noun Phrase) & As an exception, this can tie verbs (but not verb phrases) as well. \\ \hline
                        \multirow{2}{*}{\textit{hok}} & Coordinating & Equivalent to \textit{or}. \\ & (Alternative) & Sets two words or clauses as alternative to one another. \\ \hline
                        \multirow{2}{*}{\textit{pi}} & Coordinating & Equivalent to \textit{versus} or \textit{as opposed to}.\\ & (Comparative) & Sets two clauses as objects of comparison to one another. \\ \hline
                        \multirow{2}{*}{\textit{nyak}} & Subordinating & Equivalent to \textit{if}.\\ & (Hypothetical) & Marks the following clause as a hypothetical statement.\\ \hline
                        \multirow{2}{*}{\textit{dan}} & Subordinating & Equivalent to \textit{but}, \textit{however}, or \textit{if only}. \\  & (Exceptional) & Marks the following clause as an exceptional statement. \\ \hline
                        \multirow{2}{*}{\textit{yu}} & Subordinating & Equivalent to \textit{because}. \\ & (Causal) & Marks the following clause as the cause of a cause - effect relationship. \\ \hline
                        \multirow{3}{*}{\textit{wi}} & \multirow{2}{*}{Subordinating} & Equivalent to \textit{for}.\\  & \multirow{2}{*}{(Objectival)} & Marks the following clause as the objective of an objective - method \\&&relationship.  \\ \hline
                        \end{tabular}
                \end{center} 
                \vfill\newpage
    \section{Pronouns}
    \paragraph{}
    Personal Pronouns in Manmino are as follows on the left side. They are collectively from Classical Chinese, except for the plural marker which is from the Koreo-Japonic usage of the character 達. Demonstrative Pronouns are also listed on the right side, having been made \textit{a priori} simply due to there being a lack of consensus on what the words should be despite consensus on the levels of distinction.
        \begin{center}\footnotesize{}
            \begin{tabular}{|c|c|c|} \hline
                 & Singular & Plural \\ \hline
               1st. & \textit{a} & \textit{atat} \\ \hline
               2nd. & \textit{ni} & \textit{nitat} \\ \hline
                 3rd. & \textit{ta} & \textit{tatat} \\ \hline
            \end{tabular}
            \begin{tabular}{|c|c|l|} \hline
                Function &  Pronoun & Description\\ \hline
                Proximal & \textit{ko} & = This \\ \hline
                Medial  & \textit{co} & = That\\ \hline
                Distal  & \textit{yo} & = Yon \\ \hline
            \end{tabular}
        \end{center} 
    \paragraph{}
    Other types of pronouns are listed below. 
    \footnotesize \begin{center}
        \begin{tabular}{|c|c|c|c|c|c|c|c|}\hline
             \multicolumn{2}{|c|}{} &Interrogative & Particular & Universal & Negative & Alternative \\ \cline{3-7}
             \multicolumn{2}{|c|}{}& Which & Certain & All & None & Other \\ \hline
            General && \textit{ha} & \textit{hok} & \textit{man} & --- & \textit{ta} \\ \hline
            Objective &Object& \textit{hakka} & \textit{hokko} & \textbf{\textit{manmut}} & \textit{mukku} & \textit{takka} \\ \hline
             Personal &Person& \textit{hanin} & \textit{hoknin} & \textbf{\textit{manmin}} & \textit{munin} & \textit{tanin} \\ \hline
             Temporal &Time& \textit{hasi} & \textit{hoksi} & \textit{mansi} & \textit{musi} & \textit{tasi} \\ \hline
             Locative &Place& \textit{hadi} & \textit{hokdi} & \textit{mandi} & \textit{mudi} & \textit{tadi} \\ \hline
             Causal &Reason& \textit{hayu} & \textit{hokyu} & \textit{manyu} & \textit{muyu} & \textit{tayu} \\ \hline
             Modal &Method& \textit{ha'i} & \textit{hok'i} & \textit{man'i} & \textit{mu'i} & \textit{ta'i} \\ \hline
        \end{tabular}
    \end{center} 
    \normalsize
    \vfill\newpage
    \section{Grammar} 
        \subsection{Overview}
        \paragraph{}The grammar of Manmino is intended to be both simple yet expressive so that it could be an easy to learn language while retaining nuances necessarily for complex communication. Manmino generally orders its words in Subject - Object - Verb order, with modifiers always prefacing the word they modify and particles following the word they modify. While words themselves may be placed in free order as long as there is no ambiguity, the order of particles relative to the word is absolute. In writing, prefixes and suffixes may append the word they modify with a dash; dashes do not perform any phonetic function and are optionally written to provide clarity. Manmino is also a pro-drop language: particles and words both generally may be dropped as long as the context clarifies the meaning of the sentence. 
        \paragraph{}The following are acceptable and unacceptable examples of the sentence "I go to school" in Manmino. Do note that even the unacceptable examples listed here may be somewhat acceptable, at least in the most basic configurations such as this one, as the context in which these sentences are said may clarify the meaning of the sentences.
        \paragraph{}\hspace{.3in}O: \textit{A hakhaw kyo. / A hakhaw-bang kyo. }
        
        \textit{\hspace{.4in}A kyo hakhaw-bang. / Hakhaw-bang a kyo.}
        
        \paragraph{}\hspace{.3in}X: \textit{Hakhaw a kyo. / A kyo hakhaw.}
        \paragraph{}The following is a demonstration of the prefacing modifiers clause. The accusative particle \textit{lu} has been dropped in all cases as the sentences still satisfy the ambiguity clause.
        \paragraph{}\hspace{.3in}\textit{A amahi pan sik.}  
        
        \hspace{.4in}= I eat delicious bread.
        \paragraph{}\hspace{.3in}\textit{A pan amahi sik.}  
        
        \hspace{.4in}= I deliciously eat bread.
        \paragraph{}\hspace{.3in}\textit{Amahi a pan sik.}  
        
        \hspace{.4in}= (Non-grammatical sentence)
        
        \vfill\newpage
        \subsection{Formality}
        \paragraph{} For the sake of accommodating at least some of various grammatical formality systems found across East Asia, Manmino grammatically considers three degrees of formality: The formal degree, the standard degree, and the intimate degree. Each of the degrees of formality are simple to account for, but each behave differently. The formal degree, used in formal settings such as public speeches or to someone of higher social standing than you, mandates only two things: use titles instead of the second person pronouns, and if talking about the act of someone of a higher social standing than the speaker, the verb must be marked with the formal particle \textit{si}. The standard degree, meanwhile, is unmarked in any way shape or form for formality. In the intimate degree, used among close friends and family, intimate pronouns (in lieu of nicknames) may be used rather than simple second person pronouns: males may be referred to as \textit{kun}, females may be referred to as \textit{nyang}, and \textit{ini} can be used as neuter gender.
        \subsection{Numbers}
            \paragraph{} In Manmino, numbers may require being marked with a counter, much like other languages of East Asia. When counting real items in Manmino, the counter \textit{gay} is used for inanimate objects, \textit{dow} is used for animals, and \textit{meng} is used for people. When counting events, the counter \textit{hway} is used. Ordinal numbers use \textit{ban}.











































































































\chapter{\textit{Hankok'o} | 한국어}\normalsize{}
    \section{글자와 소리}
    \paragraph{}
    만민어는 동아시아 및 동남아시아에 걸쳐 널리 알려져 있는 라틴 문자(만민어: Latin Aksala 또는 Latin Munji)로 표기한다. 만민어 문자(Manmino Aksala)에서는 다음과 같은 글자를 사용한다.\marginpar[]{ \footnotesize{\textbf{Aksala}: 산스크리트에서 유래한 만민어 단어. 만민어에서 ‘munji’는 좁은 의미로는 특히 음절 또는 모라 단위로 쓰이는 문자 (한자, 한글, 가나 등)를 가리키며, ‘aksala’는 음소 단위로 쓰이는 문자(알파벳이나 데바나가리 등)을 가리킨다. 이 둘은 대체로 서로 바꾸어 쓸 수 있는 동의어이지만 엄밀한 의미로는 구분된다.}}
    \begin{center}
    A B C D E F G H I J K L M N O P S T U W Y 
    
    a b c d e f g h i j k l m n o p s t u w y '
    \end{center}
    \paragraph{}
    이 가운데 다음은 자음이다.
    \footnotesize{}
    \begin{center}
    \begin{tabular}{ | c | c | c | c | c | c | c | } \hline 
    \multicolumn{2}{|c|}{} & 순음 & 치음 & 연구개음 & 경구개음 & 성문음 \\  \hline
    \multicolumn{2}{|c|}{비음}  & m & n & & ng [\textipa{N}]$^1$ & \\        \hline
    \multirow{2}{*}{파열음} & 경음$^2$ & p [p\textsuperscript{h}] & t [t\textsuperscript{h}] & & k [k\textsuperscript{h}] & \multirow{2}{*}{' [\textipa{P}]$^3$} \\  \cline{2-6}
     & 연음$^2$ & b & d & & \textipa{g} &\\  \hline
    \multirow{2}{*}{파찰음} & 경음$^2$ & & & c [\texttctclig{}\textsuperscript{h}] & & \\ \cline{2-7}
     & 연음$^2$ & & & j [\textdctzlig{}] & & \\ \hline
    \multicolumn{2}{|c|}{마찰음} & f [\textipa{F}$\sim{}$f] & s & (\textipa{C})$^4$ &  & h [x $\sim{}$h] \\ \hline
    \multicolumn{2}{|c|}{설측음} & & l & & & \\ \hline
    \multicolumn{2}{|c|}{반모음} & & & y [j] & w & \\ \hline
    \end{tabular}
    \end{center}
    \begin{enumerate}\marginpar[]{ \footnotesize{ \textbf{음운}: 음운을 결정하는 데는 다음과 같은 방법이 사용되었다 . 광역 동아시아의 주요 언어 20여개의 전체 음운 목록을 구하고, 각 언어의 음운 목록을 기록해 그 언어의 화자 수의 세제곱근에 해당하는 값을 배정한다. 그 다음 모든 음운 목록을 더한 뒤, 중간보다 위에 있는 음운은 취하고 나머지는 버린다.}}
      \item 음소 /\textipa{N}/ 은 초성이 될 수 없다. 
      \item 경음과 연음 자음의 정확한 실현은 대상 언어들 사이의 선호가 갈리는 관계로 임시적으로 특정하지 않았다.
      \item 음소 /\textipa{P}/는 음절 사이에서만 나타날 수 있다.
      \item /s/는 /j/ 나 /i/ 앞에서는  [\textipa{C}] 으로 발음된다.
      \item 만민어의 자음 중 비음 /\textipa{m, n, N}/, 성문 파열음을 제외한 경음 파열음 [\textipa{p, t, k}/, 그리고 설측음 /l/ 만 종성이 될 수 있다.
    \end{enumerate}\normalsize{}
    \paragraph{}
    q, x, z 등 여기에 포함되지 않은 다른 라틴 문자는 고유 명사 등 만민어에서 정의되지 않은 단어를 표기하는 데 쓰일 수 있다. 그러한 경우에는 그 단어 전체를 가능한 한 원어에 가깝게 발음하여야 한다.
    
    \footnotesize{}\begin{center}
    \begin{tabular}{ | c | c | c | c | c | c | c | c | } \hline
    \multicolumn{2}{|c|}{\multirow{2}{*}{}} & \multicolumn{2}{|c|}{+\O} & \multicolumn{2}{|c|}{+/w/$^1$} & \multicolumn{2}{|c|}{+/j/$^1$} \\ \cline{3-8}
    \multicolumn{2}{|c|}{} & 전설모음 & 후설모음 & 전설모음 & 후설모음 & 전설모음 & 후설모음 \\ \hline
    \multicolumn{2}{|c|}{폐모음} & i & u & yu /ju/ & & & wi \\ \hline
    \multirow{3}{*}{중모음} & \O+ & e & o & ew & ow & ey /ej]/& oy /oj/ \\ \cline{2-8}
    & [j]$^2$+ & ye /je/ & yo /jo/ & yew /jew/ & yow /jow/ & yey /jej/ &  \\ \cline{2-8}
    & [w]+ & we & & & & wey /wej/ & \\ \hline
    \multirow{3}{*}{개모음} & \O+ & \multicolumn{2}{|c|}{a} & \multicolumn{2}{|c|}{aw} & \multicolumn{2}{|c|}{ay /aj/} \\ \cline{2-8}
    & /j/+ & \multicolumn{2}{|c|}{ya /ja/} & \multicolumn{2}{|c|}{yaw /jaw/} & \multicolumn{2}{|c|}{yay /jaj/} \\ \cline{2-8}
    & /w/+ & \multicolumn{2}{|c|}{wa} & \multicolumn{2}{|c|}{waw} & \multicolumn{2}{|c|}{way /waj/} \\ \hline
    
    
    \end{tabular}
    \end{center}\marginpar[\footnotesize  \textbf{모음}:   /\textipa{@}/를 $\langle{}$e$\rangle{}$로 도입하는 방안이 상당이 유력했으나 ( 이 경우 /e/는 $\langle{}$ye$\rangle{}$로 표기), 근소한 차이로 중간을 넘지 못해 결국 배제되었다 .음소에 성조를 포함하는 방안 또한 비슷하게 배제되었다.]]{}
    \begin{enumerate}
      \item 자음은 /w/ 또는 /j/ 뒤에는 올 수 없으나, /u/ 또는 /i/ 뒤에는 올 수 있다.
      \item 모음 /e/ 앞에서 /j/는 초성으로만 쓰일 수 있고 활음(滑音)으로는 쓰일 수 없다. 유일한 예외는 /s/ 뒤에 쓰일 때뿐이다.
      
    \end{enumerate}
    \normalsize{}
    \paragraph{}
    위에 제시한 음소를 사용해 만민어에서는C$_1$VC$_2$ 구조의 음절을 이룬다. C$_1$은 /\textipa{N}/과 /\textipa{P}/를 제외한 자음이고, V는 명시된 예외를 제외한 위의 모든 단모음, 이중 모음 또는 삼중 모음이며, C$_2$는 비음 /m, n, \textipa{N}/, 성문 파열음 이외의 경음 파열음 /p, t, k/, 그리고 설측음 /l/으로 제한된다. 상술한 대로, /\textipa{P}/는 어느 한 음절의 일부로서는 나타나지 않고 음절 사이에서만 나타난다. 성문파열음은 중의성을 회피할 때에만 필요하므로, 일반적으로는 대체로 음운성을 갖지 않는다고 보아도 무방하다.  
    \paragraph{}
    더 다양한 화자들을 포용하고 빠른 발화를 용이하게 하기 위하여 만민어에서는 특정한 이중 자음에 대하여 유연한 이음(異音) 발음을 허용한다. 규칙은 아래와 같다:
        \begin{itemize}
            \item \textbf{종성 비음 규칙:} 만약 종성 비음 $\langle{}$m, n, ng$\rangle{}$ 다음에 조음 위치가 다른 파열음이나 파찰음이 오면, 그 종성 비음은 따라 오는 자음의 조음위치에서 발음 할 수 있습니다. 예를 들어서, 이중음자 $\langle{}$mk$\rangle{}$는 [mk]가 아닌  [\textipa{N}k]으로 발음 할 수 있습니다. 이 규칙을 적용할 때 연구개음은 치음의 일종으로 취급 합니다. 
            \item \textbf{종성 파열음 규칙:} 만약 종성 파열음  $\langle{}$p, t, k$\rangle{}$ 다음에 연음 $\langle{}$b, d, j, g$\rangle{}$가 오면, 종성을 발음 하지 않고 대신 연음을 경음으로 바꿔 발음 할 수 있습니다.  
            \item \textbf{$\langle{}$ph$\rangle{}$의 규칙:} 이중음자 $\langle{}$ph$\rangle{}$ 는 [f]로 발음 할 수 있습니다.
            \item \textbf{$\langle{}$ty/dy$\rangle{}$의 규칙: }이중음자 $\langle{}$ty$\rangle{}$및  $\langle{}$dy$\rangle{}$은 각각 [\texttctclig{}]혹은  [\textdctzlig{}]로 발음 할 수 있습니다.
            \item \textbf{$\langle{}$kh$\rangle{}$의 규칙: }이중음자 $\langle{}$kh$\rangle{}$에서 [k]는 묵음으로 취급 할 수 있습니다.
        \end{itemize}
    발음에 혼란이 있는 경우에는 더미 모음을 활용하거나\footnote{더미 모음: 특정 언어 화자가 발음 할 수 없는 자음군을 발음 해야 할 경우에 임의적으로 추가되는 모음} 위 규칙들을 적용하지 않고 더 천천히 발음해 혼란을 해소 하는것을 추천합니다. 
    \vfill\newpage
    \section{조사와 단어 역할}
            \paragraph{}
    만민어에서 명사, 동사, 형용사, 부사(副詞) 등 단어의 역할은 문장에서의 위치나 전치사 및 후치사로 구분 됩니다. 따라서, 만민어에서 단독적인 단어는 명사, 동사, 형용사, 부사등으로 구분 되지 않고 물질과 개념, 행동과 현상, 그리고 성질로 구분 됩니다. 이 단어들의 역할 (명사, 동사, 형용사, 부사)는 주변에 있는 조사에 따라 결정됩니다. 
        \subsection{만민어의 명사}
            \paragraph{}
    만민어에서 명사는 문법적 수를 의무적으로 표기하지 않는 점이 특징입니다. 만민어에서 문법적 복수를 강조하고 싶다면 첩어를 \footnote{한 단어를 연이어 두번 쓰는것} 사용하는 방법과 후치사 \textit{-tat}을 사용 하는 방법이 있습니다. 만민어에서 명사는 다음과 같은 후치사들로 명사의 역할을 표기 합니다.
    \marginpar[]{\footnotesize \textbf{na}: 한국어 \textit{은/는} 및 일본어 は에서 따옴.}                                       \marginpar[]{\footnotesize \textbf{lu}: 한국어 \textit{을/를} 및 일본어 を에서 따옴.}                                       \marginpar[]{\footnotesize \textbf{ji}: 한문에서 之를 따옴.}   \marginpar[]{\footnotesize \textbf{bat}: 한문에서 發을 따옴.}  \marginpar[]{\footnotesize \textbf{bang}: 한문에서 方을 따옴.} \marginpar[]{\footnotesize \textbf{nay}: 한문에서 內을 따옴.}  \marginpar[]{\footnotesize \textbf{ne} \textsuperscript{1,2}: 한국어 \textit{에} 및 일본어 に에서 따옴.}                     \marginpar[]{\footnotesize \textbf{em}: 말레이어/인도네시아어의 \textit{an} 및 한국어 \textit{임} 에서 따옴.}                 \marginpar[]{\footnotesize \textbf{la}: 한국어 \textit{라} 에서 따옴.}  
    
    \footnotesize{}\begin{center}
    \begin{tabular}{ c | c | l }  
    \textbf{부사} & \textbf{역할} & \textbf{비고} \\ \hline
    \multirow{2}{*}{\textit{na}} & \multirow{2}{*}{화제} & 해당 명사를 문장의 화제나 주제로 표기함.  \cr{} & & "은/는" 에 해당 됨. \\ \hline
      & \multirow{2}{*}{주격} & \multirow{2}{*}{만민어에서 주어는 별도로 표기되지 않음.} \cr{} & &\\ \hline
    \multirow{4}{*}{\textit{lu}} & \multirow{4}{*}{대격} & 해당 명사를 문장의 직접목적어로 표기함. \cr{} & & 해당 명사가 문장의 주어와 동사 사이에 있으면 선택적으로 \cr{} & & 표기 할 수 있으나 그 외 상황에서는 의무적으로 표기해야 함. \cr{} & &"을/를"에 해당 됨.\\ \hline
    \multirow{2}{*}{\textit{ji}} & \multirow{2}{*}{속격} & 해당 명사를 뒤 따르는 명사가 속하는 대상으로 표기함.  \cr{} & & "의" 에 해당 됨. \\ \hline
    \multirow{2}{*}{\textit{bat}} & \multirow{2}{*}{탈격} & 해당 명사를 동사의 시작점으로 표기함.\cr{} & & "부터" 에 해당 됨. \\ \hline
    \multirow{2}{*}{\textit{bang}} & \multirow{2}{*}{향격} & 해당 명사를 동사의 목적지로 표기함.\cr{} & & 목적지를 말할 때 쓰는 "로/까지" 에 해당 됨. \\ \hline
    \multirow{2}{*}{\textit{jung}} & \multirow{2}{*}{처격} & 해당 명사를 동사의 장소를 표기함. \cr{} & & 위치를 말할 때 쓰는 "에" 에 해당 됨. \\ \hline
    \multirow{2}{*}{\textit{ne}\textsuperscript{1}} & \multirow{2}{*}{여격} & 해당 명사를 간접 목적어로 표기함.  \cr{} & & "에게" 에 해당 됨.\\ \hline
    \multirow{2}{*}{\textit{ne}\textsuperscript{2}} & \multirow{2}{*}{행위자} & 해당 명사를 수동태 문장에서 행위자로 표기함.  \cr{} & & "에게" 에 해당 됨. \\ \hline
    \multirow{2}{*}{\textit{em}} & \multirow{2}{*}{동명사} & 해당 단어가 원래 행동및 현상, 혹은 성질을 표하는 말일때 해당 단어가 \cr{} & & 명사로 활용된다 하는 표기.  \\ \hline
    \multirow{2}{*}{\textit{la}} & \multirow{2}{*}{인용} & \multirow{2}{*}{명사를 인용한 것 으로 표기함.}\cr{} & &
    \end{tabular}
    \end{center}\normalsize{}
    \vfill \newpage
            \paragraph{}\normalsize
    만민어 동사는 여러가지 조사에 따라 변형되는데, 다음 순서대로의 정보를 표기합니다.\marginpar[]{\footnotesize \textbf{동사 시스템}: 이 동사 변형 시스템은 한국어의 동사 체계를 단순화 해 따왔는데, 이는 단어의 활용도를 높일 수 있기 때문입니다.}
    \footnotesize{}
    \begin{center}
    \begin{tabular}{ | c | c| c| c| c| c | c | }  \hline
    수동태 & 경어 & 명령어$^1$ & 진행형 & 시제$^2$ & 완료$^3$ & 기능 \\ \hline
    \multirow{6}{*}{\textit{li}} & \multirow{6}{*}{\textit{si}} & \multirow{6}{*}{\textit{ye}} & \multirow{6}{*}{\textit{jung}} & \multirow{2}{*}{과거} & \multirow{2}{*}{완료} & 강조형\\
       &    &    &      & \multirow{2}{*}{\textit{le}} & \multirow{2}{*}{\textit{le}} & \textit{ya} \\ \cline{7-7}
    & & & &  &  & 의문형\\\cline{5-6}
    & & & & \multirow{2}{*}{미래}  & \multirow{2}{*}{잠재}  & \textit{ka} \\ \cline{7-7}
    & & & & \multirow{2}{*}{\textit{kalu}} & \multirow{2}{*}{\textit{kalu}} & 청유형\\
    & & & &  &  & \textit{ne} \\ \hline
    \end{tabular}
    \end{center}
    \begin{enumerate}
        \item 명령조사 이후에는 동사조사가 올 수 없습니다. 
        \item 시제는 암시할 수 있으며, 현재시제는 표기 되지 않습니다.
        \item 완료를 표하는 조사는 이미 시제사가 있는 상황에서만 기능합니다. 
    \end{enumerate}
    \normalsize{}
            \paragraph{}
    만민어에서 수식어는 항상 수식 대상 앞에 오며, 첩어를 통해 강조됩니다. 수식어의 본질에 따라 (물질과 개념, 행동과 현상, 그리고 성질) 특정 조사를 뒤에 두어 기능을 밝혀야 하는 경우도 있습니다. 물질이나 개념을 지칭하는 말이 형용사의 위치에 있는 경우에는 그 뒤에 조사 \textit{dek} 이 와야 하며, 만약 해당 말이 부사의 위치에 있는 경우에는 그 뒤에 조사 \textit{syang} 이 와야 합니다. 만약 행동이나 현상을 지칭하는 말이 형용사의 위치에 있는 경우에는 그 뒤에 조사 \textit{in} 이 와야 하며, 만약 해당 말이 부사의 위치에 있는 경우에는 그 뒤에 조사 \textit{syang} 이 와야 합니다.
            \paragraph{}
    부정사는 만민어의 유일한 접두사 입니다. 동사는 접두사 \textit{but}로 부정되며, 형용사는 접두사 \textit{bi}로 부정됩니다. 만민어에서 문장은 \textit{but cey}라는 말로 부정 하거나 줄여서 \textit{bey}으로 부정할 수 있습니다.
            \paragraph{}
            단어로부터 독립되어 작용되는 부사들도 몇 있는데, 이들은 대신 어구 전체에 작용 됩니다. 이들은 접속사로 기능하거나 문장의 기능을 표합니다. 각종 절(節)의 역할을 바꾸는 부사들은 보통 해당 절 앞에 오는것을 주의하시기 바랍니다. 
                 \begin{center} \footnotesize{}
            \begin{tabular}{|c|c|l|} \hline
                부사 & 기능 & 설명 \\ \hline
                \multirow{2}{*}{\textit{i}} & \multirow{2}{*}{추가} & 한국어의 "및, 그리고"등 에 해당됨. \\ & & 두 어구를 묶음. \\ \hline
                \multirow{2}{*}{\textit{hwa}} & \multirow{2}{*}{추가} & 한국어의 "와, 랑"등 에 해당됨. \\ & & 두 단어를 한 어구로 묶음. \\ \hline
                \multirow{2}{*}{\textit{hok}} & \multirow{2}{*}{대체} & 한국어의 "(이)나, 혹, 아니면" 등에 해당됨. \\ & & 두 어구를 상반되는 관계로 묶음. \\ \hline
                \multirow{2}{*}{\textit{pi}} & \multirow{2}{*}{비교} & 한국어의 "비해" 에 해당됨.\\ && 두 어구를 비교되는 관계로 묶음. \\ \hline
                \multirow{3}{*}{\textit{nyak}} & \multirow{3}{*}{가정} & 한국어의 "만약"에 해당됨.\\  & & nyak-A-B 구조나 B-nyak-A 구조로 활용되어\\  && A가 가정문의 역할을 하고 B가 결과문의 역할을 함.\\ \hline
                \multirow{3}{*}{\textit{dan}} & \multirow{3}{*}{제한} & 한국어의 "하지만, 다만"등에 해당됨. \\  && dan-A-B 구조나 B-dan-A 구조로 활용되어 \\ && A 가 제한 조건이 되고 B가 결과문의 역할을 함. \\ \hline
                \multirow{2}{*}{\textit{yu}} & \multirow{2}{*}{인과 1} & 한국어 "때문에"에 등에 해당됨.\\  && 앞에 오는 어구는 이유, 뒤에 오는 어구는 효과가 됨. \\ \hline
                \multirow{2}{*}{\textit{wi}} & \multirow{2}{*}{인과 2} & 한국어 "위해" 등 해당됨.\\ && 앞에 오는 어구는 목표, 뒤에 오는 어구는 과정이 됨. \\ \hline
                \end{tabular}
        \end{center}\normalsize{}
        \vfill\newpage
    \section{대명사}
    \paragraph{}
    만민어의 인칭 대명사는 좌측에 있습니다. 이는 전반적으로 한문에서 따왔으며, 복수표기는 한일식 한문에서 達의 용래 에서 따왔습니다. 지시 대명사는 우측에 있습니다. 이는 선험적(先驗的)으로 만들어 졌는데, 이는 동양 전반적으로 이 세 단계가 있어야 한다는 합의는 있어도 해당 말을 무엇으로 해야 할 지 는 합의가 없기 때문입니다. 
        \begin{center}\footnotesize{}
            \begin{tabular}{|c|c|c|} \hline
                 & 단수 & 복수 \\ \hline
                 1인칭 & \textit{a} & \textit{atat} \\ \hline
                 2인칭 & \textit{ni} & \textit{nitat} \\ \hline
                 3인칭. & \textit{ta} & \textit{tatat} \\ \hline
            \end{tabular}
            \begin{tabular}{|c|c|l|} \hline
                기능 &  대명사 & 설명 \\ \hline
                근칭 & \textit{ko} & = 이 \\ \hline
                중칭  & \textit{co} & = 그\\ \hline
                원칭  & \textit{yo} & = 저 \\ \hline
            \end{tabular}
        \end{center} \normalsize{}
    \paragraph{}
    이 외 대명사는 아래에 있습니다. 
    \footnotesize
    \begin{center}
        \begin{tabular}{|c|c|c|c|c|c|c|}\hline
             &의문형 & 특정 & 만칭 & 부정 & 타칭 \\ \cline{2-6}
             &어떤& 그런 & 모든 & 아무 & 다른 \\ \hline
            기본형 & \textit{ha} & \textit{hok} & \textit{man} & --- & \textit{ta} \\ \hline
            물질 & \textit{hakka} & \textit{hokko} & \textbf{\textit{manmut}} & \textit{mukku} & \textit{takka} \\ \hline
             인물 & \textit{hanin} & \textit{hoknin} & \textbf{\textit{manmin}} & \textit{munin} & \textit{tanin} \\ \hline
             시간& \textit{hasi} & \textit{hoksi} & \textit{mansi} & \textit{musi} & \textit{tasi} \\ \hline
             장소& \textit{hadi} & \textit{hokdi} & \textit{mandi} & \textit{mudi} & \textit{tadi} \\ \hline
             이유& \textit{hayu} & \textit{hokyu} & \textit{manyu} & \textit{muyu} & \textit{tayu} \\ \hline
             방법& \textit{ha'i} & \textit{hok'i} & \textit{man'i} & \textit{mu'i} & \textit{ta'i} \\ \hline
        \end{tabular}
    \end{center}
    \normalsize\vfill\newpage
    \section{문법} 
        \subsection{전반}
        \paragraph{}만민어의 문법은 간단하면서도 표현력이 강해 배우기도 쉬우면서 복잡하며 뉘앙스 깊은 소통에도 쓰일 수 있기를 표방한다.  만민어는 보통 주어 - 목적어 - 서술어 순서로 단어를 놓으며, 수식어는 언제나 수식되는 단어 앞에 놓이고, 부사는 단어의 뒤에 간다.  단어들 자체는 오해가 없는 한 자유롭게 놓일 수 있으나, 부사의 위치는 절대적이다.  글로 쓸 때는 수식어 및 부사는 자신이 속해있는 단어와 붙임표으로 연결 시킬 수 있다. 여기서 붙임표는 음운적인 기능을 하지 않고 선택적으로 뜻을 명백하게 하기 위해 쓰인다.  만민어는 또한 pro-drop 언어다. 즉, 단어 및 부사 둘 다 환경상 문장의 의미가 명백하면 해당 문장에서 생략 될 수 있다.
        \paragraph{}다음은 "나는 학교에 간다"라는 문장의 허용되는 예시 및 되지 않는 예시입니다. 다만 여기에서 허용되지 않는 다 하더라도, 특히 이렇게 간단한 구조인 경우에는, 환경상 문맥이 파악 되어 허용 할 수 있는 경우 도 있읍니다.
        \paragraph{}\hspace{.3in}O: \textit{A hakhaw kyo. / A hakhaw-bang kyo. }
        
        \textit{\hspace{.4in}A kyo hakhaw-bang. / Hakhaw-bang a kyo.}
        
        \paragraph{}\hspace{.3in}X: \textit{Hakhaw a kyo. / A kyo hakhaw.}
        \paragraph{}다음은 수식어가 수식되는 단어 앞에 놓이는 규칙을 보여주는 예시 입니다. 여기서 대격 부사 \textit{lu}는 생략되어도 문맥상 뜻이 파악 되기에 생략 되었습니다. 
        \paragraph{}\hspace{.3in}\textit{A amahi pan sik.}  
        
        \hspace{.4in}= 나는 맛있는 빵을 먹는다. 
        \paragraph{}\hspace{.3in}\textit{A pan amahi sik.}  
        
        \hspace{.4in}= 나는 빵을 맛있게 먹는다.
        \paragraph{}\hspace{.3in}\textit{Amahi a pan sik.}  
        
        \hspace{.4in}= (비문)
        
        \vfill\newpage
        \subsection{경어}
        \paragraph{} 동양 각각 언어 대부분에 어느 정도는 있는 경어를 감안하기 위해, 만민어는 세 등급의 경어를 갖춘다: 경어, 평어, 속어.  각 등급은 각자 다르지만 간단한 방식으로 표시됩니다. 경어는 공적으로 발언을 하거나 자신보다 사회적으로 더 높은 사람을 대상으로 말할 때 쓰이며, 평어에 비해 두가지 차이가 있는데, 상대방을 이인칭 대명사가 아닌 직함이나 다른 칭호로 불러야 하며, 경어를 써야 하는 사람을 서술 하는 동사에 부사 \textit{si}를 붙혀야 한다. 평어는 기본적인 문장 구조를 취한다. 속어는 친구 및 가족과 쓰는 어휘며, 일반 이인칭 대명사 대신 애칭 혹은 애칭 대명사를 쓸 수 있다. 남성은 \textit{kun}, 여성은 \textit{nyang}, 성에 상관 없이 쓰일때는 \textit{ini}를 쓸 수 있다.
        \subsection{수사}
            \paragraph{} 만민어에서는 동양에 있는 여러 언어처럼 수 뒤에 수사가 오는 경우가 있습니다. 물리적 물체를 셀때 무정물은 \textit{gay}를 쓰고, 동물들은  \textit{dow}를 쓰고, 사람들은 \textit{meng}을 씁니다. 어떤 행위나 사건을 셀때는 \textit{hway}를 씁니다. 순서를 셀때는 \textit{ban}을 사용합니다.
        \vfill\newpage

\chapter{Manmino Siden}\normalsize
\paragraph{} 
As words in Manmino come from a variety of sources, it is important to clarify where each word comes from in order to ensure that they have justification to be used in Manmino, as well as lay forth a set of rules when coining or borrowing new words into Manmino.

\paragraph{}
In general, the most basic ideas that are seen or dealt with on a daily basis were loaned from either a Sinitic base, the Koreo-Japonic Sprachbund, the Austronesian Language family, or a Austroasian language. Some abstract concepts were loan-ed from Sanskrit, and more complex compound words were loaned from Sinitic if it was either commonly used in at least two of the four Confucian Nations (China, Japan, Korea, Vietnam) or would have been easily recognizible. Highly modern inventions such as words pertaining to computing technology have been loaned from English, but otherwise, locally coined words were prioritized over words coined in the West.

\paragraph{}
This is not an exclusive list of words in Manmino. Rather, there are just a set of words that Team Manmino felt a need to define extemporaneously, and will periodically be updated with new editions with more words based on feedback from the community.

\paragraph{}
If a word is not yet defined in Manmino, they may still be used following these conditions: If the new word is \textit{not} a proper noun (such as the name of a specific business, specific person, or other names that have a unified Latin orthography regardless of language) the word must conform to Manmino orthographic conventions, and if the word coined based on Sinitic stems it must be made using only Manmino readings of characters -- The guidance for the specific procedure for this will be published at a future edition as it is quite difficult to define fully, but users are free to reverse-engineer the procedures used here in the meanwhile with the knowledge that the algorithm utilizes the \textit{Guangyun} dictionary of characters.
\begin{table}
    \small
    \begin{tabular}{|l|l|} 
        \multicolumn{2}{c}{\large{Key of Terms used in the Dictionary}}  \\ \hline
        &\\
         \multirow{2}{*}{Action} & Words that describe actions or phenomena. \\& By default, they are assumed to be verbs.\\ & \\
         \multirow{2}{*}{Object} & Words that describe objects or concepts.\\& By default, they are assumed to be nouns. \\ & \\ 
         \multirow{2}{*}{Quality} & Words that describe qualities.\\& By default, they are assumed to be adjectives or adverbs.\\ & \\ \hline &\\
         \multirow{3}{*}{Austroasian} &  Words of languages from Austroasia (A region normally called Indo-China)\\& or Continental South East Asia. These include Vietnamese and Khmer,\\& as well as Thai and Lao (who belong to the Kra–Dai language family).\\
        \multirow{3}{*}{Austronesian} & \multirow{2}{*}{Words of languages of Austronesia, or Maritime South East Asia.}\\& \multirow{2}{*}{These include Indonesian, Malaysian, Tagalog, Javanese.}\\& \\
        \multirow{3}{*}{Koreo-Japonic} &  Words of Korean or Japanese origin. Grouped together due to extensive\\& volume of cognates/loans shared between the two languages. \\& Research on the exact nature of their relationship remains active. \\
        \multirow{3}{*}{Sanskrit} &  \multirow{2}{*}{Words of Sanskrit origin. Sanskrit was only considered as a source language} \\& \multirow{2}{*}{if a Sanskrit word was actively used in multiple South East Asian languages. }\\& \\
        \multirow{4}{*}{ Sinitic} & Words of Sinitic origin. Sinitic vocabulary were derived based on reconstructed\\& phonetic values of characters in  Middle Chinese (spoken around Tang through \\&Song Dynasties), the source language for almost all modern readings of \\& Chinese characters.  \\& \\ \hline
    \end{tabular}
    \normalsize
\end{table}
\begin{table}

\end{table}

\vfill\newpage
\footnotesize\newgeometry{margin=.5in}
\begin{longtable}[ht]{l r l r l}
    \centering 
		\textbf{Manmino}&		\textbf{Function} 		&		 \textbf{Definition}		&		\textbf{Origin} 		&					\textbf{Etymology}	 \\\arrayrulecolor{gray} \hline
\multirow{3}{*}{	\textbf{\textit{	aca	}}}	&	\multirow{3}{*}{	O/C	}	&	\multirow{3}{*}{	morning	}	&	\multirow{3}{*}{	Koreo-Japonic	}	&	\multirow{	2	}{*}{	\textit{	ko	 - }		아침		}	\\&&&&	\multirow{	2	}{*}{	\textit{	ja	 - }		あさ		}	\\&&&&	\textit{		}					\\\arrayrulecolor{gray} \hline
\multirow{3}{*}{	\textbf{\textit{	aca (ji) sikko	}}}	&	\multirow{3}{*}{	O/C	}	&	\multirow{3}{*}{	breakfast	}	&	\multirow{3}{*}{	Compound	}	&	\multirow{	3	}{*}{	\textit{		}				}	\\&&&&				\textit{		}					\\&&&&	\textit{		}					\\\arrayrulecolor{gray} \hline
\multirow{3}{*}{	\textbf{\textit{	aka	}}}	&	\multirow{3}{*}{	O/C	}	&	\multirow{3}{*}{	baby, infant, toddler	}	&	\multirow{3}{*}{	Koreo-Japonic	}	&	\multirow{	2	}{*}{	\textit{	ko	 - }		아가		}	\\&&&&	\multirow{	2	}{*}{	\textit{	ja	 - }		あかちゃん		}	\\&&&&	\textit{		}					\\\arrayrulecolor{gray} \hline
\multirow{3}{*}{	\textbf{\textit{	aken	}}}	&	\multirow{3}{*}{	O/C	}	&	\multirow{3}{*}{	autumn	}	&	\multirow{3}{*}{	Koreo-Japonic	}	&	\multirow{	3	}{*}{	\textit{	ja	 - }		あき		}	\\&&&&				\textit{		}					\\&&&&	\textit{		}					\\\arrayrulecolor{gray} \hline
\multirow{3}{*}{	\textbf{\textit{	aksala	}}}	&	\multirow{3}{*}{	O/C	}	&	\multirow{3}{*}{	alphabet, abugida	}	&	\multirow{3}{*}{	Sanskrit	}	&	\multirow{	2	}{*}{	\textit{		}	\textsanskrit{	अक्षर 	}	}	\\&&&&	\multirow{	2	}{*}{	\textit{		}		(akṣara)		}	\\&&&&	\textit{		}					\\\arrayrulecolor{gray} \hline
\multirow{3}{*}{	\textbf{\textit{	alkohol	}}}	&	\multirow{3}{*}{	O/C	}	&	\multirow{3}{*}{	alcohol	}	&	\multirow{3}{*}{	Arabic	}	&	\multirow{	3	}{*}{	\textit{		}				}	\\&&&&				\textit{		}					\\&&&&	\textit{		}					\\\arrayrulecolor{gray} \hline
\multirow{3}{*}{	\textbf{\textit{	alon	}}}	&	\multirow{3}{*}{	O/C	}	&	\multirow{3}{*}{	wave	}	&	\multirow{3}{*}{	Austronesian	}	&	\multirow{	2	}{*}{	\textit{	ms/id/jv	 - }		alun		}	\\&&&&	\multirow{	2	}{*}{	\textit{	tg	 - }		alon		}	\\&&&&	\textit{		}					\\\arrayrulecolor{gray} \hline
\multirow{3}{*}{	\textbf{\textit{	along	}}}	&	\multirow{3}{*}{	Qual	}	&	\multirow{3}{*}{	color / stain	}	&	\multirow{3}{*}{	Koreo-Japonic	}	&	\multirow{	2	}{*}{	\textit{	ko	 - }		알록		}	\\&&&&	\multirow{	2	}{*}{	\textit{	ja	 - }		いろ		}	\\&&&&	\textit{		}					\\\arrayrulecolor{gray} \hline
\multirow{3}{*}{	\textbf{\textit{	amahi	}}}	&	\multirow{3}{*}{	Qual	}	&	\multirow{3}{*}{	yummy, tasty, sweet	}	&	\multirow{3}{*}{	Koreo-Japonic	}	&	\multirow{	2	}{*}{	\textit{	ko	 - }		맛있다		}	\\&&&&	\multirow{	2	}{*}{	\textit{	ja	 - }		うまい		}	\\&&&&	\textit{		}					\\\arrayrulecolor{gray} \hline
\multirow{3}{*}{	\textbf{\textit{	ang'in	}}}	&	\multirow{3}{*}{	O/C	}	&	\multirow{3}{*}{	wind	}	&	\multirow{3}{*}{	Austronesian	}	&	\multirow{	2	}{*}{	\textit{	ms/id	 - }		angin 		}	\\&&&&	\multirow{	2	}{*}{	\textit{	tl	 - }		hangin		}	\\&&&&	\textit{		}					\\\arrayrulecolor{gray} \hline
\multirow{3}{*}{	\textbf{\textit{	anjwen	}}}	&	\multirow{3}{*}{	Qual	}	&	\multirow{3}{*}{	safe	}	&	\multirow{3}{*}{	Sinitic	}	&	\multirow{	3	}{*}{	\textit{		}		安全		}	\\&&&&				\textit{		}					\\&&&&	\textit{		}					\\\arrayrulecolor{gray} \hline
\multirow{3}{*}{	\textbf{\textit{	anmen	}}}	&	\multirow{3}{*}{	O/C	}	&	\multirow{3}{*}{	face (human)	}	&	\multirow{3}{*}{	Sinitic	}	&	\multirow{	3	}{*}{	\textit{		}		顔面		}	\\&&&&				\textit{		}					\\&&&&	\textit{		}					\\\arrayrulecolor{gray} \hline
\multirow{3}{*}{	\textbf{\textit{	anumi	}}}	&	\multirow{3}{*}{	O/C	}	&	\multirow{3}{*}{	darkness	}	&	\multirow{3}{*}{	Koreo-Japonic	}	&	\multirow{	2	}{*}{	\textit{	ko	 - }		어둠		}	\\&&&&	\multirow{	2	}{*}{	\textit{	jp	 - }		やみ		}	\\&&&&	\textit{		}					\\\arrayrulecolor{gray} \hline
\multirow{3}{*}{	\textbf{\textit{	apae	}}}	&	\multirow{3}{*}{	Qual	}	&	\multirow{3}{*}{	forward; before; ahead	}	&	\multirow{3}{*}{	Koreo-Japonic	}	&	\multirow{	2	}{*}{	\textit{	ko	 - }		앞		}	\\&&&&	\multirow{	2	}{*}{	\textit{	ja	 - }		まえ		}	\\&&&&	\textit{		}					\\\arrayrulecolor{gray} \hline
\multirow{3}{*}{	\textbf{\textit{	apjadu	}}}	&	\multirow{3}{*}{	O/C	}	&	\multirow{3}{*}{	abjad	}	&	\multirow{3}{*}{	Arabic	}	&	\multirow{	3	}{*}{	\textit{		}				}	\\&&&&				\textit{		}					\\&&&&	\textit{		}					\\\arrayrulecolor{gray} \hline
\multirow{3}{*}{	\textbf{\textit{	apwi	}}}	&	\multirow{3}{*}{	O/C	}	&	\multirow{3}{*}{	fire	}	&	\multirow{3}{*}{	Austronesian	}	&	\multirow{	3	}{*}{	\textit{	ms/id	 - }		api		}	\\&&&&				\textit{		}					\\&&&&	\textit{		}					\\\arrayrulecolor{gray} \hline
\multirow{3}{*}{	\textbf{\textit{	apwiboti	}}}	&	\multirow{3}{*}{	O/C	}	&	\multirow{3}{*}{	Mars	}	&	\multirow{3}{*}{	Compound	}	&	\multirow{	3	}{*}{	\textit{		}		apwi + boti		}	\\&&&&				\textit{		}					\\&&&&	\textit{		}					\\\arrayrulecolor{gray} \hline
\multirow{3}{*}{	\textbf{\textit{	apwihali	}}}	&	\multirow{3}{*}{	O/C	}	&	\multirow{3}{*}{	tuesday	}	&	\multirow{3}{*}{	Compound	}	&	\multirow{	3	}{*}{	\textit{		}		apwi + hali		}	\\&&&&				\textit{		}					\\&&&&	\textit{		}					\\\arrayrulecolor{gray} \hline
\multirow{3}{*}{	\textbf{\textit{	apwiko	}}}	&	\multirow{3}{*}{	O/C	}	&	\multirow{3}{*}{	hearth, fire place, cooking heat	}	&	\multirow{3}{*}{	Compound	}	&	\multirow{	3	}{*}{	\textit{		}		apwi + ko		}	\\&&&&				\textit{		}					\\&&&&	\textit{		}					\\\arrayrulecolor{gray} \hline
\multirow{3}{*}{	\textbf{\textit{	asu	}}}	&	\multirow{3}{*}{	O/C	}	&	\multirow{3}{*}{	dog, puppy	}	&	\multirow{3}{*}{	Austronesian	}	&	\multirow{	2	}{*}{	\textit{	ms/id	 - }		asu		}	\\&&&&	\multirow{	2	}{*}{	\textit{	tl	 - }		aso		}	\\&&&&	\textit{		}					\\\arrayrulecolor{gray} \hline
\multirow{3}{*}{	\textbf{\textit{	atwi	}}}	&	\multirow{3}{*}{	Qual	}	&	\multirow{3}{*}{	backward; after; behind	}	&	\multirow{3}{*}{	Koreo-Japonic	}	&	\multirow{	2	}{*}{	\textit{	ko	 - }		뒤		}	\\&&&&	\multirow{	2	}{*}{	\textit{	ja	 - }		あと		}	\\&&&&	\textit{		}					\\\arrayrulecolor{gray} \hline
\multirow{3}{*}{	\textbf{\textit{	awan	}}}	&	\multirow{3}{*}{	O/C	}	&	\multirow{3}{*}{	cloud	}	&	\multirow{3}{*}{	Austronesian	}	&	\multirow{	3	}{*}{	\textit{	ms/id	 - }		awan		}	\\&&&&				\textit{		}					\\&&&&	\textit{		}					\\\arrayrulecolor{gray} \hline
\multirow{3}{*}{	\textbf{\textit{	ay	}}}	&	\multirow{3}{*}{	Action	}	&	\multirow{3}{*}{	love	}	&	\multirow{3}{*}{	Sinitic	}	&	\multirow{	3	}{*}{	\textit{		}		愛		}	\\&&&&				\textit{		}					\\&&&&	\textit{		}					\\\arrayrulecolor{gray} \hline
\multirow{3}{*}{	\textbf{\textit{	ayam	}}}	&	\multirow{3}{*}{	O/C	}	&	\multirow{3}{*}{	chicken	}	&	\multirow{3}{*}{	Austronesian	}	&	\multirow{	3	}{*}{	\textit{	ms/id	 - }		ayam		}	\\&&&&				\textit{		}					\\&&&&	\textit{		}					\\\arrayrulecolor{gray} \hline
\multirow{3}{*}{	\textbf{\textit{	baba	}}}	&	\multirow{3}{*}{	O/C	}	&	\multirow{3}{*}{	dad, father, papa	}	&	\multirow{3}{*}{	N/A	}	&	\multirow{	3	}{*}{	\textit{		}				}	\\&&&&				\textit{		}					\\&&&&	\textit{		}					\\\arrayrulecolor{gray} \hline
\multirow{3}{*}{	\textbf{\textit{	babey	}}}	&	\multirow{3}{*}{	O/C	}	&	\multirow{3}{*}{	pig	}	&	\multirow{3}{*}{	Austronesian	}	&	\multirow{	2	}{*}{	\textit{	ms/id	 - }		babi		}	\\&&&&	\multirow{	2	}{*}{	\textit{	tl	 - }		baboi		}	\\&&&&	\textit{		}					\\\arrayrulecolor{gray} \hline
\multirow{3}{*}{	\textbf{\textit{	baci	}}}	&	\multirow{3}{*}{	O/C	}	&	\multirow{3}{*}{	lightening	}	&	\multirow{3}{*}{	Sanskrit	}	&	\multirow{	2	}{*}{	\textit{		}	\textsanskrit{	वज्र 	}	}	\\&&&&	\multirow{	2	}{*}{	\textit{		}		(vájra) 		}	\\&&&&	\textit{		}					\\\arrayrulecolor{gray} \hline
\multirow{3}{*}{	\textbf{\textit{	bacim	}}}	&	\multirow{3}{*}{	Qual	}	&	\multirow{3}{*}{	west	}	&	\multirow{3}{*}{	Sanskrit	}	&	\multirow{	2	}{*}{	\textit{		}	\textsanskrit{	पश्चिम 	}	}	\\&&&&	\multirow{	2	}{*}{	\textit{		}		(paścima) 		}	\\&&&&	\textit{		}					\\\arrayrulecolor{gray} \hline
\multirow{3}{*}{	\textbf{\textit{	badak	}}}	&	\multirow{3}{*}{	O/C	}	&	\multirow{3}{*}{	rhino	}	&	\multirow{3}{*}{	Austronesian	}	&	\multirow{	3	}{*}{	\textit{	ms/id	 - }		badak		}	\\&&&&				\textit{		}					\\&&&&	\textit{		}					\\\arrayrulecolor{gray} \hline
\multirow{3}{*}{	\textbf{\textit{	bala	}}}	&	\multirow{3}{*}{	O/C	}	&	\multirow{3}{*}{	plains; land	}	&	\multirow{3}{*}{	Koreo-Japonic	}	&	\multirow{	2	}{*}{	\textit{	ko	 - }		벌		}	\\&&&&	\multirow{	2	}{*}{	\textit{	ja	 - }		はら		}	\\&&&&	\textit{		}					\\\arrayrulecolor{gray} \hline
\multirow{3}{*}{	\textbf{\textit{	bam	}}}	&	\multirow{3}{*}{	O/C	}	&	\multirow{3}{*}{	evening; night	}	&	\multirow{3}{*}{	Koreo-Japonic	}	&	\multirow{	2	}{*}{	\textit{	ko	 - }		밤		}	\\&&&&	\multirow{	2	}{*}{	\textit{	ja	 - }		ばん		}	\\&&&&	\textit{		}					\\\arrayrulecolor{gray} \hline
\multirow{3}{*}{	\textbf{\textit{	bam (ji) sikko	}}}	&	\multirow{3}{*}{	O/C	}	&	\multirow{3}{*}{	dinner	}	&	\multirow{3}{*}{	Compound	}	&	\multirow{	3	}{*}{	\textit{		}				}	\\&&&&				\textit{		}					\\&&&&	\textit{		}					\\\arrayrulecolor{gray} \hline
\multirow{3}{*}{	\textbf{\textit{	bama	}}}	&	\multirow{3}{*}{	O/C	}	&	\multirow{3}{*}{	parents	}	&	\multirow{3}{*}{	Sinitic	}	&	\multirow{	3	}{*}{	\textit{		}		父母		}	\\&&&&				\textit{		}					\\&&&&	\textit{		}					\\\arrayrulecolor{gray} \hline
\multirow{3}{*}{	\textbf{\textit{	bambu	}}}	&	\multirow{3}{*}{	O/C	}	&	\multirow{3}{*}{	bamboo	}	&	\multirow{3}{*}{	Austronesian	}	&	\multirow{	3	}{*}{	\textit{	ms/id	 - }		bambu		}	\\&&&&				\textit{		}					\\&&&&	\textit{		}					\\\arrayrulecolor{gray} \hline
\multirow{3}{*}{	\textbf{\textit{	bamhap	}}}	&	\multirow{3}{*}{	Qual	}	&	\multirow{3}{*}{	together	}	&	\multirow{3}{*}{	Sinitic	}	&	\multirow{	3	}{*}{	\textit{		}		凡合		}	\\&&&&				\textit{		}					\\&&&&	\textit{		}					\\\arrayrulecolor{gray} \hline
\multirow{3}{*}{	\textbf{\textit{	bana(l)	}}}	&	\multirow{3}{*}{	Qual	}	&	\multirow{3}{*}{	yes, true, right	}	&	\multirow{3}{*}{	Austronesian	}	&	\multirow{	2	}{*}{	\textit{	ms/id	 - }		benar		}	\\&&&&	\multirow{	2	}{*}{	\textit{	tg	 - }		banal		}	\\&&&&	\textit{		}					\\\arrayrulecolor{gray} \hline
\multirow{3}{*}{	\textbf{\textit{	banbun	}}}	&	\multirow{3}{*}{	O/C	}	&	\multirow{3}{*}{	half	}	&	\multirow{3}{*}{	Sinitic	}	&	\multirow{	3	}{*}{	\textit{		}		半分		}	\\&&&&				\textit{		}					\\&&&&	\textit{		}					\\\arrayrulecolor{gray} \hline
\multirow{3}{*}{	\textbf{\textit{	basu	}}}	&	\multirow{3}{*}{	O/C	}	&	\multirow{3}{*}{	bus	}	&	\multirow{3}{*}{	Western: English	}	&	\multirow{	3	}{*}{	\textit{	en	 - }		bus		}	\\&&&&				\textit{		}					\\&&&&	\textit{		}					\\\arrayrulecolor{gray} \hline
\multirow{3}{*}{	\textbf{\textit{	bat	}}}	&	\multirow{3}{*}{	O/C	}	&	\multirow{3}{*}{	eight	}	&	\multirow{3}{*}{	Sinitic	}	&	\multirow{	3	}{*}{	\textit{		}		八		}	\\&&&&				\textit{		}					\\&&&&	\textit{		}					\\\arrayrulecolor{gray} \hline
\multirow{3}{*}{	\textbf{\textit{	bati	}}}	&	\multirow{3}{*}{	O/C	}	&	\multirow{3}{*}{	bees; insects in general	}	&	\multirow{3}{*}{	Koreo-Japonic	}	&	\multirow{	2	}{*}{	\textit{	ko	 - }		벌		}	\\&&&&	\multirow{	2	}{*}{	\textit{	ja	 - }		はち		}	\\&&&&	\textit{		}					\\\arrayrulecolor{gray} \hline
\multirow{3}{*}{	\textbf{\textit{	batu	}}}	&	\multirow{3}{*}{	O/C	}	&	\multirow{3}{*}{	rock	}	&	\multirow{3}{*}{	Austronesian	}	&	\multirow{	3	}{*}{	\textit{	ms/id	 - }		batu		}	\\&&&&				\textit{		}					\\&&&&	\textit{		}					\\\arrayrulecolor{gray} \hline
\multirow{3}{*}{	\textbf{\textit{	bawang	}}}	&	\multirow{3}{*}{	O/C	}	&	\multirow{3}{*}{	Alliums	}	&	\multirow{3}{*}{	Austronesian	}	&	\multirow{	3	}{*}{	\textit{	ms/id	 - }		bawang		}	\\&&&&				\textit{		}					\\&&&&	\textit{		}					\\\arrayrulecolor{gray} \hline
\multirow{3}{*}{	\textbf{\textit{	bawho	}}}	&	\multirow{3}{*}{	Action	}	&	\multirow{3}{*}{	protect, cherish	}	&	\multirow{3}{*}{	Sinitic	}	&	\multirow{	3	}{*}{	\textit{		}		保護		}	\\&&&&				\textit{		}					\\&&&&	\textit{		}					\\\arrayrulecolor{gray} \hline
\multirow{3}{*}{	\textbf{\textit{	bawyu	}}}	&	\multirow{3}{*}{	Action	}	&	\multirow{3}{*}{	have, possess	}	&	\multirow{3}{*}{	Sinitic	}	&	\multirow{	3	}{*}{	\textit{		}		保有		}	\\&&&&				\textit{		}					\\&&&&	\textit{		}					\\\arrayrulecolor{gray} \hline
\multirow{3}{*}{	\textbf{\textit{	bay	}}}	&	\multirow{3}{*}{	O/C	}	&	\multirow{3}{*}{	cup	}	&	\multirow{3}{*}{	Sinitic	}	&	\multirow{	3	}{*}{	\textit{		}		杯		}	\\&&&&				\textit{		}					\\&&&&	\textit{		}					\\\arrayrulecolor{gray} \hline
\multirow{3}{*}{	\textbf{\textit{	bay	}}}	&	\multirow{3}{*}{	O/C	}	&	\multirow{3}{*}{	back	}	&	\multirow{3}{*}{	Sinitic	}	&	\multirow{	3	}{*}{	\textit{		}		背		}	\\&&&&				\textit{		}					\\&&&&	\textit{		}					\\\arrayrulecolor{gray} \hline
\multirow{3}{*}{	\textbf{\textit{	baybok	}}}	&	\multirow{3}{*}{	Action	}	&	\multirow{3}{*}{	lose	}	&	\multirow{3}{*}{	Sinitic	}	&	\multirow{	3	}{*}{	\textit{		}		敗北		}	\\&&&&				\textit{		}					\\&&&&	\textit{		}					\\\arrayrulecolor{gray} \hline
\multirow{3}{*}{	\textbf{\textit{	bek	}}}	&	\multirow{3}{*}{	O/C	}	&	\multirow{3}{*}{	hundred	}	&	\multirow{3}{*}{	Sinitic	}	&	\multirow{	3	}{*}{	\textit{		}		百		}	\\&&&&				\textit{		}					\\&&&&	\textit{		}					\\\arrayrulecolor{gray} \hline
\multirow{3}{*}{	\textbf{\textit{	bekhap	}}}	&	\multirow{3}{*}{	O/C	}	&	\multirow{3}{*}{	lily, lesbian (slang) 	}	&	\multirow{3}{*}{	Sinitic	}	&	\multirow{	3	}{*}{	\textit{		}		百合		}	\\&&&&				\textit{		}					\\&&&&	\textit{		}					\\\arrayrulecolor{gray} \hline
\multirow{3}{*}{	\textbf{\textit{	bekhongsik	}}}	&	\multirow{3}{*}{	Qual	}	&	\multirow{3}{*}{	pink	}	&	\multirow{3}{*}{	Sinitic	}	&	\multirow{	3	}{*}{	\textit{		}		緣紅		}	\\&&&&				\textit{		}					\\&&&&	\textit{		}					\\\arrayrulecolor{gray} \hline
\multirow{3}{*}{	\textbf{\textit{	beksik	}}}	&	\multirow{3}{*}{	Qual	}	&	\multirow{3}{*}{	white	}	&	\multirow{3}{*}{	Sinitic	}	&	\multirow{	3	}{*}{	\textit{		}		白色		}	\\&&&&				\textit{		}					\\&&&&	\textit{		}					\\\arrayrulecolor{gray} \hline
\multirow{3}{*}{	\textbf{\textit{	ben	}}}	&	\multirow{3}{*}{	Action	}	&	\multirow{3}{*}{	change, edit, alter	}	&	\multirow{3}{*}{	Sinitic	}	&	\multirow{	3	}{*}{	\textit{		}		變		}	\\&&&&				\textit{		}					\\&&&&	\textit{		}					\\\arrayrulecolor{gray} \hline
\multirow{3}{*}{	\textbf{\textit{	benci	}}}	&	\multirow{3}{*}{	Qual	}	&	\multirow{3}{*}{	hate	}	&	\multirow{3}{*}{	Austronesian	}	&	\multirow{	3	}{*}{	\textit{	ms/id	 - }		benci		}	\\&&&&				\textit{		}					\\&&&&	\textit{		}					\\\arrayrulecolor{gray} \hline
\multirow{3}{*}{	\textbf{\textit{	beng	}}}	&	\multirow{3}{*}{	O/C	}	&	\multirow{3}{*}{	sick, illness	}	&	\multirow{3}{*}{	Sinitic	}	&	\multirow{	3	}{*}{	\textit{		}		病		}	\\&&&&				\textit{		}					\\&&&&	\textit{		}					\\\arrayrulecolor{gray} \hline
\multirow{3}{*}{	\textbf{\textit{	bengtan	}}}	&	\multirow{3}{*}{	Qual	}	&	\multirow{3}{*}{	flat	}	&	\multirow{3}{*}{	Sinitic	}	&	\multirow{	3	}{*}{	\textit{		}		平坦		}	\\&&&&				\textit{		}					\\&&&&	\textit{		}					\\\arrayrulecolor{gray} \hline
\multirow{3}{*}{	\textbf{\textit{	benmay	}}}	&	\multirow{3}{*}{	Action	}	&	\multirow{3}{*}{	sell, trade	}	&	\multirow{3}{*}{	Sinitic	}	&	\multirow{	3	}{*}{	\textit{		}		販賣		}	\\&&&&				\textit{		}					\\&&&&	\textit{		}					\\\arrayrulecolor{gray} \hline
\multirow{3}{*}{	\textbf{\textit{	benmey	}}}	&	\multirow{3}{*}{	O/C	}	&	\multirow{3}{*}{	rice	}	&	\multirow{3}{*}{	Sinitic	}	&	\multirow{	3	}{*}{	\textit{		}		飯米		}	\\&&&&				\textit{		}					\\&&&&	\textit{		}					\\\arrayrulecolor{gray} \hline
\multirow{3}{*}{	\textbf{\textit{	bentay	}}}	&	\multirow{3}{*}{	Qual	}	&	\multirow{3}{*}{	perverse, hentai, pervert	}	&	\multirow{3}{*}{	Sinitic	}	&	\multirow{	3	}{*}{	\textit{		}		変態		}	\\&&&&				\textit{		}					\\&&&&	\textit{		}					\\\arrayrulecolor{gray} \hline
\multirow{3}{*}{	\textbf{\textit{	betgen	}}}	&	\multirow{3}{*}{	Action	}	&	\multirow{3}{*}{	find	}	&	\multirow{3}{*}{	Sinitic	}	&	\multirow{	3	}{*}{	\textit{		}		發見		}	\\&&&&				\textit{		}					\\&&&&	\textit{		}					\\\arrayrulecolor{gray} \hline
\multirow{3}{*}{	\textbf{\textit{	betgyaw	}}}	&	\multirow{3}{*}{	Action	}	&	\multirow{3}{*}{	fermenting	}	&	\multirow{3}{*}{	Sinitic	}	&	\multirow{	3	}{*}{	\textit{		}		醱酵		}	\\&&&&				\textit{		}					\\&&&&	\textit{		}					\\\arrayrulecolor{gray} \hline
\multirow{3}{*}{	\textbf{\textit{	bey	}}}	&	\multirow{3}{*}{	Action	}	&	\multirow{3}{*}{	close; shut	}	&	\multirow{3}{*}{	Sinitic	}	&	\multirow{	3	}{*}{	\textit{		}		閉		}	\\&&&&				\textit{		}					\\&&&&	\textit{		}					\\\arrayrulecolor{gray} \hline
\multirow{3}{*}{	\textbf{\textit{	bi'amahi	}}}	&	\multirow{3}{*}{	Qual	}	&	\multirow{3}{*}{	bland, tasteless	}	&	\multirow{3}{*}{	Compound	}	&	\multirow{	3	}{*}{	\textit{		}		bi + amahi		}	\\&&&&				\textit{		}					\\&&&&	\textit{		}					\\\arrayrulecolor{gray} \hline
\multirow{3}{*}{	\textbf{\textit{	biay	}}}	&	\multirow{3}{*}{	Qual	}	&	\multirow{3}{*}{	sad	}	&	\multirow{3}{*}{	Sinitic	}	&	\multirow{	3	}{*}{	\textit{		}		悲哀		}	\\&&&&				\textit{		}					\\&&&&	\textit{		}					\\\arrayrulecolor{gray} \hline
\multirow{3}{*}{	\textbf{\textit{	bigyaw	}}}	&	\multirow{3}{*}{	Action	}	&	\multirow{3}{*}{	compare, comparison	}	&	\multirow{3}{*}{	Sinitic	}	&	\multirow{	3	}{*}{	\textit{		}		比較		}	\\&&&&				\textit{		}					\\&&&&	\textit{		}					\\\arrayrulecolor{gray} \hline
\multirow{3}{*}{	\textbf{\textit{	bihenggi	}}}	&	\multirow{3}{*}{	O/C	}	&	\multirow{3}{*}{	airplane	}	&	\multirow{3}{*}{		}	&	\multirow{	3	}{*}{	\textit{		}				}	\\&&&&				\textit{		}					\\&&&&	\textit{		}					\\\arrayrulecolor{gray} \hline
\multirow{3}{*}{	\textbf{\textit{	binkon	}}}	&	\multirow{3}{*}{	Qual	}	&	\multirow{3}{*}{	poor	}	&	\multirow{3}{*}{	Sinitic	}	&	\multirow{	3	}{*}{	\textit{		}		貧困		}	\\&&&&				\textit{		}					\\&&&&	\textit{		}					\\\arrayrulecolor{gray} \hline
\multirow{3}{*}{	\textbf{\textit{	bityaw	}}}	&	\multirow{3}{*}{	Action	}	&	\multirow{3}{*}{	need	}	&	\multirow{3}{*}{	Sinitic	}	&	\multirow{	3	}{*}{	\textit{		}		必要		}	\\&&&&				\textit{		}					\\&&&&	\textit{		}					\\\arrayrulecolor{gray} \hline
\multirow{3}{*}{	\textbf{\textit{	bobo	}}}	&	\multirow{3}{*}{	Qual	}	&	\multirow{3}{*}{	stupid	}	&	\multirow{3}{*}{	Sound-based	}	&	\multirow{	3	}{*}{	\textit{		}				}	\\&&&&				\textit{		}					\\&&&&	\textit{		}					\\\arrayrulecolor{gray} \hline
\multirow{3}{*}{	\textbf{\textit{	bokul	}}}	&	\multirow{3}{*}{	Action	}	&	\multirow{3}{*}{	boil	}	&	\multirow{3}{*}{	Sound-based	}	&	\multirow{	3	}{*}{	\textit{		}				}	\\&&&&				\textit{		}					\\&&&&	\textit{		}					\\\arrayrulecolor{gray} \hline
\multirow{3}{*}{	\textbf{\textit{	boncow	}}}	&	\multirow{3}{*}{	O/C	}	&	\multirow{3}{*}{	herbs	}	&	\multirow{3}{*}{	Sinitic	}	&	\multirow{	3	}{*}{	\textit{		}		本草		}	\\&&&&				\textit{		}					\\&&&&	\textit{		}					\\\arrayrulecolor{gray} \hline
\multirow{3}{*}{	\textbf{\textit{	bongmit	}}}	&	\multirow{3}{*}{	O/C	}	&	\multirow{3}{*}{	honey	}	&	\multirow{3}{*}{	Sinitic	}	&	\multirow{	3	}{*}{	\textit{		}		蜂蜜		}	\\&&&&				\textit{		}					\\&&&&	\textit{		}					\\\arrayrulecolor{gray} \hline
\multirow{3}{*}{	\textbf{\textit{	boti	}}}	&	\multirow{3}{*}{	O/C	}	&	\multirow{3}{*}{	star	}	&	\multirow{3}{*}{	Koreo-Japonic	}	&	\multirow{	2	}{*}{	\textit{	ko	 - }		별		}	\\&&&&	\multirow{	2	}{*}{	\textit{	ja	 - }		ほし		}	\\&&&&	\textit{		}					\\\arrayrulecolor{gray} \hline
\multirow{3}{*}{	\textbf{\textit{	bu'e	}}}	&	\multirow{3}{*}{	O/C	}	&	\multirow{3}{*}{	boat, ship	}	&	\multirow{3}{*}{	Koreo-Japonic	}	&	\multirow{	2	}{*}{	\textit{	ko	 - }		배		}	\\&&&&	\multirow{	2	}{*}{	\textit{	ja	 - }		ふね		}	\\&&&&	\textit{		}					\\\arrayrulecolor{gray} \hline
\multirow{3}{*}{	\textbf{\textit{	bulengket	}}}	&	\multirow{3}{*}{	O/C	}	&	\multirow{3}{*}{	blanket	}	&	\multirow{3}{*}{	Western: English	}	&	\multirow{	3	}{*}{	\textit{	en	 - }		blanket		}	\\&&&&				\textit{		}					\\&&&&	\textit{		}					\\\arrayrulecolor{gray} \hline
\multirow{3}{*}{	\textbf{\textit{	bunga	}}}	&	\multirow{3}{*}{	O/C	}	&	\multirow{3}{*}{	flower	}	&	\multirow{3}{*}{	Austronesian	}	&	\multirow{	3	}{*}{	\textit{	ms/id	 - }		bunga		}	\\&&&&				\textit{		}					\\&&&&	\textit{		}					\\\arrayrulecolor{gray} \hline
\multirow{3}{*}{	\textbf{\textit{	buni	}}}	&	\multirow{3}{*}{	O/C	}	&	\multirow{3}{*}{	sound, noise	}	&	\multirow{3}{*}{	Austronesian	}	&	\multirow{	2	}{*}{	\textit{	ms/id	 - }		bunyi		}	\\&&&&	\multirow{	2	}{*}{	\textit{	tl	 - }		bunyi		}	\\&&&&	\textit{		}					\\\arrayrulecolor{gray} \hline
\multirow{3}{*}{	\textbf{\textit{	bunsek	}}}	&	\multirow{3}{*}{	Action	}	&	\multirow{3}{*}{	analyze; (ling) Analytic	}	&	\multirow{3}{*}{	Sinitic	}	&	\multirow{	3	}{*}{	\textit{		}		分析		}	\\&&&&				\textit{		}					\\&&&&	\textit{		}					\\\arrayrulecolor{gray} \hline
\multirow{3}{*}{	\textbf{\textit{	busyang	}}}	&	\multirow{3}{*}{	Action	}	&	\multirow{3}{*}{	injure, get hurt	}	&	\multirow{3}{*}{	Sinitic	}	&	\multirow{	3	}{*}{	\textit{		}		負傷		}	\\&&&&				\textit{		}					\\&&&&	\textit{		}					\\\arrayrulecolor{gray} \hline
\multirow{3}{*}{	\textbf{\textit{	butaw	}}}	&	\multirow{3}{*}{	Action	}	&	\multirow{3}{*}{	dislike	}	&	\multirow{3}{*}{	Sinitic	}	&	\multirow{	3	}{*}{	\textit{		}		不好		}	\\&&&&				\textit{		}					\\&&&&	\textit{		}					\\\arrayrulecolor{gray} \hline
\multirow{3}{*}{	\textbf{\textit{	buthaw	}}}	&	\multirow{3}{*}{	Action	}	&	\multirow{3}{*}{	dislike	}	&	\multirow{3}{*}{	Sinitic	}	&	\multirow{	3	}{*}{	\textit{		}		不好		}	\\&&&&				\textit{		}					\\&&&&	\textit{		}					\\\arrayrulecolor{gray} \hline
\multirow{3}{*}{	\textbf{\textit{	buwaya	}}}	&	\multirow{3}{*}{	O/C	}	&	\multirow{3}{*}{	croc/allegator	}	&	\multirow{3}{*}{	Austronesian	}	&	\multirow{	2	}{*}{	\textit{	ms/id	 - }		buaya		}	\\&&&&	\multirow{	2	}{*}{	\textit{	tl	 - }		buwaya		}	\\&&&&	\textit{		}					\\\arrayrulecolor{gray} \hline
\multirow{3}{*}{	\textbf{\textit{	buyu	}}}	&	\multirow{3}{*}{	Qual	}	&	\multirow{3}{*}{	rich, wealthy	}	&	\multirow{3}{*}{	Sinitic	}	&	\multirow{	3	}{*}{	\textit{		}		富有		}	\\&&&&				\textit{		}					\\&&&&	\textit{		}					\\\arrayrulecolor{gray} \hline
\multirow{3}{*}{	\textbf{\textit{	calat	}}}	&	\multirow{3}{*}{	Qual	}	&	\multirow{3}{*}{	smart	}	&	\multirow{3}{*}{	Austroasiatic	}	&	\multirow{	2	}{*}{	\textit{	th	 - }	\textthai{	ฉลาด	}	}	\\&&&&	\multirow{	2	}{*}{	\textit{	km	 - }	\textkhmer{	ឆ្លាត	}	}	\\&&&&	\textit{		}					\\\arrayrulecolor{gray} \hline
\multirow{3}{*}{	\textbf{\textit{	calenji	}}}	&	\multirow{3}{*}{	Action	}	&	\multirow{3}{*}{	challenge	}	&	\multirow{3}{*}{	Western: English	}	&	\multirow{	3	}{*}{	\textit{	en	 - }		challenge		}	\\&&&&				\textit{		}					\\&&&&	\textit{		}					\\\arrayrulecolor{gray} \hline
\multirow{3}{*}{	\textbf{\textit{	camuk	}}}	&	\multirow{3}{*}{	Action	}	&	\multirow{3}{*}{	chew, crush (in mouth)	}	&	\multirow{3}{*}{	Sound-based	}	&	\multirow{	3	}{*}{	\textit{		}				}	\\&&&&				\textit{		}					\\&&&&	\textit{		}					\\\arrayrulecolor{gray} \hline
\multirow{3}{*}{	\textbf{\textit{	can	}}}	&	\multirow{3}{*}{	O/C	}	&	\multirow{3}{*}{	plate	}	&	\multirow{3}{*}{	Austroasiatic	}	&				\textit{	th	 - }	\textthai{	จาน	}		\\&&&&				\textit{	lo	 - }	\textlao{	ຈານ	}		\\&&&&	\textit{	km	 - }	\textkhmer{	ចាន	}		\\\arrayrulecolor{gray} \hline
\multirow{3}{*}{	\textbf{\textit{	canda	}}}	&	\multirow{3}{*}{	O/C	}	&	\multirow{3}{*}{	month, moon	}	&	\multirow{3}{*}{	Sanskrit	}	&	\multirow{	2	}{*}{	\textit{		}	\textsanskrit{	चन्द्र 	}	}	\\&&&&	\multirow{	2	}{*}{	\textit{		}		(candrá)		}	\\&&&&	\textit{		}					\\\arrayrulecolor{gray} \hline
\multirow{3}{*}{	\textbf{\textit{	candahali	}}}	&	\multirow{3}{*}{	O/C	}	&	\multirow{3}{*}{	monday	}	&	\multirow{3}{*}{	Compound	}	&	\multirow{	3	}{*}{	\textit{		}		canda + hali		}	\\&&&&				\textit{		}					\\&&&&	\textit{		}					\\\arrayrulecolor{gray} \hline
\multirow{3}{*}{	\textbf{\textit{	cawwet	}}}	&	\multirow{3}{*}{	Qual	}	&	\multirow{3}{*}{	surpass, go beyond, overcome, transcend	}	&	\multirow{3}{*}{	Sinitic	}	&	\multirow{	3	}{*}{	\textit{		}		超越		}	\\&&&&				\textit{		}					\\&&&&	\textit{		}					\\\arrayrulecolor{gray} \hline
\multirow{3}{*}{	\textbf{\textit{	cay'i	}}}	&	\multirow{3}{*}{	O/C	}	&	\multirow{3}{*}{	difference	}	&	\multirow{3}{*}{	Sinitic	}	&	\multirow{	3	}{*}{	\textit{		}		差異		}	\\&&&&				\textit{		}					\\&&&&	\textit{		}					\\\arrayrulecolor{gray} \hline
\multirow{3}{*}{	\textbf{\textit{	caycu	}}}	&	\multirow{3}{*}{	Action	}	&	\multirow{3}{*}{	harvest, collect, gather	}	&	\multirow{3}{*}{	Sinitic	}	&	\multirow{	3	}{*}{	\textit{		}		採取		}	\\&&&&				\textit{		}					\\&&&&	\textit{		}					\\\arrayrulecolor{gray} \hline
\multirow{3}{*}{	\textbf{\textit{	cen	}}}	&	\multirow{3}{*}{	O/C	}	&	\multirow{3}{*}{	thousand	}	&	\multirow{3}{*}{	Sinitic	}	&	\multirow{	3	}{*}{	\textit{		}		千		}	\\&&&&				\textit{		}					\\&&&&	\textit{		}					\\\arrayrulecolor{gray} \hline
\multirow{3}{*}{	\textbf{\textit{	cengcu	}}}	&	\multirow{3}{*}{	Action	}	&	\multirow{3}{*}{	hear, listen	}	&	\multirow{3}{*}{	Sinitic	}	&	\multirow{	3	}{*}{	\textit{		}		聽取		}	\\&&&&				\textit{		}					\\&&&&	\textit{		}					\\\arrayrulecolor{gray} \hline
\multirow{3}{*}{	\textbf{\textit{	cenglamsik	}}}	&	\multirow{3}{*}{	Qual	}	&	\multirow{3}{*}{	blue	}	&	\multirow{3}{*}{	Sinitic	}	&	\multirow{	3	}{*}{	\textit{		}		靑覽色		}	\\&&&&				\textit{		}					\\&&&&	\textit{		}					\\\arrayrulecolor{gray} \hline
\multirow{3}{*}{	\textbf{\textit{	cengloksik	}}}	&	\multirow{3}{*}{	Qual	}	&	\multirow{3}{*}{	green	}	&	\multirow{3}{*}{	Sinitic	}	&	\multirow{	3	}{*}{	\textit{		}		靑綠色		}	\\&&&&				\textit{		}					\\&&&&	\textit{		}					\\\arrayrulecolor{gray} \hline
\multirow{3}{*}{	\textbf{\textit{	cengsik	}}}	&	\multirow{3}{*}{	Qual	}	&	\multirow{3}{*}{	blue-green	}	&	\multirow{3}{*}{	Sinitic	}	&	\multirow{	3	}{*}{	\textit{		}		靑色		}	\\&&&&				\textit{		}					\\&&&&	\textit{		}					\\\arrayrulecolor{gray} \hline
\multirow{3}{*}{	\textbf{\textit{	cetcu	}}}	&	\multirow{3}{*}{	Action	}	&	\multirow{3}{*}{	steal	}	&	\multirow{3}{*}{	Sinitic	}	&	\multirow{	3	}{*}{	\textit{		}		竊取		}	\\&&&&				\textit{		}					\\&&&&	\textit{		}					\\\arrayrulecolor{gray} \hline
\multirow{3}{*}{	\textbf{\textit{	cici	}}}	&	\multirow{3}{*}{	O/C	}	&	\multirow{3}{*}{	breasts (informal, infantile)	}	&	\multirow{3}{*}{	Koreo-Japonic	}	&	\multirow{	2	}{*}{	\textit{	ko	 - }		젖		}	\\&&&&	\multirow{	2	}{*}{	\textit{	ja	 - }		ちち		}	\\&&&&	\textit{		}					\\\arrayrulecolor{gray} \hline
\multirow{3}{*}{	\textbf{\textit{	ciciko	}}}	&	\multirow{3}{*}{	O/C	}	&	\multirow{3}{*}{	nipple	}	&	\multirow{3}{*}{	Koreo-Japonic	}	&	\multirow{	3	}{*}{	\textit{		}		cici + ko		}	\\&&&&				\textit{		}					\\&&&&	\textit{		}					\\\arrayrulecolor{gray} \hline
\multirow{3}{*}{	\textbf{\textit{	cikcak	}}}	&	\multirow{3}{*}{	O/C	}	&	\multirow{3}{*}{	lizard	}	&	\multirow{3}{*}{	Compound	}	&	\multirow{	2	}{*}{	\textit{	ms/id	 - }		cicak		}	\\&&&&	\multirow{	2	}{*}{	\textit{	lo	 - }	\textlao{	ជីងចក់	}	}	\\&&&&	\textit{		}					\\\arrayrulecolor{gray} \hline
\multirow{3}{*}{	\textbf{\textit{	cimjang	}}}	&	\multirow{3}{*}{	O/C	}	&	\multirow{3}{*}{	bed	}	&	\multirow{3}{*}{	Sinitic	}	&	\multirow{	3	}{*}{	\textit{		}		寢床; 寢牀		}	\\&&&&				\textit{		}					\\&&&&	\textit{		}					\\\arrayrulecolor{gray} \hline
\multirow{3}{*}{	\textbf{\textit{	cingi	}}}	&	\multirow{3}{*}{	O/C	}	&	\multirow{3}{*}{	herder	}	&	\multirow{3}{*}{	Altaic	}	&				\textit{	mn	 - }		малчин			\\&&&&				\textit{	ko	 - }		치기			\\&&&&	\textit{	jp	 - }		飼(か)う			\\\arrayrulecolor{gray} \hline
\multirow{3}{*}{	\textbf{\textit{	cinyu	}}}	&	\multirow{3}{*}{	O/C	}	&	\multirow{3}{*}{	friend	}	&	\multirow{3}{*}{	Sinitic	}	&	\multirow{	3	}{*}{	\textit{		}		親友		}	\\&&&&				\textit{		}					\\&&&&	\textit{		}					\\\arrayrulecolor{gray} \hline
\multirow{3}{*}{	\textbf{\textit{	cisyang	}}}	&	\multirow{3}{*}{	Action	}	&	\multirow{3}{*}{	stab	}	&	\multirow{3}{*}{	Sinitic	}	&	\multirow{	3	}{*}{	\textit{		}		刺傷		}	\\&&&&				\textit{		}					\\&&&&	\textit{		}					\\\arrayrulecolor{gray} \hline
\multirow{3}{*}{	\textbf{\textit{	cit	}}}	&	\multirow{3}{*}{	O/C	}	&	\multirow{3}{*}{	seven	}	&	\multirow{3}{*}{	Sinitic	}	&	\multirow{	3	}{*}{	\textit{		}		七		}	\\&&&&				\textit{		}					\\&&&&	\textit{		}					\\\arrayrulecolor{gray} \hline
\multirow{3}{*}{	\textbf{\textit{	cita	}}}	&	\multirow{3}{*}{	O/C	}	&	\multirow{3}{*}{	mind	}	&	\multirow{3}{*}{	Sanskrit	}	&	\multirow{	2	}{*}{	\textit{		}	\textsanskrit{	चित्त 	}	}	\\&&&&	\multirow{	2	}{*}{	\textit{		}		(citta)		}	\\&&&&	\textit{		}					\\\arrayrulecolor{gray} \hline
\multirow{3}{*}{	\textbf{\textit{	cita ji beng	}}}	&	\multirow{3}{*}{	O/C	}	&	\multirow{3}{*}{	mental illness	}	&	\multirow{3}{*}{	Compound	}	&	\multirow{	3	}{*}{	\textit{		}		cita + bing		}	\\&&&&				\textit{		}					\\&&&&	\textit{		}					\\\arrayrulecolor{gray} \hline
\multirow{3}{*}{	\textbf{\textit{	citawita	}}}	&	\multirow{3}{*}{	O/C	}	&	\multirow{3}{*}{	psychology	}	&	\multirow{3}{*}{	Sanskrit	}	&	\multirow{	2	}{*}{	\textit{		}	\textsanskrit{	चित्त + विद्या 	}	}	\\&&&&	\multirow{	2	}{*}{	\textit{		}		(citta + vidyā)		}	\\&&&&	\textit{		}					\\\arrayrulecolor{gray} \hline
\multirow{3}{*}{	\textbf{\textit{	co	}}}	&	\multirow{3}{*}{	O/C	}	&	\multirow{3}{*}{	that	}	&	\multirow{3}{*}{	N/A	}	&	\multirow{	3	}{*}{	\textit{		}				}	\\&&&&				\textit{		}					\\&&&&	\textit{		}					\\\arrayrulecolor{gray} \hline
\multirow{3}{*}{	\textbf{\textit{	coba	}}}	&	\multirow{3}{*}{	O/C	}	&	\multirow{3}{*}{	grandpa, grandfather	}	&	\multirow{3}{*}{	Sinitic	}	&	\multirow{	3	}{*}{	\textit{		}		祖父		}	\\&&&&				\textit{		}					\\&&&&	\textit{		}					\\\arrayrulecolor{gray} \hline
\multirow{3}{*}{	\textbf{\textit{	cobama	}}}	&	\multirow{3}{*}{	O/C	}	&	\multirow{3}{*}{	grandparents, elders	}	&	\multirow{3}{*}{	Sinitic	}	&	\multirow{	3	}{*}{	\textit{		}		祖父母		}	\\&&&&				\textit{		}					\\&&&&	\textit{		}					\\\arrayrulecolor{gray} \hline
\multirow{3}{*}{	\textbf{\textit{	codi	}}}	&	\multirow{3}{*}{	O/C	}	&	\multirow{3}{*}{	there	}	&	\multirow{3}{*}{	Compound	}	&	\multirow{	3	}{*}{	\textit{		}		co + di		}	\\&&&&				\textit{		}					\\&&&&	\textit{		}					\\\arrayrulecolor{gray} \hline
\multirow{3}{*}{	\textbf{\textit{	cokcok	}}}	&	\multirow{3}{*}{	Action	}	&	\multirow{3}{*}{	kiss, smooch, peck	}	&	\multirow{3}{*}{	Sound-based	}	&	\multirow{	3	}{*}{	\textit{		}				}	\\&&&&				\textit{		}					\\&&&&	\textit{		}					\\\arrayrulecolor{gray} \hline
\multirow{3}{*}{	\textbf{\textit{	coma	}}}	&	\multirow{3}{*}{	O/C	}	&	\multirow{3}{*}{	grandma, grandmother	}	&	\multirow{3}{*}{	Sinitic	}	&	\multirow{	3	}{*}{	\textit{		}		祖母		}	\\&&&&				\textit{		}					\\&&&&	\textit{		}					\\\arrayrulecolor{gray} \hline
\multirow{3}{*}{	\textbf{\textit{	cono	}}}	&	\multirow{3}{*}{	O/C	}	&	\multirow{3}{*}{	wolf	}	&	\multirow{3}{*}{	Altaic	}	&	\multirow{	3	}{*}{	\textit{	mn	 - }		чоно		}	\\&&&&				\textit{		}					\\&&&&	\textit{		}					\\\arrayrulecolor{gray} \hline
\multirow{3}{*}{	\textbf{\textit{	coy	}}}	&	\multirow{3}{*}{	Action	}	&	\multirow{3}{*}{	help	}	&	\multirow{3}{*}{	Sinitic	}	&	\multirow{	3	}{*}{	\textit{		}		助		}	\\&&&&				\textit{		}					\\&&&&	\textit{		}					\\\arrayrulecolor{gray} \hline
\multirow{3}{*}{	\textbf{\textit{	cu	}}}	&	\multirow{3}{*}{	Action	}	&	\multirow{3}{*}{	catch, grab, hunt	}	&	\multirow{3}{*}{	Sinitic	}	&	\multirow{	3	}{*}{	\textit{		}		取		}	\\&&&&				\textit{		}					\\&&&&	\textit{		}					\\\arrayrulecolor{gray} \hline
\multirow{3}{*}{	\textbf{\textit{	cucut	}}}	&	\multirow{3}{*}{	Action	}	&	\multirow{3}{*}{	take out, pluck, remove, delete	}	&	\multirow{3}{*}{	Sinitic	}	&	\multirow{	3	}{*}{	\textit{		}		取出		}	\\&&&&				\textit{		}					\\&&&&	\textit{		}					\\\arrayrulecolor{gray} \hline
\multirow{3}{*}{	\textbf{\textit{	cuk	}}}	&	\multirow{3}{*}{	Action	}	&	\multirow{3}{*}{	poke, pick at (something), touch	}	&	\multirow{3}{*}{	Sinitic	}	&	\multirow{	3	}{*}{	\textit{		}		觸		}	\\&&&&				\textit{		}					\\&&&&	\textit{		}					\\\arrayrulecolor{gray} \hline
\multirow{3}{*}{	\textbf{\textit{	cutkow	}}}	&	\multirow{3}{*}{	O/C	}	&	\multirow{3}{*}{	exit	}	&	\multirow{3}{*}{		}	&	\multirow{	3	}{*}{	\textit{		}				}	\\&&&&				\textit{		}					\\&&&&	\textit{		}					\\\arrayrulecolor{gray} \hline
\multirow{3}{*}{	\textbf{\textit{	da	}}}	&	\multirow{3}{*}{	Qual	}	&	\multirow{3}{*}{	more, many, a lot	}	&	\multirow{3}{*}{	Sinitic	}	&	\multirow{	3	}{*}{	\textit{		}		多		}	\\&&&&				\textit{		}					\\&&&&	\textit{		}					\\\arrayrulecolor{gray} \hline
\multirow{3}{*}{	\textbf{\textit{	dadan	}}}	&	\multirow{3}{*}{	Qual	}	&	\multirow{3}{*}{	short	}	&	\multirow{3}{*}{	Sinitic	}	&	\multirow{	3	}{*}{	\textit{		}		多短		}	\\&&&&				\textit{		}					\\&&&&	\textit{		}					\\\arrayrulecolor{gray} \hline
\multirow{3}{*}{	\textbf{\textit{	daging	}}}	&	\multirow{3}{*}{	O/C	}	&	\multirow{3}{*}{	flesh, muscle, meat	}	&	\multirow{3}{*}{	Austronesian	}	&	\multirow{	3	}{*}{	\textit{	ms/id	 - }		daging		}	\\&&&&				\textit{		}					\\&&&&	\textit{		}					\\\arrayrulecolor{gray} \hline
\multirow{3}{*}{	\textbf{\textit{	dagu	}}}	&	\multirow{3}{*}{	O/C	}	&	\multirow{3}{*}{	chin	}	&	\multirow{3}{*}{	Austronesian	}	&	\multirow{	3	}{*}{	\textit{	ms/id	 - }		dagu		}	\\&&&&				\textit{		}					\\&&&&	\textit{		}					\\\arrayrulecolor{gray} \hline
\multirow{3}{*}{	\textbf{\textit{	dajang	}}}	&	\multirow{3}{*}{	Qual	}	&	\multirow{3}{*}{	tall	}	&	\multirow{3}{*}{	Sinitic	}	&	\multirow{	3	}{*}{	\textit{		}		多長		}	\\&&&&				\textit{		}					\\&&&&	\textit{		}					\\\arrayrulecolor{gray} \hline
\multirow{3}{*}{	\textbf{\textit{	dala	}}}	&	\multirow{3}{*}{	O/C	}	&	\multirow{3}{*}{	blood	}	&	\multirow{3}{*}{	Austronesian	}	&	\multirow{	3	}{*}{	\textit{	ms/id	 - }		darah		}	\\&&&&				\textit{		}					\\&&&&	\textit{		}					\\\arrayrulecolor{gray} \hline
\multirow{3}{*}{	\textbf{\textit{	danggwa	}}}	&	\multirow{3}{*}{	O/C	}	&	\multirow{3}{*}{	candy, sweets	}	&	\multirow{3}{*}{	Sinitic	}	&	\multirow{	3	}{*}{	\textit{		}		糖菓		}	\\&&&&				\textit{		}					\\&&&&	\textit{		}					\\\arrayrulecolor{gray} \hline
\multirow{3}{*}{	\textbf{\textit{	dawdaw	}}}	&	\multirow{3}{*}{	O/C	}	&	\multirow{3}{*}{	peach	}	&	\multirow{3}{*}{	Sinitic	}	&	\multirow{	3	}{*}{	\textit{		}		桃		}	\\&&&&				\textit{		}					\\&&&&	\textit{		}					\\\arrayrulecolor{gray} \hline
\multirow{3}{*}{	\textbf{\textit{	dawgu	}}}	&	\multirow{3}{*}{	O/C	}	&	\multirow{3}{*}{	tools	}	&	\multirow{3}{*}{	Sinitic	}	&	\multirow{	3	}{*}{	\textit{		}		道具		}	\\&&&&				\textit{		}					\\&&&&	\textit{		}					\\\arrayrulecolor{gray} \hline
\multirow{3}{*}{	\textbf{\textit{	dawmang	}}}	&	\multirow{3}{*}{	Action	}	&	\multirow{3}{*}{	flee, run away, escape	}	&	\multirow{3}{*}{	Sinitic	}	&	\multirow{	3	}{*}{	\textit{		}		逃亡		}	\\&&&&				\textit{		}					\\&&&&	\textit{		}					\\\arrayrulecolor{gray} \hline
\multirow{3}{*}{	\textbf{\textit{	dawtat	}}}	&	\multirow{3}{*}{	Action	}	&	\multirow{3}{*}{	arrive	}	&	\multirow{3}{*}{	Sinitic	}	&	\multirow{	3	}{*}{	\textit{		}		到達		}	\\&&&&				\textit{		}					\\&&&&	\textit{		}					\\\arrayrulecolor{gray} \hline
\multirow{3}{*}{	\textbf{\textit{	day	}}}	&	\multirow{3}{*}{	Qual	}	&	\multirow{3}{*}{	big, large	}	&	\multirow{3}{*}{	Sinitic	}	&	\multirow{	3	}{*}{	\textit{		}		大		}	\\&&&&				\textit{		}					\\&&&&	\textit{		}					\\\arrayrulecolor{gray} \hline
\multirow{3}{*}{	\textbf{\textit{	daygun	}}}	&	\multirow{3}{*}{	O/C	}	&	\multirow{3}{*}{	archduke, prince consort	}	&	\multirow{3}{*}{	Sinitic	}	&	\multirow{	3	}{*}{	\textit{		}		大君		}	\\&&&&				\textit{		}					\\&&&&	\textit{		}					\\\arrayrulecolor{gray} \hline
\multirow{3}{*}{	\textbf{\textit{	dayhak	}}}	&	\multirow{3}{*}{	O/C	}	&	\multirow{3}{*}{	college, university	}	&	\multirow{3}{*}{	Sinitic	}	&	\multirow{	3	}{*}{	\textit{		}		大學		}	\\&&&&				\textit{		}					\\&&&&	\textit{		}					\\\arrayrulecolor{gray} \hline
\multirow{3}{*}{	\textbf{\textit{	dayho	}}}	&	\multirow{3}{*}{	O/C	}	&	\multirow{3}{*}{	lake; inland sea	}	&	\multirow{3}{*}{	Sinitic	}	&	\multirow{	3	}{*}{	\textit{		}		大湖		}	\\&&&&				\textit{		}					\\&&&&	\textit{		}					\\\arrayrulecolor{gray} \hline
\multirow{3}{*}{	\textbf{\textit{	dek-sap	}}}	&	\multirow{3}{*}{	O/C	}	&	\multirow{3}{*}{	adjective, quality	}	&	\multirow{3}{*}{	Compound	}	&	\multirow{	3	}{*}{	\textit{		}		dek + sap		}	\\&&&&				\textit{		}					\\&&&&	\textit{		}					\\\arrayrulecolor{gray} \hline
\multirow{3}{*}{	\textbf{\textit{	deng	}}}	&	\multirow{3}{*}{	Action	}	&	\multirow{3}{*}{	wait	}	&	\multirow{3}{*}{		}	&	\multirow{	3	}{*}{	\textit{		}				}	\\&&&&				\textit{		}					\\&&&&	\textit{		}					\\\arrayrulecolor{gray} \hline
\multirow{3}{*}{	\textbf{\textit{	denggek	}}}	&	\multirow{3}{*}{	Action	}	&	\multirow{3}{*}{	punch, hit, slap	}	&	\multirow{3}{*}{	Sinitic	}	&	\multirow{	3	}{*}{	\textit{		}		打擊		}	\\&&&&				\textit{		}					\\&&&&	\textit{		}					\\\arrayrulecolor{gray} \hline
\multirow{3}{*}{	\textbf{\textit{	dengji	}}}	&	\multirow{3}{*}{	Action	}	&	\multirow{3}{*}{	stop, end, cease, turn off	}	&	\multirow{3}{*}{	Sinitic	}	&	\multirow{	3	}{*}{	\textit{		}		停止		}	\\&&&&				\textit{		}					\\&&&&	\textit{		}					\\\arrayrulecolor{gray} \hline
\multirow{3}{*}{	\textbf{\textit{	deydey	}}}	&	\multirow{3}{*}{	O/C	}	&	\multirow{3}{*}{	younger brother	}	&	\multirow{3}{*}{	Sinitic	}	&	\multirow{	3	}{*}{	\textit{		}		弟弟		}	\\&&&&				\textit{		}					\\&&&&	\textit{		}					\\\arrayrulecolor{gray} \hline
\multirow{3}{*}{	\textbf{\textit{	digyow	}}}	&	\multirow{3}{*}{	O/C	}	&	\multirow{3}{*}{	earth (planet)	}	&	\multirow{3}{*}{	Sinitic	}	&	\multirow{	3	}{*}{	\textit{		}		地球		}	\\&&&&				\textit{		}					\\&&&&	\textit{		}					\\\arrayrulecolor{gray} \hline
\multirow{3}{*}{	\textbf{\textit{	ding'in	}}}	&	\multirow{3}{*}{	Qual	}	&	\multirow{3}{*}{	cold, chilled	}	&	\multirow{3}{*}{	Austronesian	}	&	\multirow{	3	}{*}{	\textit{	ms/id	 - }		dingin		}	\\&&&&				\textit{		}					\\&&&&	\textit{		}					\\\arrayrulecolor{gray} \hline
\multirow{3}{*}{	\textbf{\textit{	dok	}}}	&	\multirow{3}{*}{	Action	}	&	\multirow{3}{*}{	read	}	&	\multirow{3}{*}{	Sinitic	}	&	\multirow{	3	}{*}{	\textit{		}		讀		}	\\&&&&				\textit{		}					\\&&&&	\textit{		}					\\\arrayrulecolor{gray} \hline
\multirow{3}{*}{	\textbf{\textit{	dolong	}}}	&	\multirow{3}{*}{	Qual	}	&	\multirow{3}{*}{	mud, dirty	}	&	\multirow{3}{*}{	Koreo-Japonic	}	&	\multirow{	2	}{*}{	\textit{	ko	 - }		더러		}	\\&&&&	\multirow{	2	}{*}{	\textit{	ja	 - }		どろ		}	\\&&&&	\textit{		}					\\\arrayrulecolor{gray} \hline
\multirow{3}{*}{	\textbf{\textit{	dongmut	}}}	&	\multirow{3}{*}{	O/C	}	&	\multirow{3}{*}{	animal	}	&	\multirow{3}{*}{	Sinitic	}	&	\multirow{	3	}{*}{	\textit{		}		動物		}	\\&&&&				\textit{		}					\\&&&&	\textit{		}					\\\arrayrulecolor{gray} \hline
\multirow{3}{*}{	\textbf{\textit{	dowjek	}}}	&	\multirow{3}{*}{	Action	}	&	\multirow{3}{*}{	throw	}	&	\multirow{3}{*}{	Sinitic	}	&	\multirow{	3	}{*}{	\textit{		}		投擲		}	\\&&&&				\textit{		}					\\&&&&	\textit{		}					\\\arrayrulecolor{gray} \hline
\multirow{3}{*}{	\textbf{\textit{	dowjek (ji) pisaw	}}}	&	\multirow{3}{*}{	O/C	}	&	\multirow{3}{*}{	spear	}	&	\multirow{3}{*}{	Compound	}	&	\multirow{	3	}{*}{	\textit{		}		dowjek + pisaw		}	\\&&&&				\textit{		}					\\&&&&	\textit{		}					\\\arrayrulecolor{gray} \hline
\multirow{3}{*}{	\textbf{\textit{	downaw	}}}	&	\multirow{3}{*}{	O/C	}	&	\multirow{3}{*}{	brain	}	&	\multirow{3}{*}{	Sinitic	}	&	\multirow{	3	}{*}{	\textit{		}		頭腦		}	\\&&&&				\textit{		}					\\&&&&	\textit{		}					\\\arrayrulecolor{gray} \hline
\multirow{3}{*}{	\textbf{\textit{	dyawmu	}}}	&	\multirow{3}{*}{	Action	}	&	\multirow{3}{*}{	dance	}	&	\multirow{3}{*}{	Sinitic	}	&	\multirow{	3	}{*}{	\textit{		}		跳舞		}	\\&&&&				\textit{		}					\\&&&&	\textit{		}					\\\arrayrulecolor{gray} \hline
\multirow{3}{*}{	\textbf{\textit{	dyawyak	}}}	&	\multirow{3}{*}{	Action	}	&	\multirow{3}{*}{	jump	}	&	\multirow{3}{*}{	Sinitic	}	&	\multirow{	3	}{*}{	\textit{		}		跳躍		}	\\&&&&				\textit{		}					\\&&&&	\textit{		}					\\\arrayrulecolor{gray} \hline
\multirow{3}{*}{	\textbf{\textit{	en'o	}}}	&	\multirow{3}{*}{	O/C	}	&	\multirow{3}{*}{	language	}	&	\multirow{3}{*}{	Sinitic	}	&	\multirow{	3	}{*}{	\textit{		}		言語		}	\\&&&&				\textit{		}					\\&&&&	\textit{		}					\\\arrayrulecolor{gray} \hline
\multirow{3}{*}{	\textbf{\textit{	eng'ung	}}}	&	\multirow{3}{*}{	O/C	}	&	\multirow{3}{*}{	hero	}	&	\multirow{3}{*}{	Sinitic	}	&	\multirow{	3	}{*}{	\textit{		}		英雄		}	\\&&&&				\textit{		}					\\&&&&	\textit{		}					\\\arrayrulecolor{gray} \hline
\multirow{3}{*}{	\textbf{\textit{	fao	}}}	&	\multirow{3}{*}{	Qual	}	&	\multirow{3}{*}{	blue-green	}	&	\multirow{3}{*}{	Koreo-Japonic	}	&	\multirow{	2	}{*}{	\textit{	ko	 - }		파란		}	\\&&&&	\multirow{	2	}{*}{	\textit{	ja	 - }		あお		}	\\&&&&	\textit{		}					\\\arrayrulecolor{gray} \hline
\multirow{3}{*}{	\textbf{\textit{	fao bawang	}}}	&	\multirow{3}{*}{	O/C	}	&	\multirow{3}{*}{	green onions	}	&	\multirow{3}{*}{	Compound	}	&	\multirow{	3	}{*}{	\textit{		}				}	\\&&&&				\textit{		}					\\&&&&	\textit{		}					\\\arrayrulecolor{gray} \hline
\multirow{3}{*}{	\textbf{\textit{	fikali	}}}	&	\multirow{3}{*}{	O/C	}	&	\multirow{3}{*}{	shine, glow, light	}	&	\multirow{3}{*}{	Koreo-Japonic	}	&	\multirow{	2	}{*}{	\textit{	ko	 - }		빛		}	\\&&&&	\multirow{	2	}{*}{	\textit{	ja	 - }		ひかり		}	\\&&&&	\textit{		}					\\\arrayrulecolor{gray} \hline
\multirow{3}{*}{	\textbf{\textit{	fone	}}}	&	\multirow{3}{*}{	O/C	}	&	\multirow{3}{*}{	bone	}	&	\multirow{3}{*}{	Koreo-Japonic	}	&	\multirow{	3	}{*}{	\textit{	ko	 - }		뼈		}	\\&&&&				\textit{		}					\\&&&&	\textit{		}					\\\arrayrulecolor{gray} \hline
\multirow{3}{*}{	\textbf{\textit{	fong	}}}	&	\multirow{3}{*}{	O/C	}	&	\multirow{3}{*}{	room	}	&	\multirow{3}{*}{	Compound	}	&	\multirow{	3	}{*}{	\textit{		}				}	\\&&&&				\textit{		}					\\&&&&	\textit{		}					\\\arrayrulecolor{gray} \hline
\multirow{3}{*}{	\textbf{\textit{	fuca	}}}	&	\multirow{3}{*}{	O/C	}	&	\multirow{3}{*}{	grass	}	&	\multirow{3}{*}{	Koreo-Japonic	}	&	\multirow{	2	}{*}{	\textit{	ko	 - }		풀		}	\\&&&&	\multirow{	2	}{*}{	\textit{	ja	 - }		くさ		}	\\&&&&	\textit{		}					\\\arrayrulecolor{gray} \hline
\multirow{3}{*}{	\textbf{\textit{	fulansio	}}}	&	\multirow{3}{*}{	O/C	}	&	\multirow{3}{*}{	french (language)	}	&	\multirow{3}{*}{	Compound	}	&	\multirow{	3	}{*}{	\textit{		}				}	\\&&&&				\textit{		}					\\&&&&	\textit{		}					\\\arrayrulecolor{gray} \hline
\multirow{3}{*}{	\textbf{\textit{	fuwa	}}}	&	\multirow{3}{*}{	Qual	}	&	\multirow{3}{*}{	soft, fluffy	}	&	\multirow{3}{*}{	Koreo-Japonic	}	&	\multirow{	3	}{*}{	\textit{	ja	 - }		ふわふわ		}	\\&&&&				\textit{		}					\\&&&&	\textit{		}					\\\arrayrulecolor{gray} \hline
\multirow{3}{*}{	\textbf{\textit{	ga'o	}}}	&	\multirow{3}{*}{	O/C	}	&	\multirow{3}{*}{	river	}	&	\multirow{3}{*}{	Koreo-Japonic	}	&	\multirow{	2	}{*}{	\textit{	ko	 - }		개울		}	\\&&&&	\multirow{	2	}{*}{	\textit{	ja	 - }		かわ		}	\\&&&&	\textit{		}					\\\arrayrulecolor{gray} \hline
\multirow{3}{*}{	\textbf{\textit{	gaja	}}}	&	\multirow{3}{*}{	O/C	}	&	\multirow{3}{*}{	elephant	}	&	\multirow{3}{*}{	Sanskrit	}	&	\multirow{	2	}{*}{	\textit{		}	\textsanskrit{	गज 	}	}	\\&&&&	\multirow{	2	}{*}{	\textit{		}		(gaja)		}	\\&&&&	\textit{		}					\\\arrayrulecolor{gray} \hline
\multirow{3}{*}{	\textbf{\textit{	gajek	}}}	&	\multirow{3}{*}{	O/C	}	&	\multirow{3}{*}{	house, home	}	&	\multirow{3}{*}{	Sinitic	}	&	\multirow{	3	}{*}{	\textit{		}		家宅		}	\\&&&&				\textit{		}					\\&&&&	\textit{		}					\\\arrayrulecolor{gray} \hline
\multirow{3}{*}{	\textbf{\textit{	gaji	}}}	&	\multirow{3}{*}{	O/C	}	&	\multirow{3}{*}{	value, worth, price	}	&	\multirow{3}{*}{	Sinitic	}	&	\multirow{	3	}{*}{	\textit{		}		價値		}	\\&&&&				\textit{		}					\\&&&&	\textit{		}					\\\arrayrulecolor{gray} \hline
\multirow{3}{*}{	\textbf{\textit{	gak	}}}	&	\multirow{3}{*}{	Action	}	&	\multirow{3}{*}{	carve	}	&	\multirow{3}{*}{	Sinitic	}	&	\multirow{	3	}{*}{	\textit{		}		刻		}	\\&&&&				\textit{		}					\\&&&&	\textit{		}					\\\arrayrulecolor{gray} \hline
\multirow{3}{*}{	\textbf{\textit{	gam	}}}	&	\multirow{3}{*}{	O/C	}	&	\multirow{3}{*}{	gold	}	&	\multirow{3}{*}{	Sinitic	}	&	\multirow{	3	}{*}{	\textit{		}		金		}	\\&&&&				\textit{		}					\\&&&&	\textit{		}					\\\arrayrulecolor{gray} \hline
\multirow{3}{*}{	\textbf{\textit{	gam'ensya	}}}	&	\multirow{3}{*}{		}	&	\multirow{3}{*}{	thank you	}	&	\multirow{3}{*}{	N/A	}	&	\multirow{	3	}{*}{	\textit{		}				}	\\&&&&				\textit{		}					\\&&&&	\textit{		}					\\\arrayrulecolor{gray} \hline
\multirow{3}{*}{	\textbf{\textit{	gamgak	}}}	&	\multirow{3}{*}{	Action	}	&	\multirow{3}{*}{	feel, sense	}	&	\multirow{3}{*}{	Sinitic	}	&	\multirow{	3	}{*}{	\textit{		}		感覺		}	\\&&&&				\textit{		}					\\&&&&	\textit{		}					\\\arrayrulecolor{gray} \hline
\multirow{3}{*}{	\textbf{\textit{	gamjiboti	}}}	&	\multirow{3}{*}{	O/C	}	&	\multirow{3}{*}{	Venus	}	&	\multirow{3}{*}{	Compound	}	&	\multirow{	3	}{*}{	\textit{		}		gam + hali		}	\\&&&&				\textit{		}					\\&&&&	\textit{		}					\\\arrayrulecolor{gray} \hline
\multirow{3}{*}{	\textbf{\textit{	gamjihali	}}}	&	\multirow{3}{*}{	O/C	}	&	\multirow{3}{*}{	friday	}	&	\multirow{3}{*}{	Compound	}	&	\multirow{	3	}{*}{	\textit{		}		gam + hali		}	\\&&&&				\textit{		}					\\&&&&	\textit{		}					\\\arrayrulecolor{gray} \hline
\multirow{3}{*}{	\textbf{\textit{	gamsya	}}}	&	\multirow{3}{*}{	Action	}	&	\multirow{3}{*}{	thank, thanking	}	&	\multirow{3}{*}{	Sinitic	}	&	\multirow{	3	}{*}{	\textit{		}		感謝		}	\\&&&&				\textit{		}					\\&&&&	\textit{		}					\\\arrayrulecolor{gray} \hline
\multirow{3}{*}{	\textbf{\textit{	gandan	}}}	&	\multirow{3}{*}{	Qual	}	&	\multirow{3}{*}{	easy, simple	}	&	\multirow{3}{*}{	Sinitic	}	&	\multirow{	3	}{*}{	\textit{		}		簡單		}	\\&&&&				\textit{		}					\\&&&&	\textit{		}					\\\arrayrulecolor{gray} \hline
\multirow{3}{*}{	\textbf{\textit{	ganitasat	}}}	&	\multirow{3}{*}{	O/C	}	&	\multirow{3}{*}{	math	}	&	\multirow{3}{*}{	Sanskrit	}	&	\multirow{	2	}{*}{	\textit{		}	\textsanskrit{	गणित  + शास्त्र 	}	}	\\&&&&	\multirow{	2	}{*}{	\textit{		}		(gaṇita + śāstra) 		}	\\&&&&	\textit{		}					\\\arrayrulecolor{gray} \hline
\multirow{3}{*}{	\textbf{\textit{	gawgyaw	}}}	&	\multirow{3}{*}{	O/C	}	&	\multirow{3}{*}{	high school	}	&	\multirow{3}{*}{	Sinitic	}	&	\multirow{	3	}{*}{	\textit{		}		高校		}	\\&&&&				\textit{		}					\\&&&&	\textit{		}					\\\arrayrulecolor{gray} \hline
\multirow{3}{*}{	\textbf{\textit{	gaynem	}}}	&	\multirow{3}{*}{	O/C	}	&	\multirow{3}{*}{	idea, concept, noun, O/C	}	&	\multirow{3}{*}{	Sinitic	}	&	\multirow{	3	}{*}{	\textit{		}		槪念		}	\\&&&&				\textit{		}					\\&&&&	\textit{		}					\\\arrayrulecolor{gray} \hline
\multirow{3}{*}{	\textbf{\textit{	genki	}}}	&	\multirow{3}{*}{	O/C	}	&	\multirow{3}{*}{	full-feeling; feeling good	}	&	\multirow{3}{*}{	Sinitic	}	&	\multirow{	3	}{*}{	\textit{		}		元氣		}	\\&&&&				\textit{		}					\\&&&&	\textit{		}					\\\arrayrulecolor{gray} \hline
\multirow{3}{*}{	\textbf{\textit{	geylwan	}}}	&	\multirow{3}{*}{	O/C	}	&	\multirow{3}{*}{	egg	}	&	\multirow{3}{*}{	Sinitic	}	&	\multirow{	3	}{*}{	\textit{		}		鷄卵		}	\\&&&&				\textit{		}					\\&&&&	\textit{		}					\\\arrayrulecolor{gray} \hline
\multirow{3}{*}{	\textbf{\textit{	geymu	}}}	&	\multirow{3}{*}{	O/C	}	&	\multirow{3}{*}{	game	}	&	\multirow{3}{*}{	Western: English	}	&	\multirow{	3	}{*}{	\textit{	en	 - }		game		}	\\&&&&				\textit{		}					\\&&&&	\textit{		}					\\\arrayrulecolor{gray} \hline
\multirow{3}{*}{	\textbf{\textit{	geysuk	}}}	&	\multirow{3}{*}{	Action	}	&	\multirow{3}{*}{	continue, keep (verb)	}	&	\multirow{3}{*}{	Sinitic	}	&	\multirow{	3	}{*}{	\textit{		}		繼續		}	\\&&&&				\textit{		}					\\&&&&	\textit{		}					\\\arrayrulecolor{gray} \hline
\multirow{3}{*}{	\textbf{\textit{	geyswan	}}}	&	\multirow{3}{*}{	Action	}	&	\multirow{3}{*}{	calculate	}	&	\multirow{3}{*}{	Sinitic	}	&	\multirow{	3	}{*}{	\textit{		}		計算		}	\\&&&&				\textit{		}					\\&&&&	\textit{		}					\\\arrayrulecolor{gray} \hline
\multirow{3}{*}{	\textbf{\textit{	giik	}}}	&	\multirow{3}{*}{	Action	}	&	\multirow{3}{*}{	remember, recall	}	&	\multirow{3}{*}{	Sinitic	}	&	\multirow{	3	}{*}{	\textit{		}		記憶		}	\\&&&&				\textit{		}					\\&&&&	\textit{		}					\\\arrayrulecolor{gray} \hline
\multirow{3}{*}{	\textbf{\textit{	gim'suk	}}}	&	\multirow{3}{*}{	O/C	}	&	\multirow{3}{*}{	metal	}	&	\multirow{3}{*}{	Sinitic	}	&	\multirow{	3	}{*}{	\textit{		}		金屬		}	\\&&&&				\textit{		}					\\&&&&	\textit{		}					\\\arrayrulecolor{gray} \hline
\multirow{3}{*}{	\textbf{\textit{	gimsi	}}}	&	\multirow{3}{*}{	O/C	}	&	\multirow{3}{*}{	now	}	&	\multirow{3}{*}{	Sinitic	}	&	\multirow{	3	}{*}{	\textit{		}		今時		}	\\&&&&				\textit{		}					\\&&&&	\textit{		}					\\\arrayrulecolor{gray} \hline
\multirow{3}{*}{	\textbf{\textit{	gimya	}}}	&	\multirow{3}{*}{	O/C	}	&	\multirow{3}{*}{	tonight	}	&	\multirow{3}{*}{	Sinitic	}	&	\multirow{	3	}{*}{	\textit{		}		今夜		}	\\&&&&				\textit{		}					\\&&&&	\textit{		}					\\\arrayrulecolor{gray} \hline
\multirow{3}{*}{	\textbf{\textit{	ginsi	}}}	&	\multirow{3}{*}{	O/C	}	&	\multirow{3}{*}{	soon; recently	}	&	\multirow{3}{*}{	Sinitic	}	&	\multirow{	3	}{*}{	\textit{		}		近侍		}	\\&&&&				\textit{		}					\\&&&&	\textit{		}					\\\arrayrulecolor{gray} \hline
\multirow{3}{*}{	\textbf{\textit{	gip	}}}	&	\multirow{3}{*}{	Action	}	&	\multirow{3}{*}{	give	}	&	\multirow{3}{*}{	Sinitic	}	&	\multirow{	3	}{*}{	\textit{		}		給		}	\\&&&&				\textit{		}					\\&&&&	\textit{		}					\\\arrayrulecolor{gray} \hline
\multirow{3}{*}{	\textbf{\textit{	gokga	}}}	&	\multirow{3}{*}{	O/C	}	&	\multirow{3}{*}{	nation	}	&	\multirow{3}{*}{	Sinitic	}	&	\multirow{	3	}{*}{	\textit{		}		國家		}	\\&&&&				\textit{		}					\\&&&&	\textit{		}					\\\arrayrulecolor{gray} \hline
\multirow{3}{*}{	\textbf{\textit{	golaw	}}}	&	\multirow{3}{*}{	Qual	}	&	\multirow{3}{*}{	old, elder	}	&	\multirow{3}{*}{	Sinitic	}	&	\multirow{	3	}{*}{	\textit{		}		古老		}	\\&&&&				\textit{		}					\\&&&&	\textit{		}					\\\arrayrulecolor{gray} \hline
\multirow{3}{*}{	\textbf{\textit{	gongki	}}}	&	\multirow{3}{*}{	O/C	}	&	\multirow{3}{*}{	air	}	&	\multirow{3}{*}{	Sinitic	}	&	\multirow{	3	}{*}{	\textit{		}		空氣		}	\\&&&&				\textit{		}					\\&&&&	\textit{		}					\\\arrayrulecolor{gray} \hline
\multirow{3}{*}{	\textbf{\textit{	gowjaw	}}}	&	\multirow{3}{*}{	O/C	}	&	\multirow{3}{*}{	structure	}	&	\multirow{3}{*}{	Sinitic	}	&	\multirow{	3	}{*}{	\textit{		}		構造		}	\\&&&&				\textit{		}					\\&&&&	\textit{		}					\\\arrayrulecolor{gray} \hline
\multirow{3}{*}{	\textbf{\textit{	gowmay	}}}	&	\multirow{3}{*}{	Action	}	&	\multirow{3}{*}{	buy	}	&	\multirow{3}{*}{	Sinitic	}	&	\multirow{	3	}{*}{	\textit{		}		購買		}	\\&&&&				\textit{		}					\\&&&&	\textit{		}					\\\arrayrulecolor{gray} \hline
\multirow{3}{*}{	\textbf{\textit{	gudu	}}}	&	\multirow{3}{*}{	O/C	}	&	\multirow{3}{*}{	shoes, boots, footwear	}	&	\multirow{3}{*}{	Koreo-Japonic	}	&	\multirow{	2	}{*}{	\textit{	ko	 - }		구두		}	\\&&&&	\multirow{	2	}{*}{	\textit{		}		くつ		}	\\&&&&	\textit{		}					\\\arrayrulecolor{gray} \hline
\multirow{3}{*}{	\textbf{\textit{	gulu	}}}	&	\multirow{3}{*}{	O/C	}	&	\multirow{3}{*}{	thunder	}	&	\multirow{3}{*}{	Austronesian	}	&	\multirow{	3	}{*}{	\textit{	ms/id	 - }		guluh		}	\\&&&&				\textit{		}					\\&&&&	\textit{		}					\\\arrayrulecolor{gray} \hline
\multirow{3}{*}{	\textbf{\textit{	gwa'o	}}}	&	\multirow{3}{*}{	O/C	}	&	\multirow{3}{*}{	mistake	}	&	\multirow{3}{*}{	Sinitic	}	&	\multirow{	3	}{*}{	\textit{		}		過誤		}	\\&&&&				\textit{		}					\\&&&&	\textit{		}					\\\arrayrulecolor{gray} \hline
\multirow{3}{*}{	\textbf{\textit{	gwada	}}}	&	\multirow{3}{*}{	Qual	}	&	\multirow{3}{*}{	extra, excess, remaining	}	&	\multirow{3}{*}{	Sinitic	}	&	\multirow{	3	}{*}{	\textit{		}		過多		}	\\&&&&				\textit{		}					\\&&&&	\textit{		}					\\\arrayrulecolor{gray} \hline
\multirow{3}{*}{	\textbf{\textit{	gwada syumen	}}}	&	\multirow{3}{*}{	Action	}	&	\multirow{3}{*}{	oversleep, sleep in	}	&	\multirow{3}{*}{	Sinitic	}	&	\multirow{	3	}{*}{	\textit{		}		過多睡眠		}	\\&&&&				\textit{		}					\\&&&&	\textit{		}					\\\arrayrulecolor{gray} \hline
\multirow{3}{*}{	\textbf{\textit{	gwahan	}}}	&	\multirow{3}{*}{	Qual	}	&	\multirow{3}{*}{	brave, courageous	}	&	\multirow{3}{*}{	Austroasiatic	}	&				\textit{	th	 - }	\textthai{	กล้าหาญ	}		\\&&&&				\textit{	km	 - }	\textkhmer{	ហាន	}		\\&&&&	\textit{	lo	 - }	\textlao{	ກ້າ	}		\\\arrayrulecolor{gray} \hline
\multirow{3}{*}{	\textbf{\textit{	gwaysyow	}}}	&	\multirow{3}{*}{	O/C	}	&	\multirow{3}{*}{	monster; beast	}	&	\multirow{3}{*}{	Sinitic	}	&	\multirow{	3	}{*}{	\textit{		}		怪獸		}	\\&&&&				\textit{		}					\\&&&&	\textit{		}					\\\arrayrulecolor{gray} \hline
\multirow{3}{*}{	\textbf{\textit{	gwetdeng	}}}	&	\multirow{3}{*}{	Action	}	&	\multirow{3}{*}{	decide, pick, choose	}	&	\multirow{3}{*}{	Sinitic	}	&	\multirow{	3	}{*}{	\textit{		}		決定		}	\\&&&&				\textit{		}					\\&&&&	\textit{		}					\\\arrayrulecolor{gray} \hline
\multirow{3}{*}{	\textbf{\textit{	gwisin	}}}	&	\multirow{3}{*}{	O/C	}	&	\multirow{3}{*}{	spirits; ghosts	}	&	\multirow{3}{*}{	Sinitic	}	&	\multirow{	3	}{*}{	\textit{		}		鬼神		}	\\&&&&				\textit{		}					\\&&&&	\textit{		}					\\\arrayrulecolor{gray} \hline
\multirow{3}{*}{	\textbf{\textit{	gyanglik	}}}	&	\multirow{3}{*}{	Qual	}	&	\multirow{3}{*}{	strong	}	&	\multirow{3}{*}{		}	&	\multirow{	3	}{*}{	\textit{		}				}	\\&&&&				\textit{		}					\\&&&&	\textit{		}					\\\arrayrulecolor{gray} \hline
\multirow{3}{*}{	\textbf{\textit{	gyawyuk	}}}	&	\multirow{3}{*}{	Action	}	&	\multirow{3}{*}{	teach, inform, tell	}	&	\multirow{3}{*}{	Sinitic	}	&	\multirow{	3	}{*}{	\textit{		}		敎育		}	\\&&&&				\textit{		}					\\&&&&	\textit{		}					\\\arrayrulecolor{gray} \hline
\multirow{3}{*}{	\textbf{\textit{	habuk	}}}	&	\multirow{3}{*}{	O/C	}	&	\multirow{3}{*}{	dust	}	&	\multirow{3}{*}{	Austronesian	}	&	\multirow{	3	}{*}{	\textit{	ms	 - }		habuk		}	\\&&&&				\textit{		}					\\&&&&	\textit{		}					\\\arrayrulecolor{gray} \hline
\multirow{3}{*}{	\textbf{\textit{	hajong	}}}	&	\multirow{3}{*}{	O/C	}	&	\multirow{3}{*}{	downstairs; lower class	}	&	\multirow{3}{*}{	Sinitic	}	&	\multirow{	3	}{*}{	\textit{		}		下層		}	\\&&&&				\textit{		}					\\&&&&	\textit{		}					\\\arrayrulecolor{gray} \hline
\multirow{3}{*}{	\textbf{\textit{	haka	}}}	&	\multirow{3}{*}{	Qual	}	&	\multirow{3}{*}{	red	}	&	\multirow{3}{*}{	Koreo-Japonic	}	&	\multirow{	2	}{*}{	\textit{	ko	 - }		빨간		}	\\&&&&	\multirow{	2	}{*}{	\textit{	ja	 - }		あか		}	\\&&&&	\textit{		}					\\\arrayrulecolor{gray} \hline
\multirow{3}{*}{	\textbf{\textit{	hakdem	}}}	&	\multirow{3}{*}{	O/C	}	&	\multirow{3}{*}{	grade points (school)	}	&	\multirow{3}{*}{	Sinitic	}	&	\multirow{	3	}{*}{	\textit{		}		學点		}	\\&&&&				\textit{		}					\\&&&&	\textit{		}					\\\arrayrulecolor{gray} \hline
\multirow{3}{*}{	\textbf{\textit{	hakgi	}}}	&	\multirow{3}{*}{	O/C	}	&	\multirow{3}{*}{	semester, trimester, term 	}	&	\multirow{3}{*}{	Sinitic	}	&	\multirow{	3	}{*}{	\textit{		}		學期		}	\\&&&&				\textit{		}					\\&&&&	\textit{		}					\\\arrayrulecolor{gray} \hline
\multirow{3}{*}{	\textbf{\textit{	hakgip	}}}	&	\multirow{3}{*}{	O/C	}	&	\multirow{3}{*}{	class/grade (school)	}	&	\multirow{3}{*}{	Sinitic	}	&	\multirow{	3	}{*}{	\textit{		}		學級		}	\\&&&&				\textit{		}					\\&&&&	\textit{		}					\\\arrayrulecolor{gray} \hline
\multirow{3}{*}{	\textbf{\textit{	hakgyaw	}}}	&	\multirow{3}{*}{	O/C	}	&	\multirow{3}{*}{	school	}	&	\multirow{3}{*}{	Sinitic	}	&	\multirow{	3	}{*}{	\textit{		}		學校 		}	\\&&&&				\textit{		}					\\&&&&	\textit{		}					\\\arrayrulecolor{gray} \hline
\multirow{3}{*}{	\textbf{\textit{	hakseng	}}}	&	\multirow{3}{*}{	O/C	}	&	\multirow{3}{*}{	student	}	&	\multirow{3}{*}{	Sinitic	}	&	\multirow{	3	}{*}{	\textit{		}		學生		}	\\&&&&				\textit{		}					\\&&&&	\textit{		}					\\\arrayrulecolor{gray} \hline
\multirow{3}{*}{	\textbf{\textit{	haksip	}}}	&	\multirow{3}{*}{	Action	}	&	\multirow{3}{*}{	learn, study	}	&	\multirow{3}{*}{	Sinitic	}	&	\multirow{	3	}{*}{	\textit{		}		學習		}	\\&&&&				\textit{		}					\\&&&&	\textit{		}					\\\arrayrulecolor{gray} \hline
\multirow{3}{*}{	\textbf{\textit{	hali	}}}	&	\multirow{3}{*}{	O/C	}	&	\multirow{3}{*}{	sun / day	}	&	\multirow{3}{*}{	Austronesian	}	&	\multirow{	3	}{*}{	\textit{		}				}	\\&&&&				\textit{		}					\\&&&&	\textit{		}					\\\arrayrulecolor{gray} \hline
\multirow{3}{*}{	\textbf{\textit{	hali (ji) sikko	}}}	&	\multirow{3}{*}{	O/C	}	&	\multirow{3}{*}{	lunch	}	&	\multirow{3}{*}{	Compound	}	&	\multirow{	3	}{*}{	\textit{		}				}	\\&&&&				\textit{		}					\\&&&&	\textit{		}					\\\arrayrulecolor{gray} \hline
\multirow{3}{*}{	\textbf{\textit{	halimaw	}}}	&	\multirow{3}{*}{	O/C	}	&	\multirow{3}{*}{	tiger	}	&	\multirow{3}{*}{	Austronesian	}	&	\multirow{	3	}{*}{	\textit{	ms/id	 - }		harimau		}	\\&&&&				\textit{		}					\\&&&&	\textit{		}					\\\arrayrulecolor{gray} \hline
\multirow{3}{*}{	\textbf{\textit{	hang'at	}}}	&	\multirow{3}{*}{	Qual	}	&	\multirow{3}{*}{	warm, hot	}	&	\multirow{3}{*}{	Austronesian	}	&	\multirow{	3	}{*}{	\textit{	ms/id	 - }		hangat		}	\\&&&&				\textit{		}					\\&&&&	\textit{		}					\\\arrayrulecolor{gray} \hline
\multirow{3}{*}{	\textbf{\textit{	hangha	}}}	&	\multirow{3}{*}{	Action	}	&	\multirow{3}{*}{	to drop, fall	}	&	\multirow{3}{*}{	Sinitic	}	&	\multirow{	3	}{*}{	\textit{		}		降下		}	\\&&&&				\textit{		}					\\&&&&	\textit{		}					\\\arrayrulecolor{gray} \hline
\multirow{3}{*}{	\textbf{\textit{	hangok'o	}}}	&	\multirow{3}{*}{	O/C	}	&	\multirow{3}{*}{	korean (language)	}	&	\multirow{3}{*}{	Sinitic	}	&	\multirow{	3	}{*}{	\textit{		}		韓國語		}	\\&&&&				\textit{		}					\\&&&&	\textit{		}					\\\arrayrulecolor{gray} \hline
\multirow{3}{*}{	\textbf{\textit{	haon	}}}	&	\multirow{3}{*}{	O/C	}	&	\multirow{3}{*}{	spring	}	&	\multirow{3}{*}{	Koreo-Japonic	}	&	\multirow{	2	}{*}{	\textit{	ko	 - }		봄		}	\\&&&&	\multirow{	2	}{*}{	\textit{	ja	 - }		はる		}	\\&&&&	\textit{		}					\\\arrayrulecolor{gray} \hline
\multirow{3}{*}{	\textbf{\textit{	hatay	}}}	&	\multirow{3}{*}{	O/C	}	&	\multirow{3}{*}{	heart	}	&	\multirow{3}{*}{	Austronesian	}	&	\multirow{	3	}{*}{	\textit{		}				}	\\&&&&				\textit{		}					\\&&&&	\textit{		}					\\\arrayrulecolor{gray} \hline
\multirow{3}{*}{	\textbf{\textit{	haw	}}}	&	\multirow{3}{*}{	Action	}	&	\multirow{3}{*}{	like	}	&	\multirow{3}{*}{	Sinitic	}	&	\multirow{	3	}{*}{	\textit{		}		好		}	\\&&&&				\textit{		}					\\&&&&	\textit{		}					\\\arrayrulecolor{gray} \hline
\multirow{3}{*}{	\textbf{\textit{	haw	}}}	&	\multirow{3}{*}{	Qual	}	&	\multirow{3}{*}{	good; pleasing	}	&	\multirow{3}{*}{	Sinitic	}	&	\multirow{	3	}{*}{	\textit{		}		好		}	\\&&&&				\textit{		}					\\&&&&	\textit{		}					\\\arrayrulecolor{gray} \hline
\multirow{3}{*}{	\textbf{\textit{	hayang	}}}	&	\multirow{3}{*}{	Qual	}	&	\multirow{3}{*}{	fast, quick	}	&	\multirow{3}{*}{	Koreo-Japonic	}	&	\multirow{	2	}{*}{	\textit{	ko	 - }		허영		}	\\&&&&	\multirow{	2	}{*}{	\textit{	ja	 - }		はやく		}	\\&&&&	\textit{		}					\\\arrayrulecolor{gray} \hline
\multirow{3}{*}{	\textbf{\textit{	hayben	}}}	&	\multirow{3}{*}{	O/C	}	&	\multirow{3}{*}{	beach	}	&	\multirow{3}{*}{	Sinitic	}	&	\multirow{	3	}{*}{	\textit{		}		海邊		}	\\&&&&				\textit{		}					\\&&&&	\textit{		}					\\\arrayrulecolor{gray} \hline
\multirow{3}{*}{	\textbf{\textit{	heksik	}}}	&	\multirow{3}{*}{	Qual	}	&	\multirow{3}{*}{	black	}	&	\multirow{3}{*}{	Sinitic	}	&	\multirow{	3	}{*}{	\textit{		}		黑色		}	\\&&&&				\textit{		}					\\&&&&	\textit{		}					\\\arrayrulecolor{gray} \hline
\multirow{3}{*}{	\textbf{\textit{	hem	}}}	&	\multirow{3}{*}{	Action	}	&	\multirow{3}{*}{	hate	}	&	\multirow{3}{*}{	Sinitic	}	&	\multirow{	3	}{*}{	\textit{		}		嫌		}	\\&&&&				\textit{		}					\\&&&&	\textit{		}					\\\arrayrulecolor{gray} \hline
\multirow{3}{*}{	\textbf{\textit{	heng	}}}	&	\multirow{3}{*}{	Action	}	&	\multirow{3}{*}{	do; go	}	&	\multirow{3}{*}{	Sinitic	}	&	\multirow{	3	}{*}{	\textit{		}		行		}	\\&&&&				\textit{		}					\\&&&&	\textit{		}					\\\arrayrulecolor{gray} \hline
\multirow{3}{*}{	\textbf{\textit{	heng-sap	}}}	&	\multirow{3}{*}{	O/C	}	&	\multirow{3}{*}{	action, verb	}	&	\multirow{3}{*}{	Compound	}	&	\multirow{	3	}{*}{	\textit{		}		heng + sap		}	\\&&&&				\textit{		}					\\&&&&	\textit{		}					\\\arrayrulecolor{gray} \hline
\multirow{3}{*}{	\textbf{\textit{	hengdey	}}}	&	\multirow{3}{*}{	O/C	}	&	\multirow{3}{*}{	brothers, siblings in general	}	&	\multirow{3}{*}{	Sinitic	}	&	\multirow{	3	}{*}{	\textit{		}		兄弟		}	\\&&&&				\textit{		}					\\&&&&	\textit{		}					\\\arrayrulecolor{gray} \hline
\multirow{3}{*}{	\textbf{\textit{	hengheng	}}}	&	\multirow{3}{*}{	O/C	}	&	\multirow{3}{*}{	older brother	}	&	\multirow{3}{*}{	Sinitic	}	&	\multirow{	3	}{*}{	\textit{		}		兄兄		}	\\&&&&				\textit{		}					\\&&&&	\textit{		}					\\\arrayrulecolor{gray} \hline
\multirow{3}{*}{	\textbf{\textit{	heta'e	}}}	&	\multirow{3}{*}{	Qual	}	&	\multirow{3}{*}{	down, below, under	}	&	\multirow{3}{*}{	Koreo-Japonic	}	&	\multirow{	2	}{*}{	\textit{	ko	 - }		밑		}	\\&&&&	\multirow{	2	}{*}{	\textit{	ja	 - }		きた		}	\\&&&&	\textit{		}					\\\arrayrulecolor{gray} \hline
\multirow{3}{*}{	\textbf{\textit{	hidong	}}}	&	\multirow{3}{*}{	O/C	}	&	\multirow{3}{*}{	nose	}	&	\multirow{3}{*}{	Austronesian	}	&	\multirow{	2	}{*}{	\textit{	ms	 - }		idung		}	\\&&&&	\multirow{	2	}{*}{	\textit{	tl	 - }		hilong		}	\\&&&&	\textit{		}					\\\arrayrulecolor{gray} \hline
\multirow{3}{*}{	\textbf{\textit{	hima	}}}	&	\multirow{3}{*}{	O/C	}	&	\multirow{3}{*}{	snow	}	&	\multirow{3}{*}{	Sanskrit	}	&	\multirow{	2	}{*}{	\textit{		}	\textsanskrit{	हिम	}	}	\\&&&&	\multirow{	2	}{*}{	\textit{		}		(himá)		}	\\&&&&	\textit{		}					\\\arrayrulecolor{gray} \hline
\multirow{3}{*}{	\textbf{\textit{	himwang	}}}	&	\multirow{3}{*}{	Action	}	&	\multirow{3}{*}{	hope	}	&	\multirow{3}{*}{	Sinitic	}	&	\multirow{	3	}{*}{	\textit{		}		希望		}	\\&&&&				\textit{		}					\\&&&&	\textit{		}					\\\arrayrulecolor{gray} \hline
\multirow{3}{*}{	\textbf{\textit{	hing	}}}	&	\multirow{3}{*}{	Qual	}	&	\multirow{3}{*}{	white	}	&	\multirow{3}{*}{	Koreo-Japonic	}	&	\multirow{	2	}{*}{	\textit{	ko	 - }		하얀		}	\\&&&&	\multirow{	2	}{*}{	\textit{	ja	 - }		しろ		}	\\&&&&	\textit{		}					\\\arrayrulecolor{gray} \hline
\multirow{3}{*}{	\textbf{\textit{	hing bawang	}}}	&	\multirow{3}{*}{	O/C	}	&	\multirow{3}{*}{	onions	}	&	\multirow{3}{*}{	Compound	}	&	\multirow{	3	}{*}{	\textit{		}		hing + bawang		}	\\&&&&				\textit{		}					\\&&&&	\textit{		}					\\\arrayrulecolor{gray} \hline
\multirow{3}{*}{	\textbf{\textit{	ho	}}}	&	\multirow{3}{*}{	Action	}	&	\multirow{3}{*}{	call	}	&	\multirow{3}{*}{	Sinitic	}	&	\multirow{	3	}{*}{	\textit{		}		呼		}	\\&&&&				\textit{		}					\\&&&&	\textit{		}					\\\arrayrulecolor{gray} \hline
\multirow{3}{*}{	\textbf{\textit{	ho'jaw	}}}	&	\multirow{3}{*}{	O/C	}	&	\multirow{3}{*}{	pond; swamp	}	&	\multirow{3}{*}{	Sinitic	}	&	\multirow{	3	}{*}{	\textit{		}		湖沼		}	\\&&&&				\textit{		}					\\&&&&	\textit{		}					\\\arrayrulecolor{gray} \hline
\multirow{3}{*}{	\textbf{\textit{	hoen	}}}	&	\multirow{3}{*}{	Action	}	&	\multirow{3}{*}{	to (tell a) lie	}	&	\multirow{3}{*}{	Sinitic	}	&	\multirow{	3	}{*}{	\textit{		}		虛言		}	\\&&&&				\textit{		}					\\&&&&	\textit{		}					\\\arrayrulecolor{gray} \hline
\multirow{3}{*}{	\textbf{\textit{	hohip	}}}	&	\multirow{3}{*}{	Action	}	&	\multirow{3}{*}{	breathe, breath	}	&	\multirow{3}{*}{	Sinitic	}	&	\multirow{	3	}{*}{	\textit{		}		呼吸		}	\\&&&&				\textit{		}					\\&&&&	\textit{		}					\\\arrayrulecolor{gray} \hline
\multirow{3}{*}{	\textbf{\textit{	holi	}}}	&	\multirow{3}{*}{	O/C	}	&	\multirow{3}{*}{	ice	}	&	\multirow{3}{*}{	Koreo-Japonic	}	&	\multirow{	2	}{*}{	\textit{	ko	 - }		얼음		}	\\&&&&	\multirow{	2	}{*}{	\textit{	ja	 - }		こり		}	\\&&&&	\textit{		}					\\\arrayrulecolor{gray} \hline
\multirow{3}{*}{	\textbf{\textit{	holi ujan	}}}	&	\multirow{3}{*}{	O/C	}	&	\multirow{3}{*}{	hail	}	&	\multirow{3}{*}{	Compound	}	&	\multirow{	3	}{*}{	\textit{		}		holi + ujan		}	\\&&&&				\textit{		}					\\&&&&	\textit{		}					\\\arrayrulecolor{gray} \hline
\multirow{3}{*}{	\textbf{\textit{	hongsik	}}}	&	\multirow{3}{*}{	Qual	}	&	\multirow{3}{*}{	red	}	&	\multirow{3}{*}{	Sinitic	}	&	\multirow{	3	}{*}{	\textit{		}		紅色		}	\\&&&&				\textit{		}					\\&&&&	\textit{		}					\\\arrayrulecolor{gray} \hline
\multirow{3}{*}{	\textbf{\textit{	honi	}}}	&	\multirow{3}{*}{	O/C	}	&	\multirow{3}{*}{	sheep; goat	}	&	\multirow{3}{*}{	Altaic	}	&	\multirow{	3	}{*}{	\textit{	mn	 - }		хонь		}	\\&&&&				\textit{		}					\\&&&&	\textit{		}					\\\arrayrulecolor{gray} \hline
\multirow{3}{*}{	\textbf{\textit{	hoytak	}}}	&	\multirow{3}{*}{	O/C	}	&	\multirow{3}{*}{	snail/slug	}	&	\multirow{3}{*}{	Austroasiatic	}	&	\multirow{	2	}{*}{	\textit{	th	 - }	\textthai{	หอยทาก	}	}	\\&&&&	\multirow{	2	}{*}{	\textit{	lo	 - }	\textlao{	ຫອຍ+ທາກ	}	}	\\&&&&	\textit{		}					\\\arrayrulecolor{gray} \hline
\multirow{3}{*}{	\textbf{\textit{	hoyul	}}}	&	\multirow{3}{*}{	O/C	}	&	\multirow{3}{*}{	winter	}	&	\multirow{3}{*}{	Koreo-Japonic	}	&	\multirow{	2	}{*}{	\textit{	ko	 - }		여울		}	\\&&&&	\multirow{	2	}{*}{	\textit{	ja	 - }		ふゆ		}	\\&&&&	\textit{		}					\\\arrayrulecolor{gray} \hline
\multirow{3}{*}{	\textbf{\textit{	hujan	}}}	&	\multirow{3}{*}{	O/C	}	&	\multirow{3}{*}{	rain	}	&	\multirow{3}{*}{	Austronesian	}	&	\multirow{	3	}{*}{	\textit{	ms/id	 - }		hujan		}	\\&&&&				\textit{		}					\\&&&&	\textit{		}					\\\arrayrulecolor{gray} \hline
\multirow{3}{*}{	\textbf{\textit{	hutan	}}}	&	\multirow{3}{*}{	O/C	}	&	\multirow{3}{*}{	forest	}	&	\multirow{3}{*}{	Austronesian	}	&	\multirow{	3	}{*}{	\textit{	ms/id	 - }		hutan		}	\\&&&&				\textit{		}					\\&&&&	\textit{		}					\\\arrayrulecolor{gray} \hline
\multirow{3}{*}{	\textbf{\textit{	hwan'eng	}}}	&	\multirow{3}{*}{	Action	}	&	\multirow{3}{*}{	welcome, reception	}	&	\multirow{3}{*}{	Sinitic	}	&	\multirow{	3	}{*}{	\textit{		}		歡迎 		}	\\&&&&				\textit{		}					\\&&&&	\textit{		}					\\\arrayrulecolor{gray} \hline
\multirow{3}{*}{	\textbf{\textit{	hwangsik	}}}	&	\multirow{3}{*}{	Qual	}	&	\multirow{3}{*}{	yellow	}	&	\multirow{3}{*}{	Sinitic	}	&	\multirow{	3	}{*}{	\textit{		}		黃色		}	\\&&&&				\textit{		}					\\&&&&	\textit{		}					\\\arrayrulecolor{gray} \hline
\multirow{3}{*}{	\textbf{\textit{	hwaybi	}}}	&	\multirow{3}{*}{	Action	}	&	\multirow{3}{*}{	hide, avoid	}	&	\multirow{3}{*}{	Sinitic	}	&	\multirow{	3	}{*}{	\textit{		}		回避		}	\\&&&&				\textit{		}					\\&&&&	\textit{		}					\\\arrayrulecolor{gray} \hline
\multirow{3}{*}{	\textbf{\textit{	hwaydap	}}}	&	\multirow{3}{*}{	Action	}	&	\multirow{3}{*}{	answer	}	&	\multirow{3}{*}{	Sinitic	}	&	\multirow{	3	}{*}{	\textit{		}		回答		}	\\&&&&				\textit{		}					\\&&&&	\textit{		}					\\\arrayrulecolor{gray} \hline
\multirow{3}{*}{	\textbf{\textit{	hwaylay	}}}	&	\multirow{3}{*}{	Action	}	&	\multirow{3}{*}{	to return	}	&	\multirow{3}{*}{	Sinitic	}	&	\multirow{	3	}{*}{	\textit{		}		回來		}	\\&&&&				\textit{		}					\\&&&&	\textit{		}					\\\arrayrulecolor{gray} \hline
\multirow{3}{*}{	\textbf{\textit{	hwaymen	}}}	&	\multirow{3}{*}{	O/C	}	&	\multirow{3}{*}{	screen	}	&	\multirow{3}{*}{	Sinitic	}	&	\multirow{	3	}{*}{	\textit{		}		畵面		}	\\&&&&				\textit{		}					\\&&&&	\textit{		}					\\\arrayrulecolor{gray} \hline
\multirow{3}{*}{	\textbf{\textit{	hwetgwan	}}}	&	\multirow{3}{*}{	O/C	}	&	\multirow{3}{*}{	blood vessels	}	&	\multirow{3}{*}{	Sinitic	}	&	\multirow{	3	}{*}{	\textit{		}		血管		}	\\&&&&				\textit{		}					\\&&&&	\textit{		}					\\\arrayrulecolor{gray} \hline
\multirow{3}{*}{	\textbf{\textit{	hyowsik	}}}	&	\multirow{3}{*}{	Action	}	&	\multirow{3}{*}{	rest	}	&	\multirow{3}{*}{	Sinitic	}	&	\multirow{	3	}{*}{	\textit{		}		休息		}	\\&&&&				\textit{		}					\\&&&&	\textit{		}					\\\arrayrulecolor{gray} \hline
\multirow{3}{*}{	\textbf{\textit{	i'buk	}}}	&	\multirow{3}{*}{	O/C	}	&	\multirow{3}{*}{	clothes, clothing	}	&	\multirow{3}{*}{	Sinitic	}	&	\multirow{	3	}{*}{	\textit{		}		衣服		}	\\&&&&				\textit{		}					\\&&&&	\textit{		}					\\\arrayrulecolor{gray} \hline
\multirow{3}{*}{	\textbf{\textit{	iba	}}}	&	\multirow{3}{*}{	O/C	}	&	\multirow{3}{*}{	uncle	}	&	\multirow{3}{*}{	Sinitic	}	&	\multirow{	3	}{*}{	\textit{		}		異父		}	\\&&&&				\textit{		}					\\&&&&	\textit{		}					\\\arrayrulecolor{gray} \hline
\multirow{3}{*}{	\textbf{\textit{	ibilisi	}}}	&	\multirow{3}{*}{	O/C	}	&	\multirow{3}{*}{	demon; being of evil	}	&	\multirow{3}{*}{	Arabic	}	&	\multirow{	3	}{*}{	\textit{	ar	 - }		ʾiblīs 		}	\\&&&&				\textit{		}					\\&&&&	\textit{		}					\\\arrayrulecolor{gray} \hline
\multirow{3}{*}{	\textbf{\textit{	idong	}}}	&	\multirow{3}{*}{	Action	}	&	\multirow{3}{*}{	move	}	&	\multirow{3}{*}{	Sinitic	}	&	\multirow{	3	}{*}{	\textit{		}		移動		}	\\&&&&				\textit{		}					\\&&&&	\textit{		}					\\\arrayrulecolor{gray} \hline
\multirow{3}{*}{	\textbf{\textit{	ifa	}}}	&	\multirow{3}{*}{	O/C	}	&	\multirow{3}{*}{	leaf	}	&	\multirow{3}{*}{	Koreo-Japonic	}	&	\multirow{	2	}{*}{	\textit{	ko	 - }		잎		}	\\&&&&	\multirow{	2	}{*}{	\textit{	ja	 - }		は		}	\\&&&&	\textit{		}					\\\arrayrulecolor{gray} \hline
\multirow{3}{*}{	\textbf{\textit{	ihengdey	}}}	&	\multirow{3}{*}{	O/C	}	&	\multirow{3}{*}{	cousins	}	&	\multirow{3}{*}{	Sinitic	}	&	\multirow{	3	}{*}{	\textit{		}		異兄弟		}	\\&&&&				\textit{		}					\\&&&&	\textit{		}					\\\arrayrulecolor{gray} \hline
\multirow{3}{*}{	\textbf{\textit{	ikan	}}}	&	\multirow{3}{*}{	O/C	}	&	\multirow{3}{*}{	fish	}	&	\multirow{3}{*}{	Austronesian	}	&	\multirow{	3	}{*}{	\textit{	ms/id	 - }		ikan		}	\\&&&&				\textit{		}					\\&&&&	\textit{		}					\\\arrayrulecolor{gray} \hline
\multirow{3}{*}{	\textbf{\textit{	iknit	}}}	&	\multirow{3}{*}{	O/C	}	&	\multirow{3}{*}{	tomorrow	}	&	\multirow{3}{*}{	Sinitic	}	&	\multirow{	3	}{*}{	\textit{		}		翌日		}	\\&&&&				\textit{		}					\\&&&&	\textit{		}					\\\arrayrulecolor{gray} \hline
\multirow{3}{*}{	\textbf{\textit{	ima	}}}	&	\multirow{3}{*}{	O/C	}	&	\multirow{3}{*}{	aunt	}	&	\multirow{3}{*}{	Sinitic	}	&	\multirow{	3	}{*}{	\textit{		}		異母		}	\\&&&&				\textit{		}					\\&&&&	\textit{		}					\\\arrayrulecolor{gray} \hline
\multirow{3}{*}{	\textbf{\textit{	imak	}}}	&	\multirow{3}{*}{	O/C	}	&	\multirow{3}{*}{	music	}	&	\multirow{3}{*}{	Sinitic	}	&	\multirow{	3	}{*}{	\textit{		}		音樂		}	\\&&&&				\textit{		}					\\&&&&	\textit{		}					\\\arrayrulecolor{gray} \hline
\multirow{3}{*}{	\textbf{\textit{	imlaw	}}}	&	\multirow{3}{*}{	O/C	}	&	\multirow{3}{*}{	drink, beverage	}	&	\multirow{3}{*}{	Sinitic	}	&	\multirow{	3	}{*}{	\textit{		}		飲料		}	\\&&&&				\textit{		}					\\&&&&	\textit{		}					\\\arrayrulecolor{gray} \hline
\multirow{3}{*}{	\textbf{\textit{	imlyaw	}}}	&	\multirow{3}{*}{	O/C	}	&	\multirow{3}{*}{	drink, beverage	}	&	\multirow{3}{*}{	Sinitic	}	&	\multirow{	3	}{*}{	\textit{		}		飲料		}	\\&&&&				\textit{		}					\\&&&&	\textit{		}					\\\arrayrulecolor{gray} \hline
\multirow{3}{*}{	\textbf{\textit{	indonesyao	}}}	&	\multirow{3}{*}{	O/C	}	&	\multirow{3}{*}{	indonesian (language)	}	&	\multirow{3}{*}{	Compound	}	&	\multirow{	3	}{*}{	\textit{		}				}	\\&&&&				\textit{		}					\\&&&&	\textit{		}					\\\arrayrulecolor{gray} \hline
\multirow{3}{*}{	\textbf{\textit{	inglisio	}}}	&	\multirow{3}{*}{	O/C	}	&	\multirow{3}{*}{	english (language)	}	&	\multirow{3}{*}{	Compound	}	&	\multirow{	3	}{*}{	\textit{		}				}	\\&&&&				\textit{		}					\\&&&&	\textit{		}					\\\arrayrulecolor{gray} \hline
\multirow{3}{*}{	\textbf{\textit{	intanet	}}}	&	\multirow{3}{*}{	O/C	}	&	\multirow{3}{*}{	internet	}	&	\multirow{3}{*}{	Western: English	}	&	\multirow{	3	}{*}{	\textit{	en	 - }		internet		}	\\&&&&				\textit{		}					\\&&&&	\textit{		}					\\\arrayrulecolor{gray} \hline
\multirow{3}{*}{	\textbf{\textit{	it	}}}	&	\multirow{3}{*}{	O/C	}	&	\multirow{3}{*}{	one	}	&	\multirow{3}{*}{	Sinitic	}	&	\multirow{	3	}{*}{	\textit{		}		一		}	\\&&&&				\textit{		}					\\&&&&	\textit{		}					\\\arrayrulecolor{gray} \hline
\multirow{3}{*}{	\textbf{\textit{	jacik	}}}	&	\multirow{3}{*}{	Qual	}	&	\multirow{3}{*}{	left	}	&	\multirow{3}{*}{	Sinitic	}	&	\multirow{	3	}{*}{	\textit{		}		左		}	\\&&&&				\textit{		}					\\&&&&	\textit{		}					\\\arrayrulecolor{gray} \hline
\multirow{3}{*}{	\textbf{\textit{	jaket	}}}	&	\multirow{3}{*}{	O/C	}	&	\multirow{3}{*}{	jacket	}	&	\multirow{3}{*}{	Western: English	}	&	\multirow{	3	}{*}{	\textit{	en	 - }		jacket		}	\\&&&&				\textit{		}					\\&&&&	\textit{		}					\\\arrayrulecolor{gray} \hline
\multirow{3}{*}{	\textbf{\textit{	jaknit	}}}	&	\multirow{3}{*}{	O/C	}	&	\multirow{3}{*}{	yesterday	}	&	\multirow{3}{*}{	Sinitic	}	&	\multirow{	3	}{*}{	\textit{		}		昨日		}	\\&&&&				\textit{		}					\\&&&&	\textit{		}					\\\arrayrulecolor{gray} \hline
\multirow{3}{*}{	\textbf{\textit{	jangdan	}}}	&	\multirow{3}{*}{	O/C	}	&	\multirow{3}{*}{	length; height	}	&	\multirow{3}{*}{	Sinitic	}	&	\multirow{	3	}{*}{	\textit{		}		長短		}	\\&&&&				\textit{		}					\\&&&&	\textit{		}					\\\arrayrulecolor{gray} \hline
\multirow{3}{*}{	\textbf{\textit{	janggut	}}}	&	\multirow{3}{*}{	O/C	}	&	\multirow{3}{*}{	beard	}	&	\multirow{3}{*}{	Austronesian	}	&	\multirow{	2	}{*}{	\textit{	ms	 - }		jenggot		}	\\&&&&	\multirow{	2	}{*}{	\textit{	id	 - }		janggut		}	\\&&&&	\textit{		}					\\\arrayrulecolor{gray} \hline
\multirow{3}{*}{	\textbf{\textit{	jangki	}}}	&	\multirow{3}{*}{	O/C	}	&	\multirow{3}{*}{	organs, intestines	}	&	\multirow{3}{*}{	Sinitic	}	&	\multirow{	3	}{*}{	\textit{		}		臟器		}	\\&&&&				\textit{		}					\\&&&&	\textit{		}					\\\arrayrulecolor{gray} \hline
\multirow{3}{*}{	\textbf{\textit{	jenghwangsik	}}}	&	\multirow{3}{*}{	Qual	}	&	\multirow{3}{*}{	orange (color)	}	&	\multirow{3}{*}{	Sinitic	}	&	\multirow{	3	}{*}{	\textit{		}		橙黃色		}	\\&&&&				\textit{		}					\\&&&&	\textit{		}					\\\arrayrulecolor{gray} \hline
\multirow{3}{*}{	\textbf{\textit{	jidongca	}}}	&	\multirow{3}{*}{	O/C	}	&	\multirow{3}{*}{	car, automobile	}	&	\multirow{3}{*}{	Sinitic	}	&	\multirow{	3	}{*}{	\textit{		}		自動車		}	\\&&&&				\textit{		}					\\&&&&	\textit{		}					\\\arrayrulecolor{gray} \hline
\multirow{3}{*}{	\textbf{\textit{	jihongsik	}}}	&	\multirow{3}{*}{	Qual	}	&	\multirow{3}{*}{	purple	}	&	\multirow{3}{*}{	Sinitic	}	&	\multirow{	3	}{*}{	\textit{		}		紫紅		}	\\&&&&				\textit{		}					\\&&&&	\textit{		}					\\\arrayrulecolor{gray} \hline
\multirow{3}{*}{	\textbf{\textit{	jiji	}}}	&	\multirow{3}{*}{	O/C	}	&	\multirow{3}{*}{	older sister	}	&	\multirow{3}{*}{	Sinitic	}	&	\multirow{	3	}{*}{	\textit{		}		姉姉		}	\\&&&&				\textit{		}					\\&&&&	\textit{		}					\\\arrayrulecolor{gray} \hline
\multirow{3}{*}{	\textbf{\textit{	jikep	}}}	&	\multirow{3}{*}{	Action	}	&	\multirow{3}{*}{	work, job, career	}	&	\multirow{3}{*}{	Sinitic	}	&	\multirow{	3	}{*}{	\textit{		}		職業		}	\\&&&&				\textit{		}					\\&&&&	\textit{		}					\\\arrayrulecolor{gray} \hline
\multirow{3}{*}{	\textbf{\textit{	jikjep	}}}	&	\multirow{3}{*}{	Qual	}	&	\multirow{3}{*}{	directly	}	&	\multirow{3}{*}{	Sinitic	}	&	\multirow{	3	}{*}{	\textit{		}		直接		}	\\&&&&				\textit{		}					\\&&&&	\textit{		}					\\\arrayrulecolor{gray} \hline
\multirow{3}{*}{	\textbf{\textit{	jiksen	}}}	&	\multirow{3}{*}{	Qual	}	&	\multirow{3}{*}{	straight	}	&	\multirow{3}{*}{	Sinitic	}	&	\multirow{	3	}{*}{	\textit{		}		直線		}	\\&&&&				\textit{		}					\\&&&&	\textit{		}					\\\arrayrulecolor{gray} \hline
\multirow{3}{*}{	\textbf{\textit{	jikwi	}}}	&	\multirow{3}{*}{	Action	}	&	\multirow{3}{*}{	ascend (to the throne)	}	&	\multirow{3}{*}{	Sinitic	}	&	\multirow{	3	}{*}{	\textit{		}		卽位		}	\\&&&&				\textit{		}					\\&&&&	\textit{		}					\\\arrayrulecolor{gray} \hline
\multirow{3}{*}{	\textbf{\textit{	jilyaw	}}}	&	\multirow{3}{*}{	Action	}	&	\multirow{3}{*}{	heal, fix	}	&	\multirow{3}{*}{	Sinitic	}	&	\multirow{	3	}{*}{	\textit{		}		治療		}	\\&&&&				\textit{		}					\\&&&&	\textit{		}					\\\arrayrulecolor{gray} \hline
\multirow{3}{*}{	\textbf{\textit{	jimay	}}}	&	\multirow{3}{*}{	O/C	}	&	\multirow{3}{*}{	sisters	}	&	\multirow{3}{*}{	Sinitic	}	&	\multirow{	3	}{*}{	\textit{		}		姉妹		}	\\&&&&				\textit{		}					\\&&&&	\textit{		}					\\\arrayrulecolor{gray} \hline
\multirow{3}{*}{	\textbf{\textit{	jimun	}}}	&	\multirow{3}{*}{	Action	}	&	\multirow{3}{*}{	ask	}	&	\multirow{3}{*}{	Sinitic	}	&	\multirow{	3	}{*}{	\textit{		}		質問		}	\\&&&&				\textit{		}					\\&&&&	\textit{		}					\\\arrayrulecolor{gray} \hline
\multirow{3}{*}{	\textbf{\textit{	jimut	}}}	&	\multirow{3}{*}{	O/C	}	&	\multirow{3}{*}{	cloth	}	&	\multirow{3}{*}{	Sinitic	}	&	\multirow{	3	}{*}{	\textit{		}		織物		}	\\&&&&				\textit{		}					\\&&&&	\textit{		}					\\\arrayrulecolor{gray} \hline
\multirow{3}{*}{	\textbf{\textit{	jinen	}}}	&	\multirow{3}{*}{	Qual	}	&	\multirow{3}{*}{	nature, natural	}	&	\multirow{3}{*}{	Sinitic	}	&	\multirow{	3	}{*}{	\textit{		}		自然		}	\\&&&&				\textit{		}					\\&&&&	\textit{		}					\\\arrayrulecolor{gray} \hline
\multirow{3}{*}{	\textbf{\textit{	jingki	}}}	&	\multirow{3}{*}{	O/C	}	&	\multirow{3}{*}{	steam	}	&	\multirow{3}{*}{	Sinitic	}	&	\multirow{	3	}{*}{	\textit{		}		蒸氣		}	\\&&&&				\textit{		}					\\&&&&	\textit{		}					\\\arrayrulecolor{gray} \hline
\multirow{3}{*}{	\textbf{\textit{	jinsit	}}}	&	\multirow{3}{*}{	Qual	}	&	\multirow{3}{*}{	really, very, true, actual	}	&	\multirow{3}{*}{	Sinitic	}	&	\multirow{	3	}{*}{	\textit{		}		眞實		}	\\&&&&				\textit{		}					\\&&&&	\textit{		}					\\\arrayrulecolor{gray} \hline
\multirow{3}{*}{	\textbf{\textit{	jisin	}}}	&	\multirow{3}{*}{	O/C	}	&	\multirow{3}{*}{	self	}	&	\multirow{3}{*}{	Sinitic	}	&	\multirow{	3	}{*}{	\textit{		}		自身		}	\\&&&&				\textit{		}					\\&&&&	\textit{		}					\\\arrayrulecolor{gray} \hline
\multirow{3}{*}{	\textbf{\textit{	jitsik	}}}	&	\multirow{3}{*}{	Action	}	&	\multirow{3}{*}{	choke (on something, e.g. food), suffocate; asphyxia	}	&	\multirow{3}{*}{	Sinitic	}	&	\multirow{	3	}{*}{	\textit{		}		窒息		}	\\&&&&				\textit{		}					\\&&&&	\textit{		}					\\\arrayrulecolor{gray} \hline
\multirow{3}{*}{	\textbf{\textit{	jiyu	}}}	&	\multirow{3}{*}{	O/C	}	&	\multirow{3}{*}{	freedom	}	&	\multirow{3}{*}{		}	&	\multirow{	3	}{*}{	\textit{		}				}	\\&&&&				\textit{		}					\\&&&&	\textit{		}					\\\arrayrulecolor{gray} \hline
\multirow{3}{*}{	\textbf{\textit{	joji	}}}	&	\multirow{3}{*}{	O/C	}	&	\multirow{3}{*}{	chopstick	}	&	\multirow{3}{*}{	Sinitic	}	&	\multirow{	3	}{*}{	\textit{		}		箸子		}	\\&&&&				\textit{		}					\\&&&&	\textit{		}					\\\arrayrulecolor{gray} \hline
\multirow{3}{*}{	\textbf{\textit{	jong	}}}	&	\multirow{3}{*}{	O/C	}	&	\multirow{3}{*}{	floors (of a building)	}	&	\multirow{3}{*}{	Sinitic	}	&	\multirow{	3	}{*}{	\textit{		}		層		}	\\&&&&				\textit{		}					\\&&&&	\textit{		}					\\\arrayrulecolor{gray} \hline
\multirow{3}{*}{	\textbf{\textit{	jonggyaw	}}}	&	\multirow{3}{*}{	O/C	}	&	\multirow{3}{*}{	religion	}	&	\multirow{3}{*}{	Sinitic	}	&	\multirow{	3	}{*}{	\textit{		}		宗敎		}	\\&&&&				\textit{		}					\\&&&&	\textit{		}					\\\arrayrulecolor{gray} \hline
\multirow{3}{*}{	\textbf{\textit{	jonjay	}}}	&	\multirow{3}{*}{	Action	}	&	\multirow{3}{*}{	exist	}	&	\multirow{3}{*}{	Sinitic	}	&	\multirow{	3	}{*}{	\textit{		}		存在		}	\\&&&&				\textit{		}					\\&&&&	\textit{		}					\\\arrayrulecolor{gray} \hline
\multirow{3}{*}{	\textbf{\textit{	jow	}}}	&	\multirow{3}{*}{	O/C	}	&	\multirow{3}{*}{	week	}	&	\multirow{3}{*}{	Sinitic	}	&	\multirow{	3	}{*}{	\textit{		}		週		}	\\&&&&				\textit{		}					\\&&&&	\textit{		}					\\\arrayrulecolor{gray} \hline
\multirow{3}{*}{	\textbf{\textit{	jow'jung	}}}	&	\multirow{3}{*}{	O/C	}	&	\multirow{3}{*}{	weekday	}	&	\multirow{3}{*}{	Sinitic	}	&	\multirow{	3	}{*}{	\textit{		}		週中		}	\\&&&&				\textit{		}					\\&&&&	\textit{		}					\\\arrayrulecolor{gray} \hline
\multirow{3}{*}{	\textbf{\textit{	jow'mat	}}}	&	\multirow{3}{*}{	O/C	}	&	\multirow{3}{*}{	weekend	}	&	\multirow{3}{*}{	Sinitic	}	&	\multirow{	3	}{*}{	\textit{		}		週末		}	\\&&&&				\textit{		}					\\&&&&	\textit{		}					\\\arrayrulecolor{gray} \hline
\multirow{3}{*}{	\textbf{\textit{	junbi	}}}	&	\multirow{3}{*}{	Action	}	&	\multirow{3}{*}{	prepare, get ready	}	&	\multirow{3}{*}{	Sinitic	}	&	\multirow{	3	}{*}{	\textit{		}		準備		}	\\&&&&				\textit{		}					\\&&&&	\textit{		}					\\\arrayrulecolor{gray} \hline
\multirow{3}{*}{	\textbf{\textit{	jung	}}}	&	\multirow{3}{*}{	Qual	}	&	\multirow{3}{*}{	middle	}	&	\multirow{3}{*}{	Sinitic	}	&	\multirow{	3	}{*}{	\textit{		}		中		}	\\&&&&				\textit{		}					\\&&&&	\textit{		}					\\\arrayrulecolor{gray} \hline
\multirow{3}{*}{	\textbf{\textit{	junggok'o	}}}	&	\multirow{3}{*}{	O/C	}	&	\multirow{3}{*}{	chinese (language)	}	&	\multirow{3}{*}{	Sinitic	}	&	\multirow{	3	}{*}{	\textit{		}		中國語		}	\\&&&&				\textit{		}					\\&&&&	\textit{		}					\\\arrayrulecolor{gray} \hline
\multirow{3}{*}{	\textbf{\textit{	jungji	}}}	&	\multirow{3}{*}{	O/C	}	&	\multirow{3}{*}{	seed	}	&	\multirow{3}{*}{	Sinitic	}	&	\multirow{	3	}{*}{	\textit{		}		種子		}	\\&&&&				\textit{		}					\\&&&&	\textit{		}					\\\arrayrulecolor{gray} \hline
\multirow{3}{*}{	\textbf{\textit{	jwa	}}}	&	\multirow{3}{*}{	Action	}	&	\multirow{3}{*}{	sit	}	&	\multirow{3}{*}{	Sinitic	}	&	\multirow{	3	}{*}{	\textit{		}		坐		}	\\&&&&				\textit{		}					\\&&&&	\textit{		}					\\\arrayrulecolor{gray} \hline
\multirow{3}{*}{	\textbf{\textit{	jway'ay	}}}	&	\multirow{3}{*}{	Qual	}	&	\multirow{3}{*}{	favorite	}	&	\multirow{3}{*}{	Sinitic	}	&	\multirow{	3	}{*}{	\textit{		}		最愛		}	\\&&&&				\textit{		}					\\&&&&	\textit{		}					\\\arrayrulecolor{gray} \hline
\multirow{3}{*}{	\textbf{\textit{	jwaygun	}}}	&	\multirow{3}{*}{	O/C	}	&	\multirow{3}{*}{	recently, lately	}	&	\multirow{3}{*}{	Sinitic	}	&	\multirow{	3	}{*}{	\textit{		}		最近		}	\\&&&&				\textit{		}					\\&&&&	\textit{		}					\\\arrayrulecolor{gray} \hline
\multirow{3}{*}{	\textbf{\textit{	jwayhow	}}}	&	\multirow{3}{*}{	O/C	}	&	\multirow{3}{*}{	last, final, conclusion, end	}	&	\multirow{3}{*}{	Sinitic	}	&	\multirow{	3	}{*}{	\textit{		}		最後		}	\\&&&&				\textit{		}					\\&&&&	\textit{		}					\\\arrayrulecolor{gray} \hline
\multirow{3}{*}{	\textbf{\textit{	kacang	}}}	&	\multirow{3}{*}{	O/C	}	&	\multirow{3}{*}{	nut, peanut	}	&	\multirow{3}{*}{	Austronesian	}	&	\multirow{	3	}{*}{	\textit{	ms/id	 - }		kacang		}	\\&&&&				\textit{		}					\\&&&&	\textit{		}					\\\arrayrulecolor{gray} \hline
\multirow{3}{*}{	\textbf{\textit{	kaday	}}}	&	\multirow{3}{*}{	O/C	}	&	\multirow{3}{*}{	store, shop	}	&	\multirow{3}{*}{	Austronesian	}	&	\multirow{	3	}{*}{	\textit{	ms/in	 - }		kedai		}	\\&&&&				\textit{		}					\\&&&&	\textit{		}					\\\arrayrulecolor{gray} \hline
\multirow{3}{*}{	\textbf{\textit{	kaknin	}}}	&	\multirow{3}{*}{	Action	}	&	\multirow{3}{*}{	check (something)	}	&	\multirow{3}{*}{	Sinitic	}	&	\multirow{	3	}{*}{	\textit{		}		確認		}	\\&&&&				\textit{		}					\\&&&&	\textit{		}					\\\arrayrulecolor{gray} \hline
\multirow{3}{*}{	\textbf{\textit{	kala bawang	}}}	&	\multirow{3}{*}{	O/C	}	&	\multirow{3}{*}{	garlic	}	&	\multirow{3}{*}{	Compound	}	&	\multirow{	3	}{*}{	\textit{		}				}	\\&&&&				\textit{		}					\\&&&&	\textit{		}					\\\arrayrulecolor{gray} \hline
\multirow{3}{*}{	\textbf{\textit{	kalakala	}}}	&	\multirow{3}{*}{	Qual	}	&	\multirow{3}{*}{	spicy	}	&	\multirow{3}{*}{	Koreo-Japonic	}	&	\multirow{	2	}{*}{	\textit{	ko	 - }		칼칼		}	\\&&&&	\multirow{	2	}{*}{	\textit{	ja	 - }		から		}	\\&&&&	\textit{		}					\\\arrayrulecolor{gray} \hline
\multirow{3}{*}{	\textbf{\textit{	kalapipal	}}}	&	\multirow{3}{*}{	O/C	}	&	\multirow{3}{*}{	chile pepper	}	&	\multirow{3}{*}{	Compound	}	&	\multirow{	3	}{*}{	\textit{		}				}	\\&&&&				\textit{		}					\\&&&&	\textit{		}					\\\arrayrulecolor{gray} \hline
\multirow{3}{*}{	\textbf{\textit{	kama	}}}	&	\multirow{3}{*}{	O/C	}	&	\multirow{3}{*}{	pot	}	&	\multirow{3}{*}{	Koreo-Japonic	}	&	\multirow{	2	}{*}{	\textit{	ko	 - }		가마		}	\\&&&&	\multirow{	2	}{*}{	\textit{	ja	 - }		かま		}	\\&&&&	\textit{		}					\\\arrayrulecolor{gray} \hline
\multirow{3}{*}{	\textbf{\textit{	kambing	}}}	&	\multirow{3}{*}{	O/C	}	&	\multirow{3}{*}{	goat	}	&	\multirow{3}{*}{	Austronesian	}	&	\multirow{	2	}{*}{	\textit{	ms/id	 - }		kambing		}	\\&&&&	\multirow{	2	}{*}{	\textit{	tl	 - }		kambing		}	\\&&&&	\textit{		}					\\\arrayrulecolor{gray} \hline
\multirow{3}{*}{	\textbf{\textit{	kamela	}}}	&	\multirow{3}{*}{	O/C	}	&	\multirow{3}{*}{	camera	}	&	\multirow{3}{*}{	Western: English	}	&	\multirow{	3	}{*}{	\textit{	en	 - }		camera		}	\\&&&&				\textit{		}					\\&&&&	\textit{		}					\\\arrayrulecolor{gray} \hline
\multirow{3}{*}{	\textbf{\textit{	kamisa	}}}	&	\multirow{3}{*}{	O/C	}	&	\multirow{3}{*}{	shirt, top	}	&	\multirow{3}{*}{	Western: Spanish	}	&	\multirow{	2	}{*}{	\textit{	es	 - }		camiseta		}	\\&&&&	\multirow{	2	}{*}{	\textit{		}		(indirect loan via tl)		}	\\&&&&	\textit{		}					\\\arrayrulecolor{gray} \hline
\multirow{3}{*}{	\textbf{\textit{	kampeng	}}}	&	\multirow{3}{*}{	O/C	}	&	\multirow{3}{*}{	wall	}	&	\multirow{3}{*}{	Austroasiatic	}	&	\multirow{	2	}{*}{	\textit{	th	 - }	\textthai{	กำแพง	}	}	\\&&&&	\multirow{	2	}{*}{	\textit{	km	 - }	\textkhmer{	កំពែង	}	}	\\&&&&	\textit{		}					\\\arrayrulecolor{gray} \hline
\multirow{3}{*}{	\textbf{\textit{	kamwi	}}}	&	\multirow{3}{*}{	O/C	}	&	\multirow{3}{*}{	god (polytheism); being of good	}	&	\multirow{3}{*}{	Koreo-Japonic	}	&	\multirow{	2	}{*}{	\textit{	ko	 - }		검 (archaic)		}	\\&&&&	\multirow{	2	}{*}{	\textit{	jp	 - }		かみ		}	\\&&&&	\textit{		}					\\\arrayrulecolor{gray} \hline
\multirow{3}{*}{	\textbf{\textit{	kana	}}}	&	\multirow{3}{*}{	O/C	}	&	\multirow{3}{*}{	Japanese kana, syllabary	}	&	\multirow{3}{*}{	Koreo-Japonic	}	&	\multirow{	3	}{*}{	\textit{	ja	 - }		かな		}	\\&&&&				\textit{		}					\\&&&&	\textit{		}					\\\arrayrulecolor{gray} \hline
\multirow{3}{*}{	\textbf{\textit{	kanbeng	}}}	&	\multirow{3}{*}{	Action	}	&	\multirow{3}{*}{	take care of, nurse	}	&	\multirow{3}{*}{	Sinitic	}	&	\multirow{	3	}{*}{	\textit{		}		看病		}	\\&&&&				\textit{		}					\\&&&&	\textit{		}					\\\arrayrulecolor{gray} \hline
\multirow{3}{*}{	\textbf{\textit{	kang	}}}	&	\multirow{3}{*}{	Qual	}	&	\multirow{3}{*}{	beside	}	&	\multirow{3}{*}{	Austroasiatic	}	&	\multirow{	2	}{*}{	\textit{	th	 - }	\textthai{	ข้าง	}	}	\\&&&&	\multirow{	2	}{*}{	\textit{	lo	 - }	\textlao{	ຂ້າງ	}	}	\\&&&&	\textit{		}					\\\arrayrulecolor{gray} \hline
\multirow{3}{*}{	\textbf{\textit{	katak	}}}	&	\multirow{3}{*}{	O/C	}	&	\multirow{3}{*}{	frog/toad	}	&	\multirow{3}{*}{	Austronesian	}	&	\multirow{	2	}{*}{	\textit{	ms/id	 - }		katak		}	\\&&&&	\multirow{	2	}{*}{	\textit{	tl	 - }		palaka		}	\\&&&&	\textit{		}					\\\arrayrulecolor{gray} \hline
\multirow{3}{*}{	\textbf{\textit{	kaw'i	}}}	&	\multirow{3}{*}{	O/C	}	&	\multirow{3}{*}{	chair, seat	}	&	\multirow{3}{*}{	Austroasiatic	}	&				\textit{	th	 - }	\textthai{	เก้าอี้	}		\\&&&&				\textit{	lo	 - }	\textlao{	ເກົ້າອີ້	}		\\&&&&	\textit{	km	 - }	\textkhmer{	កៅអី 	}		\\\arrayrulecolor{gray} \hline
\multirow{3}{*}{	\textbf{\textit{	kawaca	}}}	&	\multirow{3}{*}{	O/C	}	&	\multirow{3}{*}{	shield	}	&	\multirow{3}{*}{	Compound	}	&	\multirow{	3	}{*}{	\textit{		}				}	\\&&&&				\textit{	tl	 - }					\\&&&&	\textit{		}					\\\arrayrulecolor{gray} \hline
\multirow{3}{*}{	\textbf{\textit{	kaybang	}}}	&	\multirow{3}{*}{	Action	}	&	\multirow{3}{*}{	open	}	&	\multirow{3}{*}{	Sinitic	}	&	\multirow{	3	}{*}{	\textit{		}		開放		}	\\&&&&				\textit{		}					\\&&&&	\textit{		}					\\\arrayrulecolor{gray} \hline
\multirow{3}{*}{	\textbf{\textit{	kaysi	}}}	&	\multirow{3}{*}{	Action	}	&	\multirow{3}{*}{	start, to turn on, begin	}	&	\multirow{3}{*}{	Sinitic	}	&	\multirow{	3	}{*}{	\textit{		}		開始		}	\\&&&&				\textit{		}					\\&&&&	\textit{		}					\\\arrayrulecolor{gray} \hline
\multirow{3}{*}{	\textbf{\textit{	kayu	}}}	&	\multirow{3}{*}{	Action	}	&	\multirow{3}{*}{	keep (an animal) ; herd	}	&	\multirow{3}{*}{	Koreo-Japonic 	}	&	\multirow{	2	}{*}{	\textit{	ko	 - }		키우다		}	\\&&&&	\multirow{	2	}{*}{	\textit{	jp	 - }		飼(か)う		}	\\&&&&	\textit{		}					\\\arrayrulecolor{gray} \hline
\multirow{3}{*}{	\textbf{\textit{	kel	}}}	&	\multirow{3}{*}{	O/C	}	&	\multirow{3}{*}{	shield	}	&	\multirow{3}{*}{	Austroasiatic	}	&				\textit{	th	 - }	\textthai{	เขน	}		\\&&&&				\textit{	vi	 - }		khiên			\\&&&&	\textit{	km	 - }	\textkhmer{	ខែល	}		\\\arrayrulecolor{gray} \hline
\multirow{3}{*}{	\textbf{\textit{	kelu	}}}	&	\multirow{3}{*}{	Action	}	&	\multirow{3}{*}{	to cut	}	&	\multirow{3}{*}{	Koreo-Japonic	}	&	\multirow{	2	}{*}{	\textit{	ko	 - }		가르다		}	\\&&&&	\multirow{	2	}{*}{	\textit{	ja	 - }		きる		}	\\&&&&	\textit{		}					\\\arrayrulecolor{gray} \hline
\multirow{3}{*}{	\textbf{\textit{	keluko	}}}	&	\multirow{3}{*}{	O/C	}	&	\multirow{3}{*}{	sword	}	&	\multirow{3}{*}{	Compound	}	&	\multirow{	3	}{*}{	\textit{		}				}	\\&&&&				\textit{		}					\\&&&&	\textit{		}					\\\arrayrulecolor{gray} \hline
\multirow{3}{*}{	\textbf{\textit{	kem	}}}	&	\multirow{3}{*}{	O/C	}	&	\multirow{3}{*}{	cheek	}	&	\multirow{3}{*}{	Austroasiatic	}	&	\multirow{	2	}{*}{	\textit{	th	 - }	\textthai{	แก้ม	}	}	\\&&&&	\multirow{	2	}{*}{	\textit{	lo	 - }	\textlao{	ແກ້ມ	}	}	\\&&&&	\textit{		}					\\\arrayrulecolor{gray} \hline
\multirow{3}{*}{	\textbf{\textit{	ken	}}}	&	\multirow{3}{*}{	Action	}	&	\multirow{3}{*}{	look / see	}	&	\multirow{3}{*}{	Sinitic	}	&	\multirow{	3	}{*}{	\textit{		}		見;觀		}	\\&&&&				\textit{		}					\\&&&&	\textit{		}					\\\arrayrulecolor{gray} \hline
\multirow{3}{*}{	\textbf{\textit{	kentang	}}}	&	\multirow{3}{*}{	O/C	}	&	\multirow{3}{*}{	potato	}	&	\multirow{3}{*}{	Austronesian	}	&	\multirow{	3	}{*}{	\textit{	ms/id	 - }		kentang		}	\\&&&&				\textit{		}					\\&&&&	\textit{		}					\\\arrayrulecolor{gray} \hline
\multirow{3}{*}{	\textbf{\textit{	kepala	}}}	&	\multirow{3}{*}{	O/C	}	&	\multirow{3}{*}{	head	}	&	\multirow{3}{*}{	Austronesian	}	&	\multirow{	3	}{*}{	\textit{	ms/id	 - }		kepala		}	\\&&&&				\textit{		}					\\&&&&	\textit{		}					\\\arrayrulecolor{gray} \hline
\multirow{3}{*}{	\textbf{\textit{	kibit	}}}	&	\multirow{3}{*}{	Action	}	&	\multirow{3}{*}{	write	}	&	\multirow{3}{*}{	Sinitic	}	&	\multirow{	3	}{*}{	\textit{		}		記		}	\\&&&&				\textit{		}					\\&&&&	\textit{		}					\\\arrayrulecolor{gray} \hline
\multirow{3}{*}{	\textbf{\textit{	kibot	}}}	&	\multirow{3}{*}{	O/C	}	&	\multirow{3}{*}{	keyboard	}	&	\multirow{3}{*}{	Western: English	}	&	\multirow{	3	}{*}{	\textit{	en	 - }		keyboard		}	\\&&&&				\textit{		}					\\&&&&	\textit{		}					\\\arrayrulecolor{gray} \hline
\multirow{3}{*}{	\textbf{\textit{	kibun	}}}	&	\multirow{3}{*}{	O/C	}	&	\multirow{3}{*}{	feeling	}	&	\multirow{3}{*}{	Sinitic	}	&	\multirow{	3	}{*}{	\textit{		}		氣分		}	\\&&&&				\textit{		}					\\&&&&	\textit{		}					\\\arrayrulecolor{gray} \hline
\multirow{3}{*}{	\textbf{\textit{	kicune	}}}	&	\multirow{3}{*}{	O/C	}	&	\multirow{3}{*}{	fox	}	&	\multirow{3}{*}{	Koreo-Japonic	}	&	\multirow{	3	}{*}{	\textit{	ja	 - }		きつね		}	\\&&&&				\textit{		}					\\&&&&	\textit{		}					\\\arrayrulecolor{gray} \hline
\multirow{3}{*}{	\textbf{\textit{	kijang	}}}	&	\multirow{3}{*}{	Action	}	&	\multirow{3}{*}{	wake up, getting up (from bed)	}	&	\multirow{3}{*}{	Sinitic	}	&	\multirow{	3	}{*}{	\textit{		}		起床		}	\\&&&&				\textit{		}					\\&&&&	\textit{		}					\\\arrayrulecolor{gray} \hline
\multirow{3}{*}{	\textbf{\textit{	kilesa	}}}	&	\multirow{3}{*}{	O/C	}	&	\multirow{3}{*}{	sin; stress; emotional damage	}	&	\multirow{3}{*}{	Sanskrit	}	&	\multirow{	3	}{*}{	\textit{		}	\textsanskrit{	क्लेश }(kleśa)	}	\\&&&&				\textit{		}					\\&&&&	\textit{		}					\\\arrayrulecolor{gray} \hline
\multirow{3}{*}{	\textbf{\textit{	king	}}}	&	\multirow{3}{*}{	O/C	}	&	\multirow{3}{*}{	ginger	}	&	\multirow{3}{*}{	Austroasiatic	}	&	\multirow{	2	}{*}{	\textit{	th	 - }	\textthai{	ขิง	}	}	\\&&&&	\multirow{	2	}{*}{	\textit{	lo	 - }	\textlao{	ຂີງ	}	}	\\&&&&	\textit{		}					\\\arrayrulecolor{gray} \hline
\multirow{3}{*}{	\textbf{\textit{	ko	}}}	&	\multirow{3}{*}{	O/C	}	&	\multirow{3}{*}{	this	}	&	\multirow{3}{*}{	Koreo-Japonic	}	&	\multirow{	2	}{*}{	\textit{	ko	 - }		그		}	\\&&&&	\multirow{	2	}{*}{	\textit{	ja	 - }		こ		}	\\&&&&	\textit{		}					\\\arrayrulecolor{gray} \hline
\multirow{3}{*}{	\textbf{\textit{	ko	}}}	&	\multirow{3}{*}{	O/C	}	&	\multirow{3}{*}{	portrusion (from an affixed noun)	}	&	\multirow{3}{*}{	N/A	}	&	\multirow{	3	}{*}{	\textit{		}				}	\\&&&&				\textit{		}					\\&&&&	\textit{		}					\\\arrayrulecolor{gray} \hline
\multirow{3}{*}{	\textbf{\textit{	koci	}}}	&	\multirow{3}{*}{	O/C	}	&	\multirow{3}{*}{	tip (object), twist (literary) 	}	&	\multirow{3}{*}{	Koreo-Japonic	}	&	\multirow{	2	}{*}{	\textit{	ko	 - }		꼭지		}	\\&&&&	\multirow{	2	}{*}{	\textit{	jp	 - }		おし		}	\\&&&&	\textit{		}					\\\arrayrulecolor{gray} \hline
\multirow{3}{*}{	\textbf{\textit{	kodi	}}}	&	\multirow{3}{*}{	O/C	}	&	\multirow{3}{*}{	here	}	&	\multirow{3}{*}{	Compound	}	&	\multirow{	3	}{*}{	\textit{		}		ko + di		}	\\&&&&				\textit{		}					\\&&&&	\textit{		}					\\\arrayrulecolor{gray} \hline
\multirow{3}{*}{	\textbf{\textit{	koma	}}}	&	\multirow{3}{*}{	O/C	}	&	\multirow{3}{*}{	bear	}	&	\multirow{3}{*}{	Koreo-Japonic	}	&	\multirow{	2	}{*}{	\textit{	ko	 - }		곰		}	\\&&&&	\multirow{	2	}{*}{	\textit{	ja	 - }		くま		}	\\&&&&	\textit{		}					\\\arrayrulecolor{gray} \hline
\multirow{3}{*}{	\textbf{\textit{	kompyuta	}}}	&	\multirow{3}{*}{	O/C	}	&	\multirow{3}{*}{	computer	}	&	\multirow{3}{*}{	Western: English	}	&	\multirow{	3	}{*}{	\textit{	en	 - }		computer		}	\\&&&&				\textit{		}					\\&&&&	\textit{		}					\\\arrayrulecolor{gray} \hline
\multirow{3}{*}{	\textbf{\textit{	kon	}}}	&	\multirow{3}{*}{	O/C	}	&	\multirow{3}{*}{	butt	}	&	\multirow{3}{*}{	Austroasiatic	}	&	\multirow{	2	}{*}{	\textit{	th	 - }	\textthai{	ก้น	}	}	\\&&&&	\multirow{	2	}{*}{	\textit{	lo	 - }	\textlao{	ກົ້ນ	}	}	\\&&&&	\textit{		}					\\\arrayrulecolor{gray} \hline
\multirow{3}{*}{	\textbf{\textit{	kong	}}}	&	\multirow{3}{*}{	O/C	}	&	\multirow{3}{*}{	kid, child, children	}	&	\multirow{3}{*}{	Koreo-Japonic	}	&	\multirow{	2	}{*}{	\textit{	ko	 - }		꼬마		}	\\&&&&	\multirow{	2	}{*}{	\textit{	ja	 - }		こ		}	\\&&&&	\textit{		}					\\\arrayrulecolor{gray} \hline
\multirow{3}{*}{	\textbf{\textit{	kong pipal	}}}	&	\multirow{3}{*}{	O/C	}	&	\multirow{3}{*}{	black pepper	}	&	\multirow{3}{*}{	Compound	}	&	\multirow{	3	}{*}{	\textit{		}				}	\\&&&&				\textit{		}					\\&&&&	\textit{		}					\\\arrayrulecolor{gray} \hline
\multirow{3}{*}{	\textbf{\textit{	konghong	}}}	&	\multirow{3}{*}{	O/C	}	&	\multirow{3}{*}{	airport	}	&	\multirow{3}{*}{	Sinitic	}	&	\multirow{	3	}{*}{	\textit{		}		空港		}	\\&&&&				\textit{		}					\\&&&&	\textit{		}					\\\arrayrulecolor{gray} \hline
\multirow{3}{*}{	\textbf{\textit{	konok	}}}	&	\multirow{3}{*}{	O/C	}	&	\multirow{3}{*}{	feather	}	&	\multirow{3}{*}{	Austroasiatic	}	&	\multirow{	2	}{*}{	\textit{	th	 - }	\textthai{	ขนนก	}	}	\\&&&&	\multirow{	2	}{*}{	\textit{	lo	 - }	\textlao{	ຂົນ  ນົກ	}	}	\\&&&&	\textit{		}					\\\arrayrulecolor{gray} \hline
\multirow{3}{*}{	\textbf{\textit{	kontulola	}}}	&	\multirow{3}{*}{	O/C	}	&	\multirow{3}{*}{	controller	}	&	\multirow{3}{*}{	Western: English	}	&	\multirow{	3	}{*}{	\textit{	en	 - }		controller		}	\\&&&&				\textit{		}					\\&&&&	\textit{		}					\\\arrayrulecolor{gray} \hline
\multirow{3}{*}{	\textbf{\textit{	kot	}}}	&	\multirow{3}{*}{	O/C	}	&	\multirow{3}{*}{	place / space	}	&	\multirow{3}{*}{	Koreo-Japonic	}	&	\multirow{	3	}{*}{	\textit{	ko	 - }		곳		}	\\&&&&				\textit{		}					\\&&&&	\textit{		}					\\\arrayrulecolor{gray} \hline
\multirow{3}{*}{	\textbf{\textit{	kot	}}}	&	\multirow{3}{*}{	O/C	}	&	\multirow{3}{*}{	thing	}	&	\multirow{3}{*}{	Koreo-Japonic	}	&	\multirow{	2	}{*}{	\textit{	ko	 - }		것		}	\\&&&&	\multirow{	2	}{*}{	\textit{	ja	 - }		こと		}	\\&&&&	\textit{		}					\\\arrayrulecolor{gray} \hline
\multirow{3}{*}{	\textbf{\textit{	kotong	}}}	&	\multirow{3}{*}{	O/C	}	&	\multirow{3}{*}{	pain, agony	}	&	\multirow{3}{*}{	Sinitic	}	&	\multirow{	3	}{*}{	\textit{		}		苦痛		}	\\&&&&				\textit{		}					\\&&&&	\textit{		}					\\\arrayrulecolor{gray} \hline
\multirow{3}{*}{	\textbf{\textit{	kow	}}}	&	\multirow{3}{*}{	O/C	}	&	\multirow{3}{*}{	nine	}	&	\multirow{3}{*}{	Sinitic	}	&	\multirow{	3	}{*}{	\textit{		}		九		}	\\&&&&				\textit{		}					\\&&&&	\textit{		}					\\\arrayrulecolor{gray} \hline
\multirow{3}{*}{	\textbf{\textit{	ku	}}}	&	\multirow{3}{*}{	Qual	}	&	\multirow{3}{*}{	old	}	&	\multirow{3}{*}{	Sinitic	}	&	\multirow{	3	}{*}{	\textit{		}		久		}	\\&&&&				\textit{		}					\\&&&&	\textit{		}					\\\arrayrulecolor{gray} \hline
\multirow{3}{*}{	\textbf{\textit{	kubi	}}}	&	\multirow{3}{*}{	O/C	}	&	\multirow{3}{*}{	neck, throat	}	&	\multirow{3}{*}{	Koreo-Japonic	}	&	\multirow{	3	}{*}{	\textit{	ja	 - }		くび		}	\\&&&&				\textit{		}					\\&&&&	\textit{		}					\\\arrayrulecolor{gray} \hline
\multirow{3}{*}{	\textbf{\textit{	kucing	}}}	&	\multirow{3}{*}{	O/C	}	&	\multirow{3}{*}{	cat; other felines	}	&	\multirow{3}{*}{	Austronesian	}	&	\multirow{	3	}{*}{	\textit{	ms/id	 - }		kucing		}	\\&&&&				\textit{		}					\\&&&&	\textit{		}					\\\arrayrulecolor{gray} \hline
\multirow{3}{*}{	\textbf{\textit{	kuku	}}}	&	\multirow{3}{*}{	O/C	}	&	\multirow{3}{*}{	nails (anatomy)	}	&	\multirow{3}{*}{	Austronesian	}	&	\multirow{	3	}{*}{	\textit{	ms/id	 - }		kuku		}	\\&&&&				\textit{		}					\\&&&&	\textit{		}					\\\arrayrulecolor{gray} \hline
\multirow{3}{*}{	\textbf{\textit{	kulat	}}}	&	\multirow{3}{*}{	O/C	}	&	\multirow{3}{*}{	fungi, fungus, mushroom	}	&	\multirow{3}{*}{	Austronesian	}	&	\multirow{	3	}{*}{	\textit{	ms	 - }		kulat		}	\\&&&&				\textit{		}					\\&&&&	\textit{		}					\\\arrayrulecolor{gray} \hline
\multirow{3}{*}{	\textbf{\textit{	kulit	}}}	&	\multirow{3}{*}{	O/C	}	&	\multirow{3}{*}{	skin, leather	}	&	\multirow{3}{*}{	Austronesian	}	&	\multirow{	3	}{*}{	\textit{	ms/id	 - }		kulit		}	\\&&&&				\textit{		}					\\&&&&	\textit{		}					\\\arrayrulecolor{gray} \hline
\multirow{3}{*}{	\textbf{\textit{	kung	}}}	&	\multirow{3}{*}{	Qual	}	&	\multirow{3}{*}{	black	}	&	\multirow{3}{*}{	Koreo-Japonic	}	&	\multirow{	2	}{*}{	\textit{	ko	 - }		검은		}	\\&&&&	\multirow{	2	}{*}{	\textit{	ja	 - }		くろ		}	\\&&&&	\textit{		}					\\\arrayrulecolor{gray} \hline
\multirow{3}{*}{	\textbf{\textit{	kwa	}}}	&	\multirow{3}{*}{	O/C	}	&	\multirow{3}{*}{	fear	}	&	\multirow{3}{*}{	Austroasiatic	}	&	\multirow{	2	}{*}{	\textit{	th	 - }	\textthai{	กลัว	}	}	\\&&&&	\multirow{	2	}{*}{	\textit{	lo	 - }	\textlao{	ກົວ	}	}	\\&&&&	\textit{		}					\\\arrayrulecolor{gray} \hline
\multirow{3}{*}{	\textbf{\textit{	kwakwak	}}}	&	\multirow{3}{*}{	O/C	}	&	\multirow{3}{*}{	duck, weird, strange	}	&	\multirow{3}{*}{	Sound-based	}	&	\multirow{	3	}{*}{	\textit{		}				}	\\&&&&				\textit{		}					\\&&&&	\textit{		}					\\\arrayrulecolor{gray} \hline
\multirow{3}{*}{	\textbf{\textit{	kyo	}}}	&	\multirow{3}{*}{	Action	}	&	\multirow{3}{*}{	go, pass (e.g. days)	}	&	\multirow{3}{*}{	Sinitic	}	&	\multirow{	3	}{*}{	\textit{		}		去		}	\\&&&&				\textit{		}					\\&&&&	\textit{		}					\\\arrayrulecolor{gray} \hline
\multirow{3}{*}{	\textbf{\textit{	labip	}}}	&	\multirow{3}{*}{	O/C	}	&	\multirow{3}{*}{	lips	}	&	\multirow{3}{*}{	Austronesian	}	&	\multirow{	2	}{*}{	\textit{	ms/id	 - }		bibir		}	\\&&&&	\multirow{	2	}{*}{	\textit{	tl	 - }		labi		}	\\&&&&	\textit{		}					\\\arrayrulecolor{gray} \hline
\multirow{3}{*}{	\textbf{\textit{	lagu	}}}	&	\multirow{3}{*}{	O/C	}	&	\multirow{3}{*}{	lag	}	&	\multirow{3}{*}{	Western: English	}	&	\multirow{	3	}{*}{	\textit{	en	 - }		lag		}	\\&&&&				\textit{		}					\\&&&&	\textit{		}					\\\arrayrulecolor{gray} \hline
\multirow{3}{*}{	\textbf{\textit{	laja	}}}	&	\multirow{3}{*}{	O/C	}	&	\multirow{3}{*}{	king	}	&	\multirow{3}{*}{	Sanskrit	}	&	\multirow{	2	}{*}{	\textit{		}	\textsanskrit{	राजन् 	}	}	\\&&&&	\multirow{	2	}{*}{	\textit{		}		rā́jan		}	\\&&&&	\textit{		}					\\\arrayrulecolor{gray} \hline
\multirow{3}{*}{	\textbf{\textit{	lajasaci	}}}	&	\multirow{3}{*}{	O/C	}	&	\multirow{3}{*}{	queen (wife of king)	}	&	\multirow{3}{*}{	Sanskrit	}	&	\multirow{	2	}{*}{	\textit{		}	\textsanskrit{	राजन् + स्त्री 	}	}	\\&&&&	\multirow{	2	}{*}{	\textit{		}		(rā́jan + strī́)		}	\\&&&&	\textit{		}					\\\arrayrulecolor{gray} \hline
\multirow{3}{*}{	\textbf{\textit{	lang'it	}}}	&	\multirow{3}{*}{	O/C	}	&	\multirow{3}{*}{	sky	}	&	\multirow{3}{*}{	Austronesian	}	&	\multirow{	2	}{*}{	\textit{	ms/id	 - }		langit		}	\\&&&&	\multirow{	2	}{*}{	\textit{	tl	 - }		langit		}	\\&&&&	\textit{		}					\\\arrayrulecolor{gray} \hline
\multirow{3}{*}{	\textbf{\textit{	langka	}}}	&	\multirow{3}{*}{	O/C	}	&	\multirow{3}{*}{	roof	}	&	\multirow{3}{*}{	Austroasiatic	}	&	\multirow{	2	}{*}{	\textit{	th	 - }	\textthai{	หลังคา	}	}	\\&&&&	\multirow{	2	}{*}{	\textit{	lo	 - }	\textlao{	ຫຼັງ  ຄາ}(枷)		}	\\&&&&	\textit{		}					\\\arrayrulecolor{gray} \hline
\multirow{3}{*}{	\textbf{\textit{	langkay	}}}	&	\multirow{3}{*}{	O/C	}	&	\multirow{3}{*}{	body, torso	}	&	\multirow{3}{*}{	Austroasiatic	}	&	\multirow{	2	}{*}{	\textit{	th	 - }	\textthai{	ร่างกาย	}	}	\\&&&&	\multirow{	2	}{*}{	\textit{	lo	 - }	\textlao{	ຮ່າງກາຍ 	}	}	\\&&&&	\textit{		}					\\\arrayrulecolor{gray} \hline
\multirow{3}{*}{	\textbf{\textit{	lantay	}}}	&	\multirow{3}{*}{	O/C	}	&	\multirow{3}{*}{	floor	}	&	\multirow{3}{*}{	Austronesian	}	&	\multirow{	3	}{*}{	\textit{	ms/id	 - }		lantai		}	\\&&&&				\textit{		}					\\&&&&	\textit{		}					\\\arrayrulecolor{gray} \hline
\multirow{3}{*}{	\textbf{\textit{	laot	}}}	&	\multirow{3}{*}{	O/C	}	&	\multirow{3}{*}{	sea	}	&	\multirow{3}{*}{	Austronesian	}	&	\multirow{	2	}{*}{	\textit{	ms/id	 - }		laut		}	\\&&&&	\multirow{	2	}{*}{	\textit{	tl	 - }		lawa 		}	\\&&&&	\textit{		}					\\\arrayrulecolor{gray} \hline
\multirow{3}{*}{	\textbf{\textit{	lay	}}}	&	\multirow{3}{*}{	Action	}	&	\multirow{3}{*}{	come	}	&	\multirow{3}{*}{	Sinitic	}	&	\multirow{	3	}{*}{	\textit{		}		來		}	\\&&&&				\textit{		}					\\&&&&	\textit{		}					\\\arrayrulecolor{gray} \hline
\multirow{3}{*}{	\textbf{\textit{	lema	}}}	&	\multirow{3}{*}{	Qual	}	&	\multirow{3}{*}{	weak	}	&	\multirow{3}{*}{	Austronesian	}	&	\multirow{	3	}{*}{	\textit{	ms/id	 - }		lemah		}	\\&&&&				\textit{		}					\\&&&&	\textit{		}					\\\arrayrulecolor{gray} \hline
\multirow{3}{*}{	\textbf{\textit{	leng	}}}	&	\multirow{3}{*}{	Qual	}	&	\multirow{3}{*}{	cool (literal and as English slang) 	}	&	\multirow{3}{*}{	Sinitic	}	&	\multirow{	3	}{*}{	\textit{		}		冷		}	\\&&&&				\textit{		}					\\&&&&	\textit{		}					\\\arrayrulecolor{gray} \hline
\multirow{3}{*}{	\textbf{\textit{	lenken	}}}	&	\multirow{3}{*}{	O/C	}	&	\multirow{3}{*}{	arms	}	&	\multirow{3}{*}{	Compound	}	&	\multirow{	2	}{*}{	\textit{	ms	 - }		lengan		}	\\&&&&	\multirow{	2	}{*}{	\textit{	th	 - }	\textthai{	แขน	}	}	\\&&&&	\textit{		}					\\\arrayrulecolor{gray} \hline
\multirow{3}{*}{	\textbf{\textit{	lenken-ko	}}}	&	\multirow{3}{*}{	O/C	}	&	\multirow{3}{*}{	wrist	}	&	\multirow{3}{*}{	Compound	}	&	\multirow{	3	}{*}{	\textit{		}				}	\\&&&&				\textit{		}					\\&&&&	\textit{		}					\\\arrayrulecolor{gray} \hline
\multirow{3}{*}{	\textbf{\textit{	lensip	}}}	&	\multirow{3}{*}{	Action	}	&	\multirow{3}{*}{	practice	}	&	\multirow{3}{*}{	Sinitic	}	&	\multirow{	3	}{*}{	\textit{		}		練習		}	\\&&&&				\textit{		}					\\&&&&	\textit{		}					\\\arrayrulecolor{gray} \hline
\multirow{3}{*}{	\textbf{\textit{	letca	}}}	&	\multirow{3}{*}{	O/C	}	&	\multirow{3}{*}{	train	}	&	\multirow{3}{*}{		}	&	\multirow{	3	}{*}{	\textit{		}				}	\\&&&&				\textit{		}					\\&&&&	\textit{		}					\\\arrayrulecolor{gray} \hline
\multirow{3}{*}{	\textbf{\textit{	ley	}}}	&	\multirow{3}{*}{	O/C	}	&	\multirow{3}{*}{	example	}	&	\multirow{3}{*}{	Sinitic	}	&	\multirow{	3	}{*}{	\textit{		}		例		}	\\&&&&				\textit{		}					\\&&&&	\textit{		}					\\\arrayrulecolor{gray} \hline
\multirow{3}{*}{	\textbf{\textit{	lida	}}}	&	\multirow{3}{*}{	O/C	}	&	\multirow{3}{*}{	tongue	}	&	\multirow{3}{*}{	Austronesian	}	&	\multirow{	3	}{*}{	\textit{	ms/id	 - }		lidah		}	\\&&&&				\textit{		}					\\&&&&	\textit{		}					\\\arrayrulecolor{gray} \hline
\multirow{3}{*}{	\textbf{\textit{	ligay	}}}	&	\multirow{3}{*}{	Action	}	&	\multirow{3}{*}{	understand, comprehend	}	&	\multirow{3}{*}{	Sinitic	}	&	\multirow{	3	}{*}{	\textit{		}		理解		}	\\&&&&				\textit{		}					\\&&&&	\textit{		}					\\\arrayrulecolor{gray} \hline
\multirow{3}{*}{	\textbf{\textit{	lip	}}}	&	\multirow{3}{*}{	Action	}	&	\multirow{3}{*}{	stand, to rise, to lift, grow	}	&	\multirow{3}{*}{	Sinitic	}	&	\multirow{	3	}{*}{	\textit{		}		立		}	\\&&&&				\textit{		}					\\&&&&	\textit{		}					\\\arrayrulecolor{gray} \hline
\multirow{3}{*}{	\textbf{\textit{	lip	}}}	&	\multirow{3}{*}{	O/C	}	&	\multirow{3}{*}{	elevator, a lift	}	&	\multirow{3}{*}{	Sinitic	}	&	\multirow{	3	}{*}{	\textit{		}		口+立		}	\\&&&&				\textit{		}					\\&&&&	\textit{		}					\\\arrayrulecolor{gray} \hline
\multirow{3}{*}{	\textbf{\textit{	lodu	}}}	&	\multirow{3}{*}{	O/C	}	&	\multirow{3}{*}{	season	}	&	\multirow{3}{*}{	Sanskrit	}	&	\multirow{	2	}{*}{	\textit{		}	\textsanskrit{	ऋतु  	}	}	\\&&&&	\multirow{	2	}{*}{	\textit{		}		(ṛtú)		}	\\&&&&	\textit{		}					\\\arrayrulecolor{gray} \hline
\multirow{3}{*}{	\textbf{\textit{	losyao	}}}	&	\multirow{3}{*}{	O/C	}	&	\multirow{3}{*}{	russian (language)	}	&	\multirow{3}{*}{	Compound	}	&	\multirow{	3	}{*}{	\textit{		}				}	\\&&&&				\textit{		}					\\&&&&	\textit{		}					\\\arrayrulecolor{gray} \hline
\multirow{3}{*}{	\textbf{\textit{	luk	}}}	&	\multirow{3}{*}{	O/C	}	&	\multirow{3}{*}{	six	}	&	\multirow{3}{*}{	Sinitic	}	&	\multirow{	3	}{*}{	\textit{		}		六		}	\\&&&&				\textit{		}					\\&&&&	\textit{		}					\\\arrayrulecolor{gray} \hline
\multirow{3}{*}{	\textbf{\textit{	lweng	}}}	&	\multirow{3}{*}{	Qual	}	&	\multirow{3}{*}{	yellow	}	&	\multirow{3}{*}{	Austronesian	}	&	\multirow{	3	}{*}{	\textit{		}				}	\\&&&&				\textit{		}					\\&&&&	\textit{		}					\\\arrayrulecolor{gray} \hline
\multirow{3}{*}{	\textbf{\textit{	lwisi	}}}	&	\multirow{3}{*}{	Action	}	&	\multirow{3}{*}{	like, similar	}	&	\multirow{3}{*}{	Sinitic	}	&	\multirow{	3	}{*}{	\textit{		}		類似		}	\\&&&&				\textit{		}					\\&&&&	\textit{		}					\\\arrayrulecolor{gray} \hline
\multirow{3}{*}{	\textbf{\textit{	lyawli	}}}	&	\multirow{3}{*}{	Action	}	&	\multirow{3}{*}{	to cook	}	&	\multirow{3}{*}{	Sinitic	}	&	\multirow{	3	}{*}{	\textit{		}		料理		}	\\&&&&				\textit{		}					\\&&&&	\textit{		}					\\\arrayrulecolor{gray} \hline
\multirow{3}{*}{	\textbf{\textit{	mabap	}}}	&	\multirow{3}{*}{	O/C	}	&	\multirow{3}{*}{	magic	}	&	\multirow{3}{*}{	Sinitic	}	&	\multirow{	3	}{*}{	\textit{		}		魔法		}	\\&&&&				\textit{		}					\\&&&&	\textit{		}					\\\arrayrulecolor{gray} \hline
\multirow{3}{*}{	\textbf{\textit{	maha	}}}	&	\multirow{3}{*}{	Qual	}	&	\multirow{3}{*}{	expensive	}	&	\multirow{3}{*}{	Sanskrit	}	&	\multirow{	2	}{*}{	\textit{		}	\textsanskrit{	महार्घ	}	}	\\&&&&	\multirow{	2	}{*}{	\textit{		}		(mahārgha)		}	\\&&&&	\textit{		}					\\\arrayrulecolor{gray} \hline
\multirow{3}{*}{	\textbf{\textit{	mala	}}}	&	\multirow{3}{*}{	Qual	}	&	\multirow{3}{*}{	angry, mad, upset	}	&	\multirow{3}{*}{	Austronesian	}	&	\multirow{	3	}{*}{	\textit{	ms/id	 - }		marah		}	\\&&&&				\textit{		}					\\&&&&	\textit{		}					\\\arrayrulecolor{gray} \hline
\multirow{3}{*}{	\textbf{\textit{	malay'o	}}}	&	\multirow{3}{*}{	O/C	}	&	\multirow{3}{*}{	malaysian (language)	}	&	\multirow{3}{*}{	Compound	}	&	\multirow{	3	}{*}{	\textit{		}				}	\\&&&&				\textit{		}					\\&&&&	\textit{		}					\\\arrayrulecolor{gray} \hline
\multirow{3}{*}{	\textbf{\textit{	mama	}}}	&	\multirow{3}{*}{	O/C	}	&	\multirow{3}{*}{	mom, mother, mama	}	&	\multirow{3}{*}{	N/A	}	&	\multirow{	3	}{*}{	\textit{		}				}	\\&&&&				\textit{		}					\\&&&&	\textit{		}					\\\arrayrulecolor{gray} \hline
\multirow{3}{*}{	\textbf{\textit{	man	}}}	&	\multirow{3}{*}{	O/C	}	&	\multirow{3}{*}{	ten-thousand, (poetic) all	}	&	\multirow{3}{*}{	Sinitic	}	&	\multirow{	3	}{*}{	\textit{		}		萬		}	\\&&&&				\textit{		}					\\&&&&	\textit{		}					\\\arrayrulecolor{gray} \hline
\multirow{3}{*}{	\textbf{\textit{	mang	}}}	&	\multirow{3}{*}{	Qual	}	&	\multirow{3}{*}{	busy	}	&	\multirow{3}{*}{	Sinitic	}	&	\multirow{	3	}{*}{	\textit{		}		忙		}	\\&&&&				\textit{		}					\\&&&&	\textit{		}					\\\arrayrulecolor{gray} \hline
\multirow{3}{*}{	\textbf{\textit{	mangkuk	}}}	&	\multirow{3}{*}{	O/C	}	&	\multirow{3}{*}{	bowl	}	&	\multirow{3}{*}{	Austronesian	}	&				\textit{	ms/id	 - }		mangkuk			\\&&&&				\textit{	tl	 - }		mangkok			\\&&&&	\textit{	jv	 - }		mangkok			\\\arrayrulecolor{gray} \hline
\multirow{3}{*}{	\textbf{\textit{	mangkyak	}}}	&	\multirow{3}{*}{	Action	}	&	\multirow{3}{*}{	forget	}	&	\multirow{3}{*}{	Sinitic	}	&	\multirow{	3	}{*}{	\textit{		}		忘却		}	\\&&&&				\textit{		}					\\&&&&	\textit{		}					\\\arrayrulecolor{gray} \hline
\multirow{3}{*}{	\textbf{\textit{	manhwa	}}}	&	\multirow{3}{*}{	O/C	}	&	\multirow{3}{*}{	comics, manga	}	&	\multirow{3}{*}{	Sinitic	}	&	\multirow{	3	}{*}{	\textit{		}		漫畵		}	\\&&&&				\textit{		}					\\&&&&	\textit{		}					\\\arrayrulecolor{gray} \hline
\multirow{3}{*}{	\textbf{\textit{	manmin	}}}	&	\multirow{3}{*}{	O/C	}	&	\multirow{3}{*}{	everyone	}	&	\multirow{3}{*}{	Sinitic	}	&	\multirow{	3	}{*}{	\textit{		}		萬民		}	\\&&&&				\textit{		}					\\&&&&	\textit{		}					\\\arrayrulecolor{gray} \hline
\multirow{3}{*}{	\textbf{\textit{	manusya	}}}	&	\multirow{3}{*}{	O/C	}	&	\multirow{3}{*}{	human	}	&	\multirow{3}{*}{	Sanskrit	}	&	\multirow{	2	}{*}{	\textit{		}	\textsanskrit{	मनुष्य 	}	}	\\&&&&	\multirow{	2	}{*}{	\textit{		}		(manuṣya)		}	\\&&&&	\textit{		}					\\\arrayrulecolor{gray} \hline
\multirow{3}{*}{	\textbf{\textit{	mata	}}}	&	\multirow{3}{*}{	O/C	}	&	\multirow{3}{*}{	eyes	}	&	\multirow{3}{*}{	Austronesian	}	&	\multirow{	2	}{*}{	\textit{	ms/id	 - }		mata		}	\\&&&&	\multirow{	2	}{*}{	\textit{	tl	 - }		mata		}	\\&&&&	\textit{		}					\\\arrayrulecolor{gray} \hline
\multirow{3}{*}{	\textbf{\textit{	matwi	}}}	&	\multirow{3}{*}{	O/C	}	&	\multirow{3}{*}{	village; town, rural, countryside	}	&	\multirow{3}{*}{	Koreo-Japonic	}	&	\multirow{	2	}{*}{	\textit{	ko	 - }		마을		}	\\&&&&	\multirow{	2	}{*}{	\textit{	ja	 - }		まち		}	\\&&&&	\textit{		}					\\\arrayrulecolor{gray} \hline
\multirow{3}{*}{	\textbf{\textit{	mawsu	}}}	&	\multirow{3}{*}{	O/C	}	&	\multirow{3}{*}{	mouse (computing) 	}	&	\multirow{3}{*}{	Western: English	}	&	\multirow{	3	}{*}{	\textit{	en	 - }		mouse		}	\\&&&&				\textit{		}					\\&&&&	\textit{		}					\\\arrayrulecolor{gray} \hline
\multirow{3}{*}{	\textbf{\textit{	maymay	}}}	&	\multirow{3}{*}{	O/C	}	&	\multirow{3}{*}{	younger sister	}	&	\multirow{3}{*}{	Sinitic	}	&	\multirow{	3	}{*}{	\textit{		}		妹妹		}	\\&&&&				\textit{		}					\\&&&&	\textit{		}					\\\arrayrulecolor{gray} \hline
\multirow{3}{*}{	\textbf{\textit{	maynen	}}}	&	\multirow{3}{*}{		}	&	\multirow{3}{*}{	yearly, every year	}	&	\multirow{3}{*}{	Sinitic	}	&	\multirow{	3	}{*}{	\textit{		}		每年		}	\\&&&&				\textit{		}					\\&&&&	\textit{		}					\\\arrayrulecolor{gray} \hline
\multirow{3}{*}{	\textbf{\textit{	maynit	}}}	&	\multirow{3}{*}{	O/C	}	&	\multirow{3}{*}{	everyday, daily	}	&	\multirow{3}{*}{	Sinitic	}	&	\multirow{	3	}{*}{	\textit{		}		每日		}	\\&&&&				\textit{		}					\\&&&&	\textit{		}					\\\arrayrulecolor{gray} \hline
\multirow{3}{*}{	\textbf{\textit{	mengji	}}}	&	\multirow{3}{*}{	O/C	}	&	\multirow{3}{*}{	name	}	&	\multirow{3}{*}{	Sinitic	}	&	\multirow{	3	}{*}{	\textit{		}		名字		}	\\&&&&				\textit{		}					\\&&&&	\textit{		}					\\\arrayrulecolor{gray} \hline
\multirow{3}{*}{	\textbf{\textit{	mi	}}}	&	\multirow{3}{*}{		}	&	\multirow{3}{*}{	still, yet	}	&	\multirow{3}{*}{	Sinitic	}	&	\multirow{	3	}{*}{	\textit{		}		未		}	\\&&&&				\textit{		}					\\&&&&	\textit{		}					\\\arrayrulecolor{gray} \hline
\multirow{3}{*}{	\textbf{\textit{	miji	}}}	&	\multirow{3}{*}{	O/C	}	&	\multirow{3}{*}{	rainbow	}	&	\multirow{3}{*}{	Koreo-Japonic	}	&	\multirow{	2	}{*}{	\textit{	ko	 - }		무지개		}	\\&&&&	\multirow{	2	}{*}{	\textit{	ja	 - }		ぬじ		}	\\&&&&	\textit{		}					\\\arrayrulecolor{gray} \hline
\multirow{3}{*}{	\textbf{\textit{	miley	}}}	&	\multirow{3}{*}{	Qual	}	&	\multirow{3}{*}{	beautiful	}	&	\multirow{3}{*}{	Sinitic	}	&	\multirow{	3	}{*}{	\textit{		}		美麗		}	\\&&&&				\textit{		}					\\&&&&	\textit{		}					\\\arrayrulecolor{gray} \hline
\multirow{3}{*}{	\textbf{\textit{	moji	}}}	&	\multirow{3}{*}{	O/C	}	&	\multirow{3}{*}{	hat	}	&	\multirow{3}{*}{	Sinitic	}	&	\multirow{	3	}{*}{	\textit{		}		帽子		}	\\&&&&				\textit{		}					\\&&&&	\textit{		}					\\\arrayrulecolor{gray} \hline
\multirow{3}{*}{	\textbf{\textit{	moli	}}}	&	\multirow{3}{*}{	O/C	}	&	\multirow{3}{*}{	horse	}	&	\multirow{3}{*}{	Altaic	}	&	\multirow{	3	}{*}{	\textit{	mn	 - }		морь		}	\\&&&&				\textit{		}					\\&&&&	\textit{		}					\\\arrayrulecolor{gray} \hline
\multirow{3}{*}{	\textbf{\textit{	mong	}}}	&	\multirow{3}{*}{	Action	}	&	\multirow{3}{*}{	dream	}	&	\multirow{3}{*}{	Sinitic	}	&	\multirow{	3	}{*}{	\textit{		}		夢		}	\\&&&&				\textit{		}					\\&&&&	\textit{		}					\\\arrayrulecolor{gray} \hline
\multirow{3}{*}{	\textbf{\textit{	monyet	}}}	&	\multirow{3}{*}{	O/C	}	&	\multirow{3}{*}{	monkey	}	&	\multirow{3}{*}{	Austronesian	}	&	\multirow{	3	}{*}{	\textit{	ms/id	 - }		monyet		}	\\&&&&				\textit{		}					\\&&&&	\textit{		}					\\\arrayrulecolor{gray} \hline
\multirow{3}{*}{	\textbf{\textit{	mu	}}}	&	\multirow{3}{*}{	Qual	}	&	\multirow{3}{*}{	absence, without	}	&	\multirow{3}{*}{	Sinitic	}	&	\multirow{	3	}{*}{	\textit{		}		無		}	\\&&&&				\textit{		}					\\&&&&	\textit{		}					\\\arrayrulecolor{gray} \hline
\multirow{3}{*}{	\textbf{\textit{	mula	}}}	&	\multirow{3}{*}{	Qual	}	&	\multirow{3}{*}{	cheap	}	&	\multirow{3}{*}{	Austronesian	}	&	\multirow{	2	}{*}{	\textit{	ms/in	 - }		murah		}	\\&&&&	\multirow{	2	}{*}{	\textit{	tg	 - }		mura		}	\\&&&&	\textit{		}					\\\arrayrulecolor{gray} \hline
\multirow{3}{*}{	\textbf{\textit{	muley	}}}	&	\multirow{3}{*}{	O/C	}	&	\multirow{3}{*}{	group, crowd	}	&	\multirow{3}{*}{	Koreo-Japonic	}	&	\multirow{	2	}{*}{	\textit{	ko	 - }		무리		}	\\&&&&	\multirow{	2	}{*}{	\textit{	ja	 - }		むれ		}	\\&&&&	\textit{		}					\\\arrayrulecolor{gray} \hline
\multirow{3}{*}{	\textbf{\textit{	mun	}}}	&	\multirow{3}{*}{	O/C	}	&	\multirow{3}{*}{	sentence	}	&	\multirow{3}{*}{	Sinitic	}	&	\multirow{	3	}{*}{	\textit{		}		文		}	\\&&&&				\textit{		}					\\&&&&	\textit{		}					\\\arrayrulecolor{gray} \hline
\multirow{3}{*}{	\textbf{\textit{	munbap	}}}	&	\multirow{3}{*}{	O/C	}	&	\multirow{3}{*}{	grammar	}	&	\multirow{3}{*}{	Sinitic	}	&	\multirow{	3	}{*}{	\textit{		}		文法		}	\\&&&&				\textit{		}					\\&&&&	\textit{		}					\\\arrayrulecolor{gray} \hline
\multirow{3}{*}{	\textbf{\textit{	mundey	}}}	&	\multirow{3}{*}{	O/C	}	&	\multirow{3}{*}{	problem, issue	}	&	\multirow{3}{*}{	Sinitic	}	&	\multirow{	3	}{*}{	\textit{		}		問題		}	\\&&&&				\textit{		}					\\&&&&	\textit{		}					\\\arrayrulecolor{gray} \hline
\multirow{3}{*}{	\textbf{\textit{	munhak	}}}	&	\multirow{3}{*}{	O/C	}	&	\multirow{3}{*}{	Literature	}	&	\multirow{3}{*}{	Sinitic	}	&	\multirow{	3	}{*}{	\textit{		}		文學		}	\\&&&&				\textit{		}					\\&&&&	\textit{		}					\\\arrayrulecolor{gray} \hline
\multirow{3}{*}{	\textbf{\textit{	munji	}}}	&	\multirow{3}{*}{	O/C	}	&	\multirow{3}{*}{	letters, script, writing system, morpheme	}	&	\multirow{3}{*}{	Sinitic	}	&	\multirow{	3	}{*}{	\textit{		}		文字		}	\\&&&&				\textit{		}					\\&&&&	\textit{		}					\\\arrayrulecolor{gray} \hline
\multirow{3}{*}{	\textbf{\textit{	mway'u	}}}	&	\multirow{3}{*}{	O/C	}	&	\multirow{3}{*}{	monsoon	}	&	\multirow{3}{*}{	Sinitic	}	&	\multirow{	3	}{*}{	\textit{		}		梅雨		}	\\&&&&				\textit{		}					\\&&&&	\textit{		}					\\\arrayrulecolor{gray} \hline
\multirow{3}{*}{	\textbf{\textit{	mwisu	}}}	&	\multirow{3}{*}{	O/C	}	&	\multirow{3}{*}{	water	}	&	\multirow{3}{*}{	Koreo-Japonic	}	&	\multirow{	2	}{*}{	\textit{	ko	 - }		물		}	\\&&&&	\multirow{	2	}{*}{	\textit{	ja	 - }		みず		}	\\&&&&	\textit{		}					\\\arrayrulecolor{gray} \hline
\multirow{3}{*}{	\textbf{\textit{	mwisuboti	}}}	&	\multirow{3}{*}{	O/C	}	&	\multirow{3}{*}{	Mercury (planet)	}	&	\multirow{3}{*}{	Compound	}	&	\multirow{	3	}{*}{	\textit{		}		mwisu +boti		}	\\&&&&				\textit{		}					\\&&&&	\textit{		}					\\\arrayrulecolor{gray} \hline
\multirow{3}{*}{	\textbf{\textit{	mwisuhali	}}}	&	\multirow{3}{*}{	O/C	}	&	\multirow{3}{*}{	wednesday	}	&	\multirow{3}{*}{	Compound	}	&	\multirow{	3	}{*}{	\textit{		}		mwisu +hali		}	\\&&&&				\textit{		}					\\&&&&	\textit{		}					\\\arrayrulecolor{gray} \hline
\multirow{3}{*}{	\textbf{\textit{	myaw	}}}	&	\multirow{3}{*}{	O/C	}	&	\multirow{3}{*}{	cat, kitty	}	&	\multirow{3}{*}{	Compound	}	&	\multirow{	3	}{*}{	\textit{		}				}	\\&&&&				\textit{		}					\\&&&&	\textit{		}					\\\arrayrulecolor{gray} \hline
\multirow{3}{*}{	\textbf{\textit{	naga	}}}	&	\multirow{3}{*}{	O/C	}	&	\multirow{3}{*}{	naga, dragon	}	&	\multirow{3}{*}{	Sanskrit	}	&	\multirow{	2	}{*}{	\textit{		}	\textsanskrit{	नाग 	}	}	\\&&&&	\multirow{	2	}{*}{	\textit{		}		(nāgá)		}	\\&&&&	\textit{		}					\\\arrayrulecolor{gray} \hline
\multirow{3}{*}{	\textbf{\textit{	nagan	}}}	&	\multirow{3}{*}{	O/C	}	&	\multirow{3}{*}{	city	}	&	\multirow{3}{*}{	Sanskrit	}	&	\multirow{	2	}{*}{	\textit{		}	\textsanskrit{	नगर 	}	}	\\&&&&	\multirow{	2	}{*}{	\textit{		}		(nágara)		}	\\&&&&	\textit{		}					\\\arrayrulecolor{gray} \hline
\multirow{3}{*}{	\textbf{\textit{	nakyat	}}}	&	\multirow{3}{*}{	Qual	}	&	\multirow{3}{*}{	ugly	}	&	\multirow{3}{*}{	Austroasiatic	}	&	\multirow{	2	}{*}{	\textit{	th	 - }	\textthai{	น่าเกลียด	}	}	\\&&&&	\multirow{	2	}{*}{	\textit{	lo	 - }	\textlao{	ຫນ້າກຽດ 	}	}	\\&&&&	\textit{		}					\\\arrayrulecolor{gray} \hline
\multirow{3}{*}{	\textbf{\textit{	nala	}}}	&	\multirow{3}{*}{	O/C	}	&	\multirow{3}{*}{	man; male	}	&	\multirow{3}{*}{	Sanskrit	}	&	\multirow{	2	}{*}{	\textit{		}	\textsanskrit{	नर 	}	}	\\&&&&	\multirow{	2	}{*}{	\textit{		}		(nára)		}	\\&&&&	\textit{		}					\\\arrayrulecolor{gray} \hline
\multirow{3}{*}{	\textbf{\textit{	namsek	}}}	&	\multirow{3}{*}{	O/C	}	&	\multirow{3}{*}{	gay, male homosexuality	}	&	\multirow{3}{*}{	Sinitic	}	&	\multirow{	3	}{*}{	\textit{		}		男色		}	\\&&&&				\textit{		}					\\&&&&	\textit{		}					\\\arrayrulecolor{gray} \hline
\multirow{3}{*}{	\textbf{\textit{	namta	}}}	&	\multirow{3}{*}{	Action	}	&	\multirow{3}{*}{	cry, weep, tears	}	&	\multirow{3}{*}{	Austroasiatic	}	&				\textit{	th	 - }	\textthai{	น้ำตา	}		\\&&&&				\textit{	lo	 - }	\textlao{	ນ້ຳ ຕາ 	}		\\&&&&	\textit{	km	 - }	\textkhmer{ តា 	}		\\\arrayrulecolor{gray} \hline
\multirow{3}{*}{	\textbf{\textit{	nangay	}}}	&	\multirow{3}{*}{	Qual	}	&	\multirow{3}{*}{	difficult, hard (task), tough	}	&	\multirow{3}{*}{	Sinitic	}	&	\multirow{	3	}{*}{	\textit{		}		難解		}	\\&&&&				\textit{		}					\\&&&&	\textit{		}					\\\arrayrulecolor{gray} \hline
\multirow{3}{*}{	\textbf{\textit{	nay	}}}	&	\multirow{3}{*}{	Qual	}	&	\multirow{3}{*}{	inside	}	&	\multirow{3}{*}{	Sinitic	}	&	\multirow{	3	}{*}{	\textit{		}		內		}	\\&&&&				\textit{		}					\\&&&&	\textit{		}					\\\arrayrulecolor{gray} \hline
\multirow{3}{*}{	\textbf{\textit{	nehoy	}}}	&	\multirow{3}{*}{	Qual	}	&	\multirow{3}{*}{	scent, smell	}	&	\multirow{3}{*}{	Koreo-Japonic	}	&	\multirow{	2	}{*}{	\textit{	ko	 - }		냄새		}	\\&&&&	\multirow{	2	}{*}{	\textit{	ja	 - }		におい		}	\\&&&&	\textit{		}					\\\arrayrulecolor{gray} \hline
\multirow{3}{*}{	\textbf{\textit{	nemjak	}}}	&	\multirow{3}{*}{	Action	}	&	\multirow{3}{*}{	sticky, (ling) Agglutinative	}	&	\multirow{3}{*}{	Sinitic	}	&	\multirow{	3	}{*}{	\textit{		}		粘着		}	\\&&&&				\textit{		}					\\&&&&	\textit{		}					\\\arrayrulecolor{gray} \hline
\multirow{3}{*}{	\textbf{\textit{	nenglik	}}}	&	\multirow{3}{*}{	O/C	}	&	\multirow{3}{*}{	ability, talent	}	&	\multirow{3}{*}{	Sinitic	}	&	\multirow{	3	}{*}{	\textit{		}		能力		}	\\&&&&				\textit{		}					\\&&&&	\textit{		}					\\\arrayrulecolor{gray} \hline
\multirow{3}{*}{	\textbf{\textit{	ni	}}}	&	\multirow{3}{*}{	O/C	}	&	\multirow{3}{*}{	two	}	&	\multirow{3}{*}{	Sinitic	}	&	\multirow{	3	}{*}{	\textit{		}		二		}	\\&&&&				\textit{		}					\\&&&&	\textit{		}					\\\arrayrulecolor{gray} \hline
\multirow{3}{*}{	\textbf{\textit{	nifa	}}}	&	\multirow{3}{*}{	O/C	}	&	\multirow{3}{*}{	teeth	}	&	\multirow{3}{*}{	Koreo-Japonic	}	&	\multirow{	2	}{*}{	\textit{	ko	 - }		이빨		}	\\&&&&	\multirow{	2	}{*}{	\textit{	ja	 - }		は		}	\\&&&&	\textit{		}					\\\arrayrulecolor{gray} \hline
\multirow{3}{*}{	\textbf{\textit{	nimsin	}}}	&	\multirow{3}{*}{	O/C	}	&	\multirow{3}{*}{	pregnancy, conception	}	&	\multirow{3}{*}{	Sinitic	}	&	\multirow{	3	}{*}{	\textit{		}		妊娠		}	\\&&&&				\textit{		}					\\&&&&	\textit{		}					\\\arrayrulecolor{gray} \hline
\multirow{3}{*}{	\textbf{\textit{	ninsik	}}}	&	\multirow{3}{*}{	Action	}	&	\multirow{3}{*}{	recognize	}	&	\multirow{3}{*}{	Sinitic	}	&	\multirow{	3	}{*}{	\textit{		}		認識		}	\\&&&&				\textit{		}					\\&&&&	\textit{		}					\\\arrayrulecolor{gray} \hline
\multirow{3}{*}{	\textbf{\textit{	nipkow	}}}	&	\multirow{3}{*}{	O/C	}	&	\multirow{3}{*}{	enterance	}	&	\multirow{3}{*}{		}	&	\multirow{	3	}{*}{	\textit{		}				}	\\&&&&				\textit{		}					\\&&&&	\textit{		}					\\\arrayrulecolor{gray} \hline
\multirow{3}{*}{	\textbf{\textit{	nitbon'o	}}}	&	\multirow{3}{*}{	O/C	}	&	\multirow{3}{*}{	japanese (language)	}	&	\multirow{3}{*}{	Sinitic	}	&	\multirow{	3	}{*}{	\textit{		}		日本語		}	\\&&&&				\textit{		}					\\&&&&	\textit{		}					\\\arrayrulecolor{gray} \hline
\multirow{3}{*}{	\textbf{\textit{	nitcut	}}}	&	\multirow{3}{*}{	Action	}	&	\multirow{3}{*}{	sunrise	}	&	\multirow{3}{*}{	Sinitic	}	&	\multirow{	3	}{*}{	\textit{		}		日出		}	\\&&&&				\textit{		}					\\&&&&	\textit{		}					\\\arrayrulecolor{gray} \hline
\multirow{3}{*}{	\textbf{\textit{	nitjihali	}}}	&	\multirow{3}{*}{	O/C	}	&	\multirow{3}{*}{	sunday	}	&	\multirow{3}{*}{	Compound	}	&	\multirow{	3	}{*}{	\textit{		}		nit + hali		}	\\&&&&				\textit{		}					\\&&&&	\textit{		}					\\\arrayrulecolor{gray} \hline
\multirow{3}{*}{	\textbf{\textit{	nitsyang	}}}	&	\multirow{3}{*}{	Qual	}	&	\multirow{3}{*}{	daily life, ordinary	}	&	\multirow{3}{*}{	Sinitic	}	&	\multirow{	3	}{*}{	\textit{		}		日常		}	\\&&&&				\textit{		}					\\&&&&	\textit{		}					\\\arrayrulecolor{gray} \hline
\multirow{3}{*}{	\textbf{\textit{	numpa	}}}	&	\multirow{3}{*}{	O/C	}	&	\multirow{3}{*}{	pond; swamp	}	&	\multirow{3}{*}{	Koreo-Japonic	}	&	\multirow{	2	}{*}{	\textit{	ko	 - }		늪		}	\\&&&&	\multirow{	2	}{*}{	\textit{	ja	 - }		ぬま		}	\\&&&&	\textit{		}					\\\arrayrulecolor{gray} \hline
\multirow{3}{*}{	\textbf{\textit{	nyatun	}}}	&	\multirow{3}{*}{	O/C	}	&	\multirow{3}{*}{	summer	}	&	\multirow{3}{*}{	Koreo-Japonic	}	&	\multirow{	2	}{*}{	\textit{	ko	 - }		여름		}	\\&&&&	\multirow{	2	}{*}{	\textit{	ja	 - }		なつ		}	\\&&&&	\textit{		}					\\\arrayrulecolor{gray} \hline
\multirow{3}{*}{	\textbf{\textit{	o	}}}	&	\multirow{3}{*}{	O/C	}	&	\multirow{3}{*}{	five	}	&	\multirow{3}{*}{	Sinitic	}	&	\multirow{	3	}{*}{	\textit{		}		五		}	\\&&&&				\textit{		}					\\&&&&	\textit{		}					\\\arrayrulecolor{gray} \hline
\multirow{3}{*}{	\textbf{\textit{	ogao	}}}	&	\multirow{3}{*}{	O/C	}	&	\multirow{3}{*}{	face	}	&	\multirow{3}{*}{	Koreo-Japonic	}	&	\multirow{	2	}{*}{	\textit{	ko	 - }		얼굴		}	\\&&&&	\multirow{	2	}{*}{	\textit{	ja	 - }		がお		}	\\&&&&	\textit{		}					\\\arrayrulecolor{gray} \hline
\multirow{3}{*}{	\textbf{\textit{	oji	}}}	&	\multirow{3}{*}{	O/C	}	&	\multirow{3}{*}{	sir	}	&	\multirow{3}{*}{	Koreo-Japonic	}	&	\multirow{	2	}{*}{	\textit{	ko	 - }		아쩌씨		}	\\&&&&	\multirow{	2	}{*}{	\textit{	ja	 - }		おじさん		}	\\&&&&	\textit{		}					\\\arrayrulecolor{gray} \hline
\multirow{3}{*}{	\textbf{\textit{	olasi	}}}	&	\multirow{3}{*}{	O/C	}	&	\multirow{3}{*}{	snake	}	&	\multirow{3}{*}{	Austronesian	}	&	\multirow{	2	}{*}{	\textit{	ms/id	 - }		ular		}	\\&&&&	\multirow{	2	}{*}{	\textit{	tl	 - }		ahas		}	\\&&&&	\textit{		}					\\\arrayrulecolor{gray} \hline
\multirow{3}{*}{	\textbf{\textit{	omay	}}}	&	\multirow{3}{*}{	O/C	}	&	\multirow{3}{*}{	ma'am	}	&	\multirow{3}{*}{	Koreo-Japonic	}	&	\multirow{	2	}{*}{	\textit{	ko	 - }		어머니		}	\\&&&&	\multirow{	2	}{*}{	\textit{	ja	 - }		おばさん		}	\\&&&&	\textit{		}					\\\arrayrulecolor{gray} \hline
\multirow{3}{*}{	\textbf{\textit{	onlain	}}}	&	\multirow{3}{*}{	Qual	}	&	\multirow{3}{*}{	online	}	&	\multirow{3}{*}{	Western: English	}	&	\multirow{	3	}{*}{	\textit{	en	 - }		online		}	\\&&&&				\textit{		}					\\&&&&	\textit{		}					\\\arrayrulecolor{gray} \hline
\multirow{3}{*}{	\textbf{\textit{	otaw	}}}	&	\multirow{3}{*}{	Action	}	&	\multirow{3}{*}{	sing, song	}	&	\multirow{3}{*}{	Koreo-Japonic	}	&	\multirow{	2	}{*}{	\textit{	ko	 - }		부르다		}	\\&&&&	\multirow{	2	}{*}{	\textit{	ja	 - }		うた		}	\\&&&&	\textit{		}					\\\arrayrulecolor{gray} \hline
\multirow{3}{*}{	\textbf{\textit{	otaw	}}}	&	\multirow{3}{*}{	O/C	}	&	\multirow{3}{*}{	sing, song	}	&	\multirow{3}{*}{	Koreo-Japonic	}	&	\multirow{	2	}{*}{	\textit{	ko	 - }		부르다		}	\\&&&&	\multirow{	2	}{*}{	\textit{	ja	 - }		うた		}	\\&&&&	\textit{		}					\\\arrayrulecolor{gray} \hline
\multirow{3}{*}{	\textbf{\textit{	oton	}}}	&	\multirow{3}{*}{	O/C	}	&	\multirow{3}{*}{	adult	}	&	\multirow{3}{*}{	Koreo-Japonic	}	&	\multirow{	2	}{*}{	\textit{	ko	 - }		어른		}	\\&&&&	\multirow{	2	}{*}{	\textit{	ja	 - }		おとな		}	\\&&&&	\textit{		}					\\\arrayrulecolor{gray} \hline
\multirow{3}{*}{	\textbf{\textit{	pa'ok (fway)	}}}	&	\multirow{3}{*}{	Action	}	&	\multirow{3}{*}{	stir-fry	}	&	\multirow{3}{*}{	Compound	}	&	\multirow{	2	}{*}{	\textit{	th	 - }	\textthai{	ผัด	}	}	\\&&&&	\multirow{	2	}{*}{	\textit{	kr	 - }		볶다		}	\\&&&&	\textit{		}					\\\arrayrulecolor{gray} \hline
\multirow{3}{*}{	\textbf{\textit{	pagway	}}}	&	\multirow{3}{*}{	Action	}	&	\multirow{3}{*}{	destroy, break	}	&	\multirow{3}{*}{	Sinitic	}	&	\multirow{	3	}{*}{	\textit{		}		破壞		}	\\&&&&				\textit{		}					\\&&&&	\textit{		}					\\\arrayrulecolor{gray} \hline
\multirow{3}{*}{	\textbf{\textit{	pak	}}}	&	\multirow{3}{*}{	O/C	}	&	\multirow{3}{*}{	mouth	}	&	\multirow{3}{*}{	Austroasiatic	}	&	\multirow{	2	}{*}{	\textit{	th	 - }	\textthai{	ปาก	}	}	\\&&&&	\multirow{	2	}{*}{	\textit{	lo	 - }	\textlao{	ປາກ	}	}	\\&&&&	\textit{		}					\\\arrayrulecolor{gray} \hline
\multirow{3}{*}{	\textbf{\textit{	pal	}}}	&	\multirow{3}{*}{	O/C	}	&	\multirow{3}{*}{	fruit	}	&	\multirow{3}{*}{	Sanskrit	}	&	\multirow{	2	}{*}{	\textit{		}	\textsanskrit{	फल 	}	}	\\&&&&	\multirow{	2	}{*}{	\textit{		}		(phála)		}	\\&&&&	\textit{		}					\\\arrayrulecolor{gray} \hline
\multirow{3}{*}{	\textbf{\textit{	palu	}}}	&	\multirow{3}{*}{	O/C	}	&	\multirow{3}{*}{	hammer	}	&	\multirow{3}{*}{	Austronesian	}	&	\multirow{	3	}{*}{	\textit{	ms/id	 - }		palu		}	\\&&&&				\textit{		}					\\&&&&	\textit{		}					\\\arrayrulecolor{gray} \hline
\multirow{3}{*}{	\textbf{\textit{	pang	}}}	&	\multirow{3}{*}{	O/C	}	&	\multirow{3}{*}{	bread	}	&	\multirow{3}{*}{	Western: Portuguese	}	&	\multirow{	3	}{*}{	\textit{	pt	 - }		paõ		}	\\&&&&				\textit{		}					\\&&&&	\textit{		}					\\\arrayrulecolor{gray} \hline
\multirow{3}{*}{	\textbf{\textit{	pantalon	}}}	&	\multirow{3}{*}{	O/C	}	&	\multirow{3}{*}{	pants	}	&	\multirow{3}{*}{	Western: Spanish	}	&	\multirow{	2	}{*}{	\textit{	es	 - }		pantalones		}	\\&&&&	\multirow{	2	}{*}{	\textit{		}		(indirect loan via tl)		}	\\&&&&	\textit{		}					\\\arrayrulecolor{gray} \hline
\multirow{3}{*}{	\textbf{\textit{	pasawa	}}}	&	\multirow{3}{*}{	O/C	}	&	\multirow{3}{*}{	urine; pee	}	&	\multirow{3}{*}{	Sanskrit	}	&	\multirow{	3	}{*}{	\textit{		}	\textsanskrit{		}	}	\\&&&&				\textit{		}					\\&&&&	\textit{		}					\\\arrayrulecolor{gray} \hline
\multirow{3}{*}{	\textbf{\textit{	pating	}}}	&	\multirow{3}{*}{	O/C	}	&	\multirow{3}{*}{	Shark	}	&	\multirow{3}{*}{	Austronesian	}	&	\multirow{	3	}{*}{	\textit{	tl	 - }		pating		}	\\&&&&				\textit{		}					\\&&&&	\textit{		}					\\\arrayrulecolor{gray} \hline
\multirow{3}{*}{	\textbf{\textit{	pedan	}}}	&	\multirow{3}{*}{	O/C	}	&	\multirow{3}{*}{	ceiling	}	&	\multirow{3}{*}{	Austroasiatic	}	&				\textit{	th	 - }	\textthai{	เพดาน	}		\\&&&&				\textit{	lo	 - }	\textlao{	ເພດານ 	}		\\&&&&	\textit{	km	 - }	\textkhmer{	ពិតាន 	}		\\\arrayrulecolor{gray} \hline
\multirow{3}{*}{	\textbf{\textit{	penyek	}}}	&	\multirow{3}{*}{	Action	}	&	\multirow{3}{*}{	translate	}	&	\multirow{3}{*}{	Sinitic	}	&	\multirow{	3	}{*}{	\textit{		}		飜譯		}	\\&&&&				\textit{		}					\\&&&&	\textit{		}					\\\arrayrulecolor{gray} \hline
\multirow{3}{*}{	\textbf{\textit{	penyek-ko	}}}	&	\multirow{3}{*}{	O/C	}	&	\multirow{3}{*}{	the translation	}	&	\multirow{3}{*}{	Compound	}	&	\multirow{	3	}{*}{	\textit{		}				}	\\&&&&				\textit{		}					\\&&&&	\textit{		}					\\\arrayrulecolor{gray} \hline
\multirow{3}{*}{	\textbf{\textit{	penyu	}}}	&	\multirow{3}{*}{	O/C	}	&	\multirow{3}{*}{	turtles	}	&	\multirow{3}{*}{	Austronesian	}	&	\multirow{	3	}{*}{	\textit{	ms/id	 - }		penyu		}	\\&&&&				\textit{		}					\\&&&&	\textit{		}					\\\arrayrulecolor{gray} \hline
\multirow{3}{*}{	\textbf{\textit{	pilo	}}}	&	\multirow{3}{*}{	O/C	}	&	\multirow{3}{*}{	pillow	}	&	\multirow{3}{*}{	Western: English	}	&	\multirow{	3	}{*}{	\textit{	en	 - }		pillow		}	\\&&&&				\textit{		}					\\&&&&	\textit{		}					\\\arrayrulecolor{gray} \hline
\multirow{3}{*}{	\textbf{\textit{	pintu	}}}	&	\multirow{3}{*}{	O/C	}	&	\multirow{3}{*}{	door	}	&	\multirow{3}{*}{	Austronesian	}	&	\multirow{	3	}{*}{	\textit{	ms/id	 - }		pintu		}	\\&&&&				\textit{		}					\\&&&&	\textit{		}					\\\arrayrulecolor{gray} \hline
\multirow{3}{*}{	\textbf{\textit{	pipal	}}}	&	\multirow{3}{*}{	O/C	}	&	\multirow{3}{*}{	pepper	}	&	\multirow{3}{*}{	Sanskrit	}	&	\multirow{	2	}{*}{	\textit{		}	\textsanskrit{	पिप्पलि 	}	}	\\&&&&	\multirow{	2	}{*}{	\textit{		}		(pippali)		}	\\&&&&	\textit{		}					\\\arrayrulecolor{gray} \hline
\multirow{3}{*}{	\textbf{\textit{	pisaw	}}}	&	\multirow{3}{*}{	O/C	}	&	\multirow{3}{*}{	knife, dagger, sharp object	}	&	\multirow{3}{*}{	Austronesian	}	&	\multirow{	3	}{*}{	\textit{	ms/id	 - }		pisau		}	\\&&&&				\textit{		}					\\&&&&	\textit{		}					\\\arrayrulecolor{gray} \hline
\multirow{3}{*}{	\textbf{\textit{	pom	}}}	&	\multirow{3}{*}{	O/C	}	&	\multirow{3}{*}{	hair	}	&	\multirow{3}{*}{	Austroasiatic	}	&	\multirow{	2	}{*}{	\textit{	th	 - }	\textthai{	ผม	}	}	\\&&&&	\multirow{	2	}{*}{	\textit{	lo	 - }	\textlao{	ຜົມ 	}	}	\\&&&&	\textit{		}					\\\arrayrulecolor{gray} \hline
\multirow{3}{*}{	\textbf{\textit{	pon	}}}	&	\multirow{3}{*}{	O/C	}	&	\multirow{3}{*}{	tree	}	&	\multirow{3}{*}{	Austronesian	}	&	\multirow{	2	}{*}{	\textit{	ms/id	 - }		pohon		}	\\&&&&	\multirow{	2	}{*}{	\textit{	tl	 - }		puno		}	\\&&&&	\textit{		}					\\\arrayrulecolor{gray} \hline
\multirow{3}{*}{	\textbf{\textit{	ponjiboti	}}}	&	\multirow{3}{*}{	O/C	}	&	\multirow{3}{*}{	Jupiter	}	&	\multirow{3}{*}{	Compound	}	&	\multirow{	3	}{*}{	\textit{		}		pon + boti		}	\\&&&&				\textit{		}					\\&&&&	\textit{		}					\\\arrayrulecolor{gray} \hline
\multirow{3}{*}{	\textbf{\textit{	ponjihali	}}}	&	\multirow{3}{*}{	O/C	}	&	\multirow{3}{*}{	thursday	}	&	\multirow{3}{*}{	Compound	}	&	\multirow{	3	}{*}{	\textit{		}		pon + hali		}	\\&&&&				\textit{		}					\\&&&&	\textit{		}					\\\arrayrulecolor{gray} \hline
\multirow{3}{*}{	\textbf{\textit{	pontiku	}}}	&	\multirow{3}{*}{	O/C	}	&	\multirow{3}{*}{	squirrel	}	&	\multirow{3}{*}{	Austronesian	}	&	\multirow{	3	}{*}{	\textit{		}				}	\\&&&&				\textit{		}					\\&&&&	\textit{		}					\\\arrayrulecolor{gray} \hline
\multirow{3}{*}{	\textbf{\textit{	puba	}}}	&	\multirow{3}{*}{	Qual	}	&	\multirow{3}{*}{	east	}	&	\multirow{3}{*}{	Sanskrit	}	&	\multirow{	2	}{*}{	\textit{		}	\textsanskrit{	पूर्व 	}	}	\\&&&&	\multirow{	2	}{*}{	\textit{		}		(pū́rva)		}	\\&&&&	\textit{		}					\\\arrayrulecolor{gray} \hline
\multirow{3}{*}{	\textbf{\textit{	puyuk	}}}	&	\multirow{3}{*}{	Action	}	&	\multirow{3}{*}{	raise, grow	}	&	\multirow{3}{*}{	Sinitic	}	&	\multirow{	3	}{*}{	\textit{		}		撫育		}	\\&&&&				\textit{		}					\\&&&&	\textit{		}					\\\arrayrulecolor{gray} \hline
\multirow{3}{*}{	\textbf{\textit{	saba	}}}	&	\multirow{3}{*}{	O/C	}	&	\multirow{3}{*}{	server	}	&	\multirow{3}{*}{	Western: English	}	&	\multirow{	3	}{*}{	\textit{	en	 - }		server		}	\\&&&&				\textit{		}					\\&&&&	\textit{		}					\\\arrayrulecolor{gray} \hline
\multirow{3}{*}{	\textbf{\textit{	saci	}}}	&	\multirow{3}{*}{	O/C	}	&	\multirow{3}{*}{	woman; female	}	&	\multirow{3}{*}{	Sanskrit	}	&	\multirow{	2	}{*}{	\textit{		}	\textsanskrit{	स्त्री 	}	}	\\&&&&	\multirow{	2	}{*}{	\textit{		}		(strī́)		}	\\&&&&	\textit{		}					\\\arrayrulecolor{gray} \hline
\multirow{3}{*}{	\textbf{\textit{	sacilaja	}}}	&	\multirow{3}{*}{	O/C	}	&	\multirow{3}{*}{	queen (female ruler)	}	&	\multirow{3}{*}{	Sanskrit	}	&	\multirow{	2	}{*}{	\textit{		}	\textsanskrit{	स्त्री + राजन्	}	}	\\&&&&	\multirow{	2	}{*}{	\textit{		}		(strī́ +rā́jan)		}	\\&&&&	\textit{		}					\\\arrayrulecolor{gray} \hline
\multirow{3}{*}{	\textbf{\textit{	sadang	}}}	&	\multirow{3}{*}{	O/C	}	&	\multirow{3}{*}{	brown sugar	}	&	\multirow{3}{*}{	Sinitic	}	&	\multirow{	3	}{*}{	\textit{		}		沙糖		}	\\&&&&				\textit{		}					\\&&&&	\textit{		}					\\\arrayrulecolor{gray} \hline
\multirow{3}{*}{	\textbf{\textit{	sala	}}}	&	\multirow{3}{*}{	Qual	}	&	\multirow{3}{*}{	no, false, incorrect	}	&	\multirow{3}{*}{	Austronesian	}	&	\multirow{	2	}{*}{	\textit{	ms/id	 - }		salah		}	\\&&&&	\multirow{	2	}{*}{	\textit{	tg	 - }		sala		}	\\&&&&	\textit{		}					\\\arrayrulecolor{gray} \hline
\multirow{3}{*}{	\textbf{\textit{	sam	}}}	&	\multirow{3}{*}{	O/C	}	&	\multirow{3}{*}{	three	}	&	\multirow{3}{*}{	Sinitic	}	&	\multirow{	3	}{*}{	\textit{		}		三		}	\\&&&&				\textit{		}					\\&&&&	\textit{		}					\\\arrayrulecolor{gray} \hline
\multirow{3}{*}{	\textbf{\textit{	sampa	}}}	&	\multirow{3}{*}{	O/C	}	&	\multirow{3}{*}{	trash, garbage	}	&	\multirow{3}{*}{	Austronesian	}	&	\multirow{	3	}{*}{	\textit{	ms/id	 - }		sampah		}	\\&&&&				\textit{		}					\\&&&&	\textit{		}					\\\arrayrulecolor{gray} \hline
\multirow{3}{*}{	\textbf{\textit{	sanbo	}}}	&	\multirow{3}{*}{	Action	}	&	\multirow{3}{*}{	walk	}	&	\multirow{3}{*}{	Sinitic	}	&	\multirow{	3	}{*}{	\textit{		}		散步		}	\\&&&&				\textit{		}					\\&&&&	\textit{		}					\\\arrayrulecolor{gray} \hline
\multirow{3}{*}{	\textbf{\textit{	sap	}}}	&	\multirow{3}{*}{	O/C	}	&	\multirow{3}{*}{	word; to talk; to claim; voice	}	&	\multirow{3}{*}{	Sanskrit	}	&	\multirow{	2	}{*}{	\textit{		}	\textsanskrit{	शब्द 	}	}	\\&&&&	\multirow{	2	}{*}{	\textit{		}		(śábda)		}	\\&&&&	\textit{		}					\\\arrayrulecolor{gray} \hline
\multirow{3}{*}{	\textbf{\textit{	sapan	}}}	&	\multirow{3}{*}{	O/C	}	&	\multirow{3}{*}{	bridge	}	&	\multirow{3}{*}{	Austroasiatic	}	&				\textit{	th	 - }	\textthai{	สะพาน	}		\\&&&&				\textit{	km	 - }	\textkhmer{	ស្ពាន	}		\\&&&&	\textit{	lo	 - }	\textlao{	ສະພານ	}		\\\arrayrulecolor{gray} \hline
\multirow{3}{*}{	\textbf{\textit{	sata	}}}	&	\multirow{3}{*}{	Action	}	&	\multirow{3}{*}{	to know	}	&	\multirow{3}{*}{	Sanskrit	}	&	\multirow{	2	}{*}{	\textit{		}	\textsanskrit{	शास्त्र 	}	}	\\&&&&	\multirow{	2	}{*}{	\textit{		}		(śāstra)		}	\\&&&&	\textit{		}					\\\arrayrulecolor{gray} \hline
\multirow{3}{*}{	\textbf{\textit{	sata	}}}	&	\multirow{3}{*}{	O/C	}	&	\multirow{3}{*}{	knowledge	}	&	\multirow{3}{*}{	Sanskrit	}	&	\multirow{	2	}{*}{	\textit{		}	\textsanskrit{	शास्त्र 	}	}	\\&&&&	\multirow{	2	}{*}{	\textit{		}		(śāstra)		}	\\&&&&	\textit{		}					\\\arrayrulecolor{gray} \hline
\multirow{3}{*}{	\textbf{\textit{	satan	}}}	&	\multirow{3}{*}{	O/C	}	&	\multirow{3}{*}{	Satan; supreme evil	}	&	\multirow{3}{*}{	Western: English	}	&	\multirow{	3	}{*}{	\textit{	en	 - }		satan		}	\\&&&&				\textit{		}					\\&&&&	\textit{		}					\\\arrayrulecolor{gray} \hline
\multirow{3}{*}{	\textbf{\textit{	sathay	}}}	&	\multirow{3}{*}{	Action	}	&	\multirow{3}{*}{	kill, murder	}	&	\multirow{3}{*}{	Sinitic	}	&	\multirow{	3	}{*}{	\textit{		}		殺害		}	\\&&&&				\textit{		}					\\&&&&	\textit{		}					\\\arrayrulecolor{gray} \hline
\multirow{3}{*}{	\textbf{\textit{	satu	}}}	&	\multirow{3}{*}{	O/C	}	&	\multirow{3}{*}{	enemy	}	&	\multirow{3}{*}{	Sanskrit	}	&	\multirow{	2	}{*}{	\textit{		}	\textsanskrit{	शत्रु 	}	}	\\&&&&	\multirow{	2	}{*}{	\textit{		}		śátru		}	\\&&&&	\textit{		}					\\\arrayrulecolor{gray} \hline
\multirow{3}{*}{	\textbf{\textit{	sekyang	}}}	&	\multirow{3}{*}{	Action	}	&	\multirow{3}{*}{	sunset	}	&	\multirow{3}{*}{	Sinitic	}	&	\multirow{	3	}{*}{	\textit{		}		夕陽		}	\\&&&&				\textit{		}					\\&&&&	\textit{		}					\\\arrayrulecolor{gray} \hline
\multirow{3}{*}{	\textbf{\textit{	senang	}}}	&	\multirow{3}{*}{	Qual	}	&	\multirow{3}{*}{	happy	}	&	\multirow{3}{*}{	Austronesian	}	&	\multirow{	3	}{*}{	\textit{	ms/id	 - }		senang		}	\\&&&&				\textit{		}					\\&&&&	\textit{		}					\\\arrayrulecolor{gray} \hline
\multirow{3}{*}{	\textbf{\textit{	senghwa	}}}	&	\multirow{3}{*}{	Acion	}	&	\multirow{3}{*}{	become, turn into, becoming	}	&	\multirow{3}{*}{	Sinitic	}	&	\multirow{	3	}{*}{	\textit{		}		成化		}	\\&&&&				\textit{		}					\\&&&&	\textit{		}					\\\arrayrulecolor{gray} \hline
\multirow{3}{*}{	\textbf{\textit{	senghwat	}}}	&	\multirow{3}{*}{	Action	}	&	\multirow{3}{*}{	to live life	}	&	\multirow{3}{*}{	Sinitic	}	&	\multirow{	3	}{*}{	\textit{		}		生活		}	\\&&&&				\textit{		}					\\&&&&	\textit{		}					\\\arrayrulecolor{gray} \hline
\multirow{3}{*}{	\textbf{\textit{	sengmeng	}}}	&	\multirow{3}{*}{	O/C	}	&	\multirow{3}{*}{	life	}	&	\multirow{3}{*}{	Sinitic	}	&	\multirow{	3	}{*}{	\textit{		}		生命		}	\\&&&&				\textit{		}					\\&&&&	\textit{		}					\\\arrayrulecolor{gray} \hline
\multirow{3}{*}{	\textbf{\textit{	sengmeng yu	}}}	&	\multirow{3}{*}{	Action	}	&	\multirow{3}{*}{	to have life	}	&	\multirow{3}{*}{	Sinitic	}	&	\multirow{	3	}{*}{	\textit{		}		生命 有		}	\\&&&&				\textit{		}					\\&&&&	\textit{		}					\\\arrayrulecolor{gray} \hline
\multirow{3}{*}{	\textbf{\textit{	senseng	}}}	&	\multirow{3}{*}{	O/C	}	&	\multirow{3}{*}{	teacher	}	&	\multirow{3}{*}{	Sinitic	}	&	\multirow{	3	}{*}{	\textit{		}		先生		}	\\&&&&				\textit{		}					\\&&&&	\textit{		}					\\\arrayrulecolor{gray} \hline
\multirow{3}{*}{	\textbf{\textit{	setdang	}}}	&	\multirow{3}{*}{	O/C	}	&	\multirow{3}{*}{	white sugar	}	&	\multirow{3}{*}{	Sinitic	}	&	\multirow{	3	}{*}{	\textit{		}		雪糖		}	\\&&&&				\textit{		}					\\&&&&	\textit{		}					\\\arrayrulecolor{gray} \hline
\multirow{3}{*}{	\textbf{\textit{	seydek	}}}	&	\multirow{3}{*}{	Action	}	&	\multirow{3}{*}{	wash, clean	}	&	\multirow{3}{*}{	Sinitic	}	&	\multirow{	3	}{*}{	\textit{		}		洗滌		}	\\&&&&				\textit{		}					\\&&&&	\textit{		}					\\\arrayrulecolor{gray} \hline
\multirow{3}{*}{	\textbf{\textit{	si	}}}	&	\multirow{3}{*}{	O/C	}	&	\multirow{3}{*}{	four	}	&	\multirow{3}{*}{	Sinitic	}	&	\multirow{	3	}{*}{	\textit{		}		四		}	\\&&&&				\textit{		}					\\&&&&	\textit{		}					\\\arrayrulecolor{gray} \hline
\multirow{3}{*}{	\textbf{\textit{	siden	}}}	&	\multirow{3}{*}{	O/C	}	&	\multirow{3}{*}{	dictionary	}	&	\multirow{3}{*}{	Sinitic	}	&	\multirow{	3	}{*}{	\textit{		}		辭典		}	\\&&&&				\textit{		}					\\&&&&	\textit{		}					\\\arrayrulecolor{gray} \hline
\multirow{3}{*}{	\textbf{\textit{	sido	}}}	&	\multirow{3}{*}{	Action	}	&	\multirow{3}{*}{	try	}	&	\multirow{3}{*}{	Sinitic	}	&	\multirow{	3	}{*}{	\textit{		}		試圖		}	\\&&&&				\textit{		}					\\&&&&	\textit{		}					\\\arrayrulecolor{gray} \hline
\multirow{3}{*}{	\textbf{\textit{	sik	}}}	&	\multirow{3}{*}{	Action	}	&	\multirow{3}{*}{	eat, drink, swallow	}	&	\multirow{3}{*}{	Sinitic	}	&	\multirow{	3	}{*}{	\textit{		}		食		}	\\&&&&				\textit{		}					\\&&&&	\textit{		}					\\\arrayrulecolor{gray} \hline
\multirow{3}{*}{	\textbf{\textit{	sikaw	}}}	&	\multirow{3}{*}{	Action	}	&	\multirow{3}{*}{	think	}	&	\multirow{3}{*}{	Sinitic	}	&	\multirow{	3	}{*}{	\textit{		}		思考		}	\\&&&&				\textit{		}					\\&&&&	\textit{		}					\\\arrayrulecolor{gray} \hline
\multirow{3}{*}{	\textbf{\textit{	sikcit	}}}	&	\multirow{3}{*}{	Action	}	&	\multirow{3}{*}{	draw, paint	}	&	\multirow{3}{*}{	Sinitic	}	&	\multirow{	3	}{*}{	\textit{		}		色漆		}	\\&&&&				\textit{		}					\\&&&&	\textit{		}					\\\arrayrulecolor{gray} \hline
\multirow{3}{*}{	\textbf{\textit{	sikko	}}}	&	\multirow{3}{*}{	O/C	}	&	\multirow{3}{*}{	food, meal	}	&	\multirow{3}{*}{	Compound	}	&	\multirow{	3	}{*}{	\textit{		}				}	\\&&&&				\textit{		}					\\&&&&	\textit{		}					\\\arrayrulecolor{gray} \hline
\multirow{3}{*}{	\textbf{\textit{	sikmut	}}}	&	\multirow{3}{*}{	O/C	}	&	\multirow{3}{*}{	plant, vegetation	}	&	\multirow{3}{*}{	Sinitic	}	&	\multirow{	3	}{*}{	\textit{		}		植物		}	\\&&&&				\textit{		}					\\&&&&	\textit{		}					\\\arrayrulecolor{gray} \hline
\multirow{3}{*}{	\textbf{\textit{	simang	}}}	&	\multirow{3}{*}{	Action	}	&	\multirow{3}{*}{	die, death, pass away	}	&	\multirow{3}{*}{	Sinitic	}	&	\multirow{	3	}{*}{	\textit{		}		死亡		}	\\&&&&				\textit{		}					\\&&&&	\textit{		}					\\\arrayrulecolor{gray} \hline
\multirow{3}{*}{	\textbf{\textit{	simga	}}}	&	\multirow{3}{*}{	Action	}	&	\multirow{3}{*}{	join, participate	}	&	\multirow{3}{*}{	Sinitic	}	&	\multirow{	3	}{*}{	\textit{		}		參加		}	\\&&&&				\textit{		}					\\&&&&	\textit{		}					\\\arrayrulecolor{gray} \hline
\multirow{3}{*}{	\textbf{\textit{	sin	}}}	&	\multirow{3}{*}{	Qual	}	&	\multirow{3}{*}{	new	}	&	\multirow{3}{*}{	Sinitic	}	&	\multirow{	3	}{*}{	\textit{		}		新		}	\\&&&&				\textit{		}					\\&&&&	\textit{		}					\\\arrayrulecolor{gray} \hline
\multirow{3}{*}{	\textbf{\textit{	singa	}}}	&	\multirow{3}{*}{	O/C	}	&	\multirow{3}{*}{	lion	}	&	\multirow{3}{*}{	Sanskrit	}	&	\multirow{	2	}{*}{	\textit{		}	\textsanskrit{	सिंह 	}	}	\\&&&&	\multirow{	2	}{*}{	\textit{		}		(siṃhá)		}	\\&&&&	\textit{		}					\\\arrayrulecolor{gray} \hline
\multirow{3}{*}{	\textbf{\textit{	singli	}}}	&	\multirow{3}{*}{	Action	}	&	\multirow{3}{*}{	win	}	&	\multirow{3}{*}{	Sinitic	}	&	\multirow{	3	}{*}{	\textit{		}		勝利		}	\\&&&&				\textit{		}					\\&&&&	\textit{		}					\\\arrayrulecolor{gray} \hline
\multirow{3}{*}{	\textbf{\textit{	sip	}}}	&	\multirow{3}{*}{	O/C	}	&	\multirow{3}{*}{	ten	}	&	\multirow{3}{*}{	Sinitic	}	&	\multirow{	3	}{*}{	\textit{		}		十		}	\\&&&&				\textit{		}					\\&&&&	\textit{		}					\\\arrayrulecolor{gray} \hline
\multirow{3}{*}{	\textbf{\textit{	sisim	}}}	&	\multirow{3}{*}{	O/C	}	&	\multirow{3}{*}{	deer	}	&	\multirow{3}{*}{	Koreo-Japonic	}	&	\multirow{	2	}{*}{	\textit{	ko	 - }		사슴		}	\\&&&&	\multirow{	2	}{*}{	\textit{	ja	 - }		しし		}	\\&&&&	\textit{		}					\\\arrayrulecolor{gray} \hline
\multirow{3}{*}{	\textbf{\textit{	sitlik	}}}	&	\multirow{3}{*}{	O/C	}	&	\multirow{3}{*}{	skill, talent	}	&	\multirow{3}{*}{	Sinitic	}	&	\multirow{	3	}{*}{	\textit{		}		實力		}	\\&&&&				\textit{		}					\\&&&&	\textit{		}					\\\arrayrulecolor{gray} \hline
\multirow{3}{*}{	\textbf{\textit{	socek	}}}	&	\multirow{3}{*}{	O/C	}	&	\multirow{3}{*}{	book	}	&	\multirow{3}{*}{	Sinitic	}	&	\multirow{	3	}{*}{	\textit{		}		書冊		}	\\&&&&				\textit{		}					\\&&&&	\textit{		}					\\\arrayrulecolor{gray} \hline
\multirow{3}{*}{	\textbf{\textit{	sokosoko	}}}	&	\multirow{3}{*}{	Action	}	&	\multirow{3}{*}{	whisper, mumur	}	&	\multirow{3}{*}{	Sound-based	}	&	\multirow{	3	}{*}{	\textit{		}				}	\\&&&&				\textit{		}					\\&&&&	\textit{		}					\\\arrayrulecolor{gray} \hline
\multirow{3}{*}{	\textbf{\textit{	suna	}}}	&	\multirow{3}{*}{	O/C	}	&	\multirow{3}{*}{	sand	}	&	\multirow{3}{*}{	Koreo-Japonic	}	&	\multirow{	3	}{*}{	\textit{	ja	 - }				}	\\&&&&				\textit{		}					\\&&&&	\textit{		}					\\\arrayrulecolor{gray} \hline
\multirow{3}{*}{	\textbf{\textit{	supika	}}}	&	\multirow{3}{*}{	O/C	}	&	\multirow{3}{*}{	speaker (computing)	}	&	\multirow{3}{*}{	Western: English	}	&	\multirow{	3	}{*}{	\textit{	en	 - }		speaker		}	\\&&&&				\textit{		}					\\&&&&	\textit{		}					\\\arrayrulecolor{gray} \hline
\multirow{3}{*}{	\textbf{\textit{	susi	}}}	&	\multirow{3}{*}{	O/C	}	&	\multirow{3}{*}{	numbers	}	&	\multirow{3}{*}{	Sinitic	}	&	\multirow{	3	}{*}{	\textit{		}		數詞		}	\\&&&&				\textit{		}					\\&&&&	\textit{		}					\\\arrayrulecolor{gray} \hline
\multirow{3}{*}{	\textbf{\textit{	susuk	}}}	&	\multirow{3}{*}{	Action	}	&	\multirow{3}{*}{	brush, polish, scrub	}	&	\multirow{3}{*}{	Sound-based	}	&	\multirow{	3	}{*}{	\textit{		}				}	\\&&&&				\textit{		}					\\&&&&	\textit{		}					\\\arrayrulecolor{gray} \hline
\multirow{3}{*}{	\textbf{\textit{	swa, s'wa	}}}	&	\multirow{3}{*}{	Action	}	&	\multirow{3}{*}{	become, to be transformed	}	&	\multirow{3}{*}{	Sinitic	}	&	\multirow{	3	}{*}{	\textit{		}		成化		}	\\&&&&				\textit{		}					\\&&&&	\textit{		}					\\\arrayrulecolor{gray} \hline
\multirow{3}{*}{	\textbf{\textit{	swansu	}}}	&	\multirow{3}{*}{	Action	}	&	\multirow{3}{*}{	count, numbering	}	&	\multirow{3}{*}{	Sinitic	}	&	\multirow{	3	}{*}{	\textit{		}		計数		}	\\&&&&				\textit{		}					\\&&&&	\textit{		}					\\\arrayrulecolor{gray} \hline
\multirow{3}{*}{	\textbf{\textit{	swi	}}}	&	\multirow{3}{*}{	Qual	}	&	\multirow{3}{*}{	sour	}	&	\multirow{3}{*}{	Koreo-Japonic	}	&	\multirow{	2	}{*}{	\textit{	ko	 - }		신 < 쉰		}	\\&&&&	\multirow{	2	}{*}{	\textit{	ja	 - }		すい		}	\\&&&&	\textit{		}					\\\arrayrulecolor{gray} \hline
\multirow{3}{*}{	\textbf{\textit{	syang'ep	}}}	&	\multirow{3}{*}{	O/C	}	&	\multirow{3}{*}{	business; trade; commerce	}	&	\multirow{3}{*}{	Sinitic	}	&	\multirow{	3	}{*}{	\textit{		}		商業		}	\\&&&&				\textit{		}					\\&&&&	\textit{		}					\\\arrayrulecolor{gray} \hline
\multirow{3}{*}{	\textbf{\textit{	syangdey	}}}	&	\multirow{3}{*}{	O/C	}	&	\multirow{3}{*}{	God (monotheism); supreme good	}	&	\multirow{3}{*}{	Sinitic	}	&	\multirow{	3	}{*}{	\textit{		}		上帝		}	\\&&&&				\textit{		}					\\&&&&	\textit{		}					\\\arrayrulecolor{gray} \hline
\multirow{3}{*}{	\textbf{\textit{	syangjong	}}}	&	\multirow{3}{*}{	O/C	}	&	\multirow{3}{*}{	upstairs; upper class	}	&	\multirow{3}{*}{	Sinitic	}	&	\multirow{	3	}{*}{	\textit{		}		上層		}	\\&&&&				\textit{		}					\\&&&&	\textit{		}					\\\arrayrulecolor{gray} \hline
\multirow{3}{*}{	\textbf{\textit{	syangsing	}}}	&	\multirow{3}{*}{	Action	}	&	\multirow{3}{*}{	to lift	}	&	\multirow{3}{*}{	Sinitic	}	&	\multirow{	3	}{*}{	\textit{		}		上昇		}	\\&&&&				\textit{		}					\\&&&&	\textit{		}					\\\arrayrulecolor{gray} \hline
\multirow{3}{*}{	\textbf{\textit{	syaw	}}}	&	\multirow{3}{*}{	Qual	}	&	\multirow{3}{*}{	small; less	}	&	\multirow{3}{*}{	Sinitic	}	&	\multirow{	3	}{*}{	\textit{		}		小, 少		}	\\&&&&				\textit{		}					\\&&&&	\textit{		}					\\\arrayrulecolor{gray} \hline
\multirow{3}{*}{	\textbf{\textit{	syawgay	}}}	&	\multirow{3}{*}{	Action	}	&	\multirow{3}{*}{	introduce	}	&	\multirow{3}{*}{	Sinitic	}	&	\multirow{	3	}{*}{	\textit{		}		紹介		}	\\&&&&				\textit{		}					\\&&&&	\textit{		}					\\\arrayrulecolor{gray} \hline
\multirow{3}{*}{	\textbf{\textit{	syawnen	}}}	&	\multirow{3}{*}{	O/C	}	&	\multirow{3}{*}{	boy	}	&	\multirow{3}{*}{	Sinitic	}	&	\multirow{	3	}{*}{	\textit{		}		少年		}	\\&&&&				\textit{		}					\\&&&&	\textit{		}					\\\arrayrulecolor{gray} \hline
\multirow{3}{*}{	\textbf{\textit{	syawnyo	}}}	&	\multirow{3}{*}{	O/C	}	&	\multirow{3}{*}{	girl	}	&	\multirow{3}{*}{	Sinitic	}	&	\multirow{	3	}{*}{	\textit{		}		少女		}	\\&&&&				\textit{		}					\\&&&&	\textit{		}					\\\arrayrulecolor{gray} \hline
\multirow{3}{*}{	\textbf{\textit{	syem	}}}	&	\multirow{3}{*}{	O/C	}	&	\multirow{3}{*}{	island	}	&	\multirow{3}{*}{	Koreo-Japonic	}	&	\multirow{	2	}{*}{	\textit{	ko	 - }		섬		}	\\&&&&	\multirow{	2	}{*}{	\textit{	ja	 - }		しま		}	\\&&&&	\textit{		}					\\\arrayrulecolor{gray} \hline
\multirow{3}{*}{	\textbf{\textit{	syensi	}}}	&	\multirow{3}{*}{	O/C	}	&	\multirow{3}{*}{	science	}	&	\multirow{3}{*}{	Western: English	}	&	\multirow{	3	}{*}{	\textit{	en	 - }		science		}	\\&&&&				\textit{		}					\\&&&&	\textit{		}					\\\arrayrulecolor{gray} \hline
\multirow{3}{*}{	\textbf{\textit{	syoka	}}}	&	\multirow{3}{*}{	O/C	}	&	\multirow{3}{*}{	salt	}	&	\multirow{3}{*}{	Koreo-Japonic	}	&	\multirow{	2	}{*}{	\textit{	ko	 - }		소금		}	\\&&&&	\multirow{	2	}{*}{	\textit{	ja	 - }		しお		}	\\&&&&	\textit{		}					\\\arrayrulecolor{gray} \hline
\multirow{3}{*}{	\textbf{\textit{	syokamwisu	}}}	&	\multirow{3}{*}{	O/C	}	&	\multirow{3}{*}{	unami fermented sauces (fish sauce, soy sauce, etc.)	}	&	\multirow{3}{*}{	Compound	}	&	\multirow{	3	}{*}{	\textit{		}				}	\\&&&&				\textit{		}					\\&&&&	\textit{		}					\\\arrayrulecolor{gray} \hline
\multirow{3}{*}{	\textbf{\textit{	syonghol	}}}	&	\multirow{3}{*}{	O/C	}	&	\multirow{3}{*}{	bird of prey	}	&	\multirow{3}{*}{	Altaic	}	&	\multirow{	3	}{*}{	\textit{	mn	 - }		шонхор		}	\\&&&&				\textit{		}					\\&&&&	\textit{		}					\\\arrayrulecolor{gray} \hline
\multirow{3}{*}{	\textbf{\textit{	syow	}}}	&	\multirow{3}{*}{	O/C	}	&	\multirow{3}{*}{	milk	}	&	\multirow{3}{*}{	Compound	}	&				\textit{	ko	 - }		젖			\\&&&&				\textit{	vt	 - }		sữa			\\&&&&	\textit{	mn	 - }		сүү 			\\\arrayrulecolor{gray} \hline
\multirow{3}{*}{	\textbf{\textit{	syowcu	}}}	&	\multirow{3}{*}{	Action	}	&	\multirow{3}{*}{	receive	}	&	\multirow{3}{*}{	Sinitic	}	&	\multirow{	3	}{*}{	\textit{		}		受取		}	\\&&&&				\textit{		}					\\&&&&	\textit{		}					\\\arrayrulecolor{gray} \hline
\multirow{3}{*}{	\textbf{\textit{	syowep	}}}	&	\multirow{3}{*}{		}	&	\multirow{3}{*}{	lesson, a class	}	&	\multirow{3}{*}{	Sinitic	}	&	\multirow{	3	}{*}{	\textit{		}		授業		}	\\&&&&				\textit{		}					\\&&&&	\textit{		}					\\\arrayrulecolor{gray} \hline
\multirow{3}{*}{	\textbf{\textit{	syumen	}}}	&	\multirow{3}{*}{	Action	}	&	\multirow{3}{*}{	sleep	}	&	\multirow{3}{*}{	Sinitic	}	&	\multirow{	3	}{*}{	\textit{		}		睡眠		}	\\&&&&				\textit{		}					\\&&&&	\textit{		}					\\\arrayrulecolor{gray} \hline
\multirow{3}{*}{	\textbf{\textit{	syumen-wan-in	}}}	&	\multirow{3}{*}{	Qual	}	&	\multirow{3}{*}{	tired, sleepy	}	&	\multirow{3}{*}{	Sinitic	}	&	\multirow{	3	}{*}{	\textit{		}		睡眠願		}	\\&&&&				\textit{		}					\\&&&&	\textit{		}					\\\arrayrulecolor{gray} \hline
\multirow{3}{*}{	\textbf{\textit{	syunso	}}}	&	\multirow{3}{*}{	O/C	}	&	\multirow{3}{*}{	order (in a series)	}	&	\multirow{3}{*}{	Sinitic	}	&	\multirow{	3	}{*}{	\textit{		}		順序		}	\\&&&&				\textit{		}					\\&&&&	\textit{		}					\\\arrayrulecolor{gray} \hline
\multirow{3}{*}{	\textbf{\textit{	syuweng	}}}	&	\multirow{3}{*}{	Action	}	&	\multirow{3}{*}{	swim	}	&	\multirow{3}{*}{	Sinitic	}	&	\multirow{	3	}{*}{	\textit{		}		水泳		}	\\&&&&				\textit{		}					\\&&&&	\textit{		}					\\\arrayrulecolor{gray} \hline
\multirow{3}{*}{	\textbf{\textit{	tabako	}}}	&	\multirow{3}{*}{	O/C	}	&	\multirow{3}{*}{	cigarette, tabaco	}	&	\multirow{3}{*}{	Western: Spanish	}	&	\multirow{	2	}{*}{	\textit{	es	 - }		tabaco		}	\\&&&&	\multirow{	2	}{*}{	\textit{		}		(indirect loan via tl)		}	\\&&&&	\textit{		}					\\\arrayrulecolor{gray} \hline
\multirow{3}{*}{	\textbf{\textit{	tagalog'o	}}}	&	\multirow{3}{*}{	O/C	}	&	\multirow{3}{*}{	tagalog	}	&	\multirow{3}{*}{	Compound	}	&	\multirow{	3	}{*}{	\textit{		}				}	\\&&&&				\textit{		}					\\&&&&	\textit{		}					\\\arrayrulecolor{gray} \hline
\multirow{3}{*}{	\textbf{\textit{	tahi	}}}	&	\multirow{3}{*}{	O/C	}	&	\multirow{3}{*}{	time	}	&	\multirow{3}{*}{	Koreo-Japonic	}	&	\multirow{	2	}{*}{	\textit{	ko	 - }		때		}	\\&&&&	\multirow{	2	}{*}{	\textit{	ja	 - }		とき		}	\\&&&&	\textit{		}					\\\arrayrulecolor{gray} \hline
\multirow{3}{*}{	\textbf{\textit{	tahi-tahi	}}}	&	\multirow{3}{*}{	O/C	}	&	\multirow{3}{*}{	sometimes	}	&	\multirow{3}{*}{	Compound: Koreo-Japonic	}	&	\multirow{	3	}{*}{	\textit{		}				}	\\&&&&				\textit{		}					\\&&&&	\textit{		}					\\\arrayrulecolor{gray} \hline
\multirow{3}{*}{	\textbf{\textit{	tahi-tahi-dek	}}}	&	\multirow{3}{*}{	Qual	}	&	\multirow{3}{*}{	often, habitually	}	&	\multirow{3}{*}{	Compound: Koreo-Japonic	}	&	\multirow{	3	}{*}{	\textit{		}				}	\\&&&&				\textit{		}					\\&&&&	\textit{		}					\\\arrayrulecolor{gray} \hline
\multirow{3}{*}{	\textbf{\textit{	tahon	}}}	&	\multirow{3}{*}{	O/C	}	&	\multirow{3}{*}{	year	}	&	\multirow{3}{*}{	Austronesian	}	&	\multirow{	3	}{*}{	\textit{		}				}	\\&&&&				\textit{		}					\\&&&&	\textit{		}					\\\arrayrulecolor{gray} \hline
\multirow{3}{*}{	\textbf{\textit{	taksi	}}}	&	\multirow{3}{*}{	O/C	}	&	\multirow{3}{*}{	taxi	}	&	\multirow{3}{*}{	Western: English	}	&	\multirow{	3	}{*}{	\textit{	en	 - }		taxi		}	\\&&&&				\textit{		}					\\&&&&	\textit{		}					\\\arrayrulecolor{gray} \hline
\multirow{3}{*}{	\textbf{\textit{	taksin	}}}	&	\multirow{3}{*}{	Qual	}	&	\multirow{3}{*}{	south	}	&	\multirow{3}{*}{	Sanskrit	}	&	\multirow{	2	}{*}{	\textit{		}	\textsanskrit{	दक्षिण 	}	}	\\&&&&	\multirow{	2	}{*}{	\textit{		}		(dákṣiṇa)		}	\\&&&&	\textit{		}					\\\arrayrulecolor{gray} \hline
\multirow{3}{*}{	\textbf{\textit{	takut	}}}	&	\multirow{3}{*}{	Qual	}	&	\multirow{3}{*}{	afraid, scared	}	&	\multirow{3}{*}{	Austronesian	}	&	\multirow{	3	}{*}{	\textit{	ms/id	 - }		takut		}	\\&&&&				\textit{		}					\\&&&&	\textit{		}					\\\arrayrulecolor{gray} \hline
\multirow{3}{*}{	\textbf{\textit{	talang	}}}	&	\multirow{3}{*}{	O/C	}	&	\multirow{3}{*}{	table	}	&	\multirow{3}{*}{	Austroasiatic	}	&	\multirow{	2	}{*}{	\textit{	th	 - }	\textthai{	ตาราง	}	}	\\&&&&	\multirow{	2	}{*}{	\textit{	km	 - }	\textkhmer{	តារាង	}	}	\\&&&&	\textit{		}					\\\arrayrulecolor{gray} \hline
\multirow{3}{*}{	\textbf{\textit{	tana	}}}	&	\multirow{3}{*}{	O/C	}	&	\multirow{3}{*}{	earth, soil	}	&	\multirow{3}{*}{	Austronesian	}	&	\multirow{	3	}{*}{	\textit{	ms/id	 - }		tanah		}	\\&&&&				\textit{		}					\\&&&&	\textit{		}					\\\arrayrulecolor{gray} \hline
\multirow{3}{*}{	\textbf{\textit{	tanaboti	}}}	&	\multirow{3}{*}{	O/C	}	&	\multirow{3}{*}{	Saturn	}	&	\multirow{3}{*}{	Compound	}	&	\multirow{	3	}{*}{	\textit{		}		tana + boti		}	\\&&&&				\textit{		}					\\&&&&	\textit{		}					\\\arrayrulecolor{gray} \hline
\multirow{3}{*}{	\textbf{\textit{	tanahali	}}}	&	\multirow{3}{*}{	O/C	}	&	\multirow{3}{*}{	saturday	}	&	\multirow{3}{*}{	Compound	}	&	\multirow{	3	}{*}{	\textit{		}		tana + hali		}	\\&&&&				\textit{		}					\\&&&&	\textit{		}					\\\arrayrulecolor{gray} \hline
\multirow{3}{*}{	\textbf{\textit{	tangsi	}}}	&	\multirow{3}{*}{	O/C	}	&	\multirow{3}{*}{	spoon	}	&	\multirow{3}{*}{	Sinitic	}	&	\multirow{	3	}{*}{	\textit{		}		湯匙		}	\\&&&&				\textit{		}					\\&&&&	\textit{		}					\\\arrayrulecolor{gray} \hline
\multirow{3}{*}{	\textbf{\textit{	tanseng	}}}	&	\multirow{3}{*}{	Action	}	&	\multirow{3}{*}{	to be born; birth	}	&	\multirow{3}{*}{	Sinitic	}	&	\multirow{	3	}{*}{	\textit{		}		誕生		}	\\&&&&				\textit{		}					\\&&&&	\textit{		}					\\\arrayrulecolor{gray} \hline
\multirow{3}{*}{	\textbf{\textit{	tasik	}}}	&	\multirow{3}{*}{	O/C	}	&	\multirow{3}{*}{	lake; inland sea	}	&	\multirow{3}{*}{	Austronesian	}	&	\multirow{	3	}{*}{	\textit{	ms/id	 - }		tasik		}	\\&&&&				\textit{		}					\\&&&&	\textit{		}					\\\arrayrulecolor{gray} \hline
\multirow{3}{*}{	\textbf{\textit{	tay	}}}	&	\multirow{3}{*}{	O/C	}	&	\multirow{3}{*}{	hands	}	&	\multirow{3}{*}{	Austroasiatic	}	&	\multirow{	3	}{*}{	\textit{	vi	 - }		tay		}	\\&&&&				\textit{		}					\\&&&&	\textit{		}					\\\arrayrulecolor{gray} \hline
\multirow{3}{*}{	\textbf{\textit{	tay'o	}}}	&	\multirow{3}{*}{	O/C	}	&	\multirow{3}{*}{	thai (language)	}	&	\multirow{3}{*}{	Compound	}	&	\multirow{	3	}{*}{	\textit{		}				}	\\&&&&				\textit{		}					\\&&&&	\textit{		}					\\\arrayrulecolor{gray} \hline
\multirow{3}{*}{	\textbf{\textit{	tayfung	}}}	&	\multirow{3}{*}{	O/C	}	&	\multirow{3}{*}{	storm	}	&	\multirow{3}{*}{	Sinitic	}	&	\multirow{	3	}{*}{	\textit{		}		颱風		}	\\&&&&				\textit{		}					\\&&&&	\textit{		}					\\\arrayrulecolor{gray} \hline
\multirow{3}{*}{	\textbf{\textit{	tayko	}}}	&	\multirow{3}{*}{	O/C	}	&	\multirow{3}{*}{	fingers	}	&	\multirow{3}{*}{	Compound	}	&	\multirow{	3	}{*}{	\textit{		}				}	\\&&&&				\textit{		}					\\&&&&	\textit{		}					\\\arrayrulecolor{gray} \hline
\multirow{3}{*}{	\textbf{\textit{	teley	}}}	&	\multirow{3}{*}{	Action	}	&	\multirow{3}{*}{	carry, bring, move (something)	}	&	\multirow{3}{*}{	Koreo-Japonic	}	&	\multirow{	2	}{*}{	\textit{	ko	 - }		데리다		}	\\&&&&	\multirow{	2	}{*}{	\textit{	ja	 - }		つれる		}	\\&&&&	\textit{		}					\\\arrayrulecolor{gray} \hline
\multirow{3}{*}{	\textbf{\textit{	telinga	}}}	&	\multirow{3}{*}{	O/C	}	&	\multirow{3}{*}{	ears	}	&	\multirow{3}{*}{	Austronesian	}	&	\multirow{	3	}{*}{	\textit{	ms/id	 - }		telinga		}	\\&&&&				\textit{		}					\\&&&&	\textit{		}					\\\arrayrulecolor{gray} \hline
\multirow{3}{*}{	\textbf{\textit{	teluk	}}}	&	\multirow{3}{*}{	O/C	}	&	\multirow{3}{*}{	bay	}	&	\multirow{3}{*}{	Austronesian	}	&	\multirow{	3	}{*}{	\textit{	ms/id	 - }		teluk		}	\\&&&&				\textit{		}					\\&&&&	\textit{		}					\\\arrayrulecolor{gray} \hline
\multirow{3}{*}{	\textbf{\textit{	tenki	}}}	&	\multirow{3}{*}{		}	&	\multirow{3}{*}{	weather	}	&	\multirow{3}{*}{	Sinitic	}	&	\multirow{	3	}{*}{	\textit{		}		天氣		}	\\&&&&				\textit{		}					\\&&&&	\textit{		}					\\\arrayrulecolor{gray} \hline
\multirow{3}{*}{	\textbf{\textit{	tensi	}}}	&	\multirow{3}{*}{	O/C	}	&	\multirow{3}{*}{	angel; agent of the order and good	}	&	\multirow{3}{*}{	Sinitic	}	&	\multirow{	3	}{*}{	\textit{		}		天使		}	\\&&&&				\textit{		}					\\&&&&	\textit{		}					\\\arrayrulecolor{gray} \hline
\multirow{3}{*}{	\textbf{\textit{	teyjung	}}}	&	\multirow{3}{*}{	Qual	}	&	\multirow{3}{*}{	weight	}	&	\multirow{3}{*}{	Sinitic	}	&	\multirow{	3	}{*}{	\textit{		}		體重		}	\\&&&&				\textit{		}					\\&&&&	\textit{		}					\\\arrayrulecolor{gray} \hline
\multirow{3}{*}{	\textbf{\textit{	tiku	}}}	&	\multirow{3}{*}{	O/C	}	&	\multirow{3}{*}{	mouse, rat	}	&	\multirow{3}{*}{	Austronesian	}	&	\multirow{	3	}{*}{	\textit{	ms/id	 - }		tikus		}	\\&&&&				\textit{		}					\\&&&&	\textit{		}					\\\arrayrulecolor{gray} \hline
\multirow{3}{*}{	\textbf{\textit{	tofu	}}}	&	\multirow{3}{*}{	O/C	}	&	\multirow{3}{*}{	tofu	}	&	\multirow{3}{*}{	Sinitic	}	&	\multirow{	3	}{*}{	\textit{		}		豆腐		}	\\&&&&				\textit{		}					\\&&&&	\textit{		}					\\\arrayrulecolor{gray} \hline
\multirow{3}{*}{	\textbf{\textit{	tok (fway)	}}}	&	\multirow{3}{*}{	Action	}	&	\multirow{3}{*}{	deep-fry	}	&	\multirow{3}{*}{	Compound	}	&	\multirow{	3	}{*}{	\textit{		}				}	\\&&&&				\textit{		}					\\&&&&	\textit{		}					\\\arrayrulecolor{gray} \hline
\multirow{3}{*}{	\textbf{\textit{	toli	}}}	&	\multirow{3}{*}{	O/C	}	&	\multirow{3}{*}{	bird, fowl	}	&	\multirow{3}{*}{	Koreo-Japonic	}	&	\multirow{	3	}{*}{	\textit{	ja	 - }		とり		}	\\&&&&				\textit{		}					\\&&&&	\textit{		}					\\\arrayrulecolor{gray} \hline
\multirow{3}{*}{	\textbf{\textit{	tosu	}}}	&	\multirow{3}{*}{	Action	}	&	\multirow{3}{*}{	fight	}	&	\multirow{3}{*}{	Austroasiatic	}	&	\multirow{	2	}{*}{	\textit{	th	 - }	\textthai{	ต่อสู้	}	}	\\&&&&	\multirow{	2	}{*}{	\textit{	lo	 - }	\textlao{	ຕໍ່ສູ້ 	}	}	\\&&&&	\textit{		}					\\\arrayrulecolor{gray} \hline
\multirow{3}{*}{	\textbf{\textit{	tow'meng	}}}	&	\multirow{3}{*}{	Qual	}	&	\multirow{3}{*}{	clear, transparent	}	&	\multirow{3}{*}{	Sinitic	}	&	\multirow{	3	}{*}{	\textit{		}		透明		}	\\&&&&				\textit{		}					\\&&&&	\textit{		}					\\\arrayrulecolor{gray} \hline
\multirow{3}{*}{	\textbf{\textit{	towsagi	}}}	&	\multirow{3}{*}{	O/C	}	&	\multirow{3}{*}{	rabbit / hare	}	&	\multirow{3}{*}{	Koreo-Japonic	}	&	\multirow{	2	}{*}{	\textit{	ko	 - }		토끼		}	\\&&&&	\multirow{	2	}{*}{	\textit{	ja	 - }		うさぎ		}	\\&&&&	\textit{		}					\\\arrayrulecolor{gray} \hline
\multirow{3}{*}{	\textbf{\textit{	tuk	}}}	&	\multirow{3}{*}{	Qual	}	&	\multirow{3}{*}{	cheap	}	&	\multirow{3}{*}{	Austroasiatic	}	&	\multirow{	2	}{*}{	\textit{	th	 - }	\textthai{	ถูก	}	}	\\&&&&	\multirow{	2	}{*}{	\textit{	lo	 - }	\textlao{	ຖືກ	}	}	\\&&&&	\textit{		}					\\\arrayrulecolor{gray} \hline
\multirow{3}{*}{	\textbf{\textit{	tya	}}}	&	\multirow{3}{*}{	O/C	}	&	\multirow{3}{*}{	tea	}	&	\multirow{3}{*}{	Sinitic	}	&	\multirow{	3	}{*}{	\textit{		}		茶		}	\\&&&&				\textit{		}					\\&&&&	\textit{		}					\\\arrayrulecolor{gray} \hline
\multirow{3}{*}{	\textbf{\textit{	uhe	}}}	&	\multirow{3}{*}{	O/C	}	&	\multirow{3}{*}{	cow	}	&	\multirow{3}{*}{	Altaic	}	&	\multirow{	3	}{*}{	\textit{	mn	 - }		үхэр		}	\\&&&&				\textit{		}					\\&&&&	\textit{		}					\\\arrayrulecolor{gray} \hline
\multirow{3}{*}{	\textbf{\textit{	unci	}}}	&	\multirow{3}{*}{	O/C	}	&	\multirow{3}{*}{	poop	}	&	\multirow{3}{*}{	Koreo-Japonic	}	&	\multirow{	3	}{*}{	\textit{	ja	 - }		うんち		}	\\&&&&				\textit{		}					\\&&&&	\textit{		}					\\\arrayrulecolor{gray} \hline
\multirow{3}{*}{	\textbf{\textit{	uncikaki	}}}	&	\multirow{3}{*}{	O/C	}	&	\multirow{3}{*}{	ankle biters; bad children	}	&	\multirow{3}{*}{	Compound	}	&	\multirow{	3	}{*}{	\textit{		}				}	\\&&&&				\textit{		}					\\&&&&	\textit{		}					\\\arrayrulecolor{gray} \hline
\multirow{3}{*}{	\textbf{\textit{	unjen	}}}	&	\multirow{3}{*}{	Action	}	&	\multirow{3}{*}{	drive (e.g. a car)	}	&	\multirow{3}{*}{	Sinitic	}	&	\multirow{	3	}{*}{	\textit{		}		運轉		}	\\&&&&				\textit{		}					\\&&&&	\textit{		}					\\\arrayrulecolor{gray} \hline
\multirow{3}{*}{	\textbf{\textit{	utala	}}}	&	\multirow{3}{*}{	Qual	}	&	\multirow{3}{*}{	north	}	&	\multirow{3}{*}{	Sanskrit	}	&	\multirow{	2	}{*}{	\textit{		}	\textsanskrit{	उत्तर 	}	}	\\&&&&	\multirow{	2	}{*}{	\textit{		}		(úttara)		}	\\&&&&	\textit{		}					\\\arrayrulecolor{gray} \hline
\multirow{3}{*}{	\textbf{\textit{	uwe	}}}	&	\multirow{3}{*}{	Qual	}	&	\multirow{3}{*}{	up	}	&	\multirow{3}{*}{	Koreo-Japonic	}	&	\multirow{	2	}{*}{	\textit{	ko	 - }		위/우에		}	\\&&&&	\multirow{	2	}{*}{	\textit{	ja	 - }		うえ		}	\\&&&&	\textit{		}					\\\arrayrulecolor{gray} \hline
\multirow{3}{*}{	\textbf{\textit{	wajaw	}}}	&	\multirow{3}{*}{	Action	}	&	\multirow{3}{*}{	smile; laugh	}	&	\multirow{3}{*}{	Koreo-Japonic	}	&	\multirow{	2	}{*}{	\textit{	ko	 - }		웃음		}	\\&&&&	\multirow{	2	}{*}{	\textit{	jp	 - }		わらう		}	\\&&&&	\textit{		}					\\\arrayrulecolor{gray} \hline
\multirow{3}{*}{	\textbf{\textit{	wak	}}}	&	\multirow{3}{*}{	O/C	}	&	\multirow{3}{*}{	wok	}	&	\multirow{3}{*}{	Sinitic	}	&	\multirow{	3	}{*}{	\textit{		}		鑊		}	\\&&&&				\textit{		}					\\&&&&	\textit{		}					\\\arrayrulecolor{gray} \hline
\multirow{3}{*}{	\textbf{\textit{	wan	}}}	&	\multirow{3}{*}{	Action	}	&	\multirow{3}{*}{	want	}	&	\multirow{3}{*}{	Sinitic	}	&	\multirow{	3	}{*}{	\textit{		}		願		}	\\&&&&				\textit{		}					\\&&&&	\textit{		}					\\\arrayrulecolor{gray} \hline
\multirow{3}{*}{	\textbf{\textit{	wankuk	}}}	&	\multirow{3}{*}{	Action	}	&	\multirow{3}{*}{	bend	}	&	\multirow{3}{*}{	Sinitic	}	&	\multirow{	3	}{*}{	\textit{		}		彎曲		}	\\&&&&				\textit{		}					\\&&&&	\textit{		}					\\\arrayrulecolor{gray} \hline
\multirow{3}{*}{	\textbf{\textit{	watnam'o	}}}	&	\multirow{3}{*}{	O/C	}	&	\multirow{3}{*}{	vietnamese (language)	}	&	\multirow{3}{*}{	Sinitic	}	&	\multirow{	3	}{*}{	\textit{		}		越南語		}	\\&&&&				\textit{		}					\\&&&&	\textit{		}					\\\arrayrulecolor{gray} \hline
\multirow{3}{*}{	\textbf{\textit{	wihem	}}}	&	\multirow{3}{*}{	Qual	}	&	\multirow{3}{*}{	danger	}	&	\multirow{3}{*}{	Sinitic	}	&	\multirow{	3	}{*}{	\textit{		}		危險		}	\\&&&&				\textit{		}					\\&&&&	\textit{		}					\\\arrayrulecolor{gray} \hline
\multirow{3}{*}{	\textbf{\textit{	wihem	}}}	&	\multirow{3}{*}{	Qual	}	&	\multirow{3}{*}{	dangerous	}	&	\multirow{3}{*}{	Sinitic	}	&	\multirow{	3	}{*}{	\textit{		}		危險		}	\\&&&&				\textit{		}					\\&&&&	\textit{		}					\\\arrayrulecolor{gray} \hline
\multirow{3}{*}{	\textbf{\textit{	winyan	}}}	&	\multirow{3}{*}{	O/C	}	&	\multirow{3}{*}{	soul	}	&	\multirow{3}{*}{	Sanskrit	}	&	\multirow{	2	}{*}{	\textit{		}	\textsanskrit{	विज्ञान 	}	}	\\&&&&	\multirow{	2	}{*}{	\textit{		}		(vijñāna)		}	\\&&&&	\textit{		}					\\\arrayrulecolor{gray} \hline
\multirow{3}{*}{	\textbf{\textit{	witasat	}}}	&	\multirow{3}{*}{	O/C	}	&	\multirow{3}{*}{	science 	}	&	\multirow{3}{*}{	Sanskrit	}	&	\multirow{	2	}{*}{	\textit{		}	\textsanskrit{	विद्या + शास्त्र 	}	}	\\&&&&	\multirow{	2	}{*}{	\textit{		}		(vidyā + śāstra) 		}	\\&&&&	\textit{		}					\\\arrayrulecolor{gray} \hline
\multirow{3}{*}{	\textbf{\textit{	yacay	}}}	&	\multirow{3}{*}{	O/C	}	&	\multirow{3}{*}{	science 	}	&	\multirow{3}{*}{	Sinitic	}	&	\multirow{	3	}{*}{	\textit{		}		野菜		}	\\&&&&				\textit{		}					\\&&&&	\textit{		}					\\\arrayrulecolor{gray} \hline
\multirow{3}{*}{	\textbf{\textit{	yaksuk	}}}	&	\multirow{3}{*}{	Action	}	&	\multirow{3}{*}{	promise	}	&	\multirow{3}{*}{	Sinitic	}	&	\multirow{	3	}{*}{	\textit{		}		約束		}	\\&&&&				\textit{		}					\\&&&&	\textit{		}					\\\arrayrulecolor{gray} \hline
\multirow{3}{*}{	\textbf{\textit{	yame	}}}	&	\multirow{3}{*}{	O/C	}	&	\multirow{3}{*}{	mountain	}	&	\multirow{3}{*}{	Koreo-Japonic	}	&	\multirow{	2	}{*}{	\textit{	ko	 - }		뫼		}	\\&&&&	\multirow{	2	}{*}{	\textit{	ja	 - }		やま		}	\\&&&&	\textit{		}					\\\arrayrulecolor{gray} \hline
\multirow{3}{*}{	\textbf{\textit{	yaway	}}}	&	\multirow{3}{*}{	O/C	}	&	\multirow{3}{*}{	outdoors, outside	}	&	\multirow{3}{*}{	Sinitic	}	&	\multirow{	3	}{*}{	\textit{		}		野外		}	\\&&&&				\textit{		}					\\&&&&	\textit{		}					\\\arrayrulecolor{gray} \hline
\multirow{3}{*}{	\textbf{\textit{	yowsyaw	}}}	&	\multirow{3}{*}{	Qual	}	&	\multirow{3}{*}{	young	}	&	\multirow{3}{*}{	Sinitic	}	&	\multirow{	3	}{*}{	\textit{		}		幼少		}	\\&&&&				\textit{		}					\\&&&&	\textit{		}					\\\arrayrulecolor{gray} \hline
\multirow{3}{*}{	\textbf{\textit{	yucik	}}}	&	\multirow{3}{*}{	Qual	}	&	\multirow{3}{*}{	right (direction)	}	&	\multirow{3}{*}{	Sinitic	}	&	\multirow{	3	}{*}{	\textit{		}		右		}	\\&&&&				\textit{		}					\\&&&&	\textit{		}					\\\arrayrulecolor{gray} \hline
\end{longtable}
 
\end{document}
